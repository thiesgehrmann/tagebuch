\subsection{25. II. 1942}
%\mktitle

Meine Hilde!
Es ist heute 21. Geburtstag, der Tag Deiner Grossjahrigkeit.
Ich muss sagen, dass ich in jeder Hinsicht bei den gegebenen Verh\"{a}ltnissen, die allem, wass irgendwie gelebtes leben und Pers\"{o}nlicheit heist den lebhaftesten Widerstand entgegen setzen, dennoch diesen Tag f\"{u}r mich "festlich" gegangen habe.
Gestern schon putzte ich meine s\"{a}mtlichen Schuhe - immerhin 3 Paar und eine Leistung - heute legte ich dann meine Uniform an und lies von mir aus jeglichen Unterricht und Unterrichtung fallen.
Zum Nachmittag gab es Kaffee mit Sahne, jawohl, gefiltelter Bohnenkaffee aus Tassen, allerdings auf Prisonerart: Blechaschbecher und in Konservendosen aufbereitet.
So sitze ich Jetzt am Abend und denke nur {\color{red} [an] } Dich, all meine Gedanken rufen Deinen Lieben namen, ich sehe Dich mit meinem Blumengruss in der hand von 21 roten Rosen mitten im kalten Winter. -

Und nun stehe ich da mit meinem Talent, ein armer Prisoner, ein eingepferchtes Wesen, aus blindem Hass aller, aber auch aller Freiheit und menschlichen Bindung beraubt und entrissen. -

Du bist meine Braut, sollst einmal die Mutter meiner Kinder sein.
Und wass machst du?
Stellst mich grad wie mein lieber, kleiner, dummer, Peter einmal.
Nummerierst Deine Briefe auf die unm\"{o}glichsten Art, schreibst in der unbek\"{u}mmertsten Weise von Deinen kleinen t\"{a}glichen Dingen, stellst mir nur Euer unbesorgtes Leben in den rosigsten Farben vor mich.
Da jagt eine Feier die andere, aber von keinem Buch sprichst Du, Verlobungen vermeldest Du mir p\"{u}nktlich nicht ohne Seufzer, aber mein Kontostand l\"{a}sst auf sich warten.
Da beginnen zwei Briefe mit Ausfl\"{u}chten, dass Du keine Zeit zu einen Brief f\"{u}r mich fandest und all Dein Empfinden f\"{u}r mich gibfelt in der Feststellung, die schade es ist, dass ich nicht mit Dir Feiern kann.

\clearpage
