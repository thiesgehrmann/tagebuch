\subsection{7{\color{red} [/8] }.3.1942}
%\mktitle

Der 1. sogennante oder auch bunte Abend im neuen Camp in Canada steigt heute.
Ich bin nicht hingegangen.
Hilde, Du Liebste, ich m\"{o}chte dir t\"{a}glich schreiben, aber ach - ich kann nicht.
Ich kann es nicht.
Ob Du jesmals{\color{red} [sic, jemals] } mich darum verstehen wirst?
- Du machst mir mit deinen lieben dummen Briefen dieses Gefangenenleben um vieles schwerer.
Ich darf Dir nur die wenigen Zeilen und die lesen Zensoren, aber nicht nur dass diese mir g\"{a}nzend fremden und gleichg\"{u}ltigen Menschen sich da hineindr\"{a}ngen, auch unter uns m\"{u}ssen sich Menschen finden, die meine Briefe zu zensieren sich erdreisten.
Es soll angeblich notwendig sein!
- Da ist wieder der Kampf.
Masse, Gemeinschaft, Pers\"{o}nlichkeit. -

Liebste, in meinen Briefen zu Dir habe ich ja einzig Gelebt! -

Ja, es ist so unendlich schwer, in einer solch zusammengew\"{u}rfelten Masse, wie diese Prisoner \ul{eine} Haltung zu bringen.
Zum bunten Abend l\"{a}dt man kanadische Offiziere ein
{\color{red} [ Durchgestrichen und ausradiert: ] } "damit sie, einmal sehen, wie primitiv wir uns behelfen m\"{u}ssen".

Grade heute f\"{a}llt mir ein Aufsatz aus einer Tageszeitung in die Finger \"{u}ber den letzten Gefangenentransport von England nach Kanada.
{\color{red} [ Durchgestrichen und ausradiert:] }
Diese Zweilen haben mir wieder gen\"{u}gend gesaft.

Wass hilft uns die ... ???
{\color{red} [ 4 weitere Zeilen sind unleserlich ] }

\clearpage
