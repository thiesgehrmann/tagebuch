\subsection{17.3.1942}
%\mktitle

Graublau, schwer Regen- und Schneewolkenverhangen zieht der Himmel tief \"{u}ber die Erde dahin.
Nasskalte Winde peitschen dichte Flockenschauer, die bisweilen schon in grossen Tropfen herunterkommen, und im halb starren halb aufgeweichten Lehmboden versinkt man bisweilen kn\"{o}cheltief, stehen die Scheepf\"{u}tzen knietief.

War es denn Wirklichkeit?
War es denn erst gestern?
Es w\"{a}re ja gleich, aber es ist wahr gewesen.
Die goldenen Mittagsstunden konnte man im freien, auf einer Bank hingestreckt verbringen.
Es war wie ein Geschenk des Himmels, der mir sein sch\"{o}nes licht in einem unbeschreiblichen stillen Jubel \st{zu trinken} gab, an dem ich mich satt trinken durfte.
Der Glanz, die festliche Stille, die glitzernde Helle auf dem Schnee.
Du Licht, Du Lebensspenderin, du heissest mich mein Los f\"{u}r kurtze Zeit vergessen. -

Und wie zum g\"{u}tigen Trost, zeigt mein Gesicht auch heute noch, da wieder ein tr\"{u}bseliger Tag uns Prisoner in den Baracken h\"{a}llt, eng einer auf den andern, eine frohe gesunde Farbe.
Also ist doch alles wahr gewesen. -

Am 10. 3. kamen 5 Briefe.
4 von ihr.
Sie schreibt von einen Buch, das eine neue Welt ich er\"{o}ffnet die sie begierig zu Wissen begehrt!
Auch kann sie gar nicht sich auf den Urlaub freuen!
Und manches Liebe noch.
Ach, Hilde k\"{o}nnt \ul{ich} Dir schreiben, \ul{Briefe} schreib{\color{red} [sic] } die leben heissen, ich wollte schon gl\"{u}cklich sein.
Aber auch das nimmt man Gefangenen.
Sie sollen Zerm\"{u}rben!
- Dein Bild steht wieder auf dem Tisch, Dein Gesich{\color{red} [Gesicht] } schaut wieder auf mich herab.
Das Dreiteil mit der Nta{\color{red} [?] }, unserem M\"{a}rchenwald und Dir ist wieder vollz\"{a}hlig.
Da gab es ja auch seit langem etwas pers\"{o}nliches zu schaffen am Sonnabend, als die kleinen Schr\"{a}nkchen, - sprich Regale - kamen.
Unsere Ecke gef\"{a}llt mir nun.
Man m\"{u}sste sie festhalten k\"{o}nnen, auch \"{A}userlich. -

Gestern, Montag, habe ich meine Schularbeit wieder aufgenomen.
Im Revier ist meine Schreibarbeit beendet.
bis zum Sommer wird wohl die Schule noch gut laufen.
F\"{u}r Aussenarbeiten habe ich mich noch nicht entschlossen.
Ein paar kleine "G\"{u}ter" ein winziger Besitz muss{\color{red} [?] } in meine Kantinenbestellung.
eine Aktentasche habe ich bestellt, einen Koffer lasse ich Folgen.
Man f\"{u}hlt aber in allem, Prisoner sein ist hart.
Auch unser Geld sieht sich anders an!
Gut ist, das wir vielf Obst, Milch, Schokolade und viele andere Esswaren billig kaufen k\"{o}nnen.
So ist ein Nachmittg{\color{red} [sic] } mit Bohnenkaffee, Schlagsahne mit Fr\"{u}chten (Ananas, Aprikosen, Pfirsiche Usw.) nicht selten.
F\"{u}r 11{\$} gleich 25 Mark kauft man viel.
Industrie Artikel sind teuer.
- Meine W\"{a}sche ist fertig, das schr\"{a}nkchen gibt ein Eigenes inmitten all der unpers\"{o}nlichen Massengedr\"{a}nktheit.
Meine Zwietschriften beleben auch etwas. -

Das wichtigste ist und bleibt, das etwas geschiet.
Ich gehe den Weg der Selbstbesch\"{a}ftigung weiter. -

Schreiben muss ich, es k\"{o}nnte sein dass mein Brief irgendwie zu sp\"{a}t kommt.
Das w\"{a}re auch nicht recht.
Aber niemand soll mir mein Schweigen nachtragen.
- Das Buch habe ich mit freuden weitergelesen, als ich es zum 2. Mal zur Hand nahm.

\clearpage
