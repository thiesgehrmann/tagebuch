\subsection{19. III. 42.}
%\mktitle

Prisonerleben in Canada: T\"{a}glich wirklich gutes Essen - dass kan man sich selbst mal eingestehen,
{\color{red} [ Eine Zeile unleserlich ausradiert] } -
Nach dem Essen, wenn kein Unterricht ist, ein schl\"{a}fchen; wacht man auf kocht das Kaffeewasser, oftmals macht man sich Schlagsahne mit Fr\"{u}chten, abends Br\"{a}t und backt man - aber was hilft das alles?
Etwas Sport, lange Spazierg\"{a}nge - immer innen im K\"{a}fig herum, dass h\"{a}lt den K\"{o}rper gesund.
Ich f\"{u}hle mich nun mit meinem Gewicht ganz leidlich wohl auf.

Die bef\"{o}rderung von Meinhard Milde warf heute das leidige Geldproblem auf.
Ich machte mir also auch einen neuen \"{U}berschalg auf Grund des neuesten Standes der Forschung.
Jeder weiss ja etwas anderes und keiner was richtiges.

Auf dem Spaziergang dachte ich unwillk\"{u}rlich an meine Abkommandierung vom "Richard Beitzen" zur\"{u}ck.
Wie wenn ich doch auf meinem Weg von Berlin in Kiel noch einmal an Bord gegangen w\"{a}re und mich beim I. abgemendet h\"{a}tte.
Interessant w\"{a}re diese Unterredung auf alle F\"{a}lle geworden - und vielleicht nicht unwichtig f\"{u}r mich.
Doch es ist vorbei!
Und nun bewegen sich all meine Gedanken, so oft sie auf der Suche nach eniem neuen m\"{o}glichen Weg sich aufmachen, in einem vernichtenden, zerm\"{u}rbendem Kreislauf stets zum Ausganspunkt zur\"{u}ck.

\clearpage
