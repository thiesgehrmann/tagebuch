\subsection{19. III. 1942 Fortstzg.}
%\mktitle

Haltung der Deutschen Kriegsgefangenen: Wie hat mich eine unfreiwillig aufgefangene \"{a}userung des Kam. H\"{o}gel - der werdende Kunstzeichner den ich vom Gastenlehrgang herkenne - abgestossen und mein Urteil \"{u}ber ihn von seinem K\"{o}nnen befreit.
{\color{red} [ 9 Zeilen durchgestrichen und ausradiert ] }

Mein Buch "Pauline aus Kreuzburg" habe ich auch ausgelesen.
Ich empfand dabei den Wunsch, meiner eigenen \st{siechen} nun siechen und altersschwachen Grosmutter, die die 85 \"{u}berschritten hat, dieses Buch einmal vorlesen zu d\"{u}rfen und ihr dabei lieb in voller dankbarkeit die alten, von einem Leben harter Pflicht und Arbeit runzligen H\"{a}nge zu striecheln, die auch so viel Leben und Liebe einmal spenden konnten. -

Ich muss schreiben, bevor der M\"{a}rz zu ende geht.
Ich glaube ich habe die Bitterkeit ihrer doch unbedachten Briefe \"{u}berwunden.

Vielleicht schrieb ich sie mir vom Herzen in diesen Zeilen hier.
Und dennoch sage ich, wenn mein geliebtes M\"{a}del daheim nach meiner Heimkehr mich ein einziges nur mit vollem Herzen und Sinn wird verstehen k\"{o}nnen, wie ich diese Gefangenschaft erlebte und durchstehen musste, nur dann kann sie mir die Gef\"{a}hrtin in einem Leben lang sein, und \st{Gl\"{u}ck und} Leid und Freud, und Gl\"{u}ck und Schmerz so miteinander tragen.
\hline

\clearpage
