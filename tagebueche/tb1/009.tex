\subsection{26. III. 42.}
%\mktitle

Wieder sind goldene Sonnentage dahingegangen.
Ich habe sie genutzt und bin dar\"{u}ber mit mir zufrieden.
Wenn ich jetzt nach Hause kommen k\"{o}nnte, sie w\"{u}rden sich alle \"{u}ber mein frisches Aussehen freuen, denn im Gesicht und an den Armen bin ich schon sehr ordentlich gebr\"{a}unt.
Heute konnte ich auch zum 1. Mal ein ganzes Sonnenbad nehmen.
So vergehen mir die Tage.

Jetzt muss ich etwas aus meinem Leben bemerken, wass mich wohl nie wieder verlassen wird, den dazu war meine Neigung zu ihr zu bedeutend f\"{u}r mich - und wohl auch, wie ich es \"{u}ber Meere zu f\"{u}hlen glaube, - f\"{u}r sie!
Margret.
Ich kann ihren Namen nur in Verehrung ausprechen, auch mir denken.
Ich weiss, dass es mir gelungen ist, uns beiden dieses Gef\"{u}hl f\"{u}reinander erhalten zu haben.
Darum war unsere Trennung, der Tag \st{an dem} bis zu dem sie sich zu mir geh\"{o}rig f\"{u}hlte - ich weiss nun dass dazu kein einziges Wort geh\"{o}rt um 2 Menschen zu binden - un an dem ich sie zum letzten Male sah, ein Gewinn dennoch f\"{u}r beide geblieben ist, aber jeden von uns reichgemacht hat.

Es war zu eigenartig, was ich heute dazu erlebte.
Ja war es wirklich erlebt?
Eigentlich war es ja nur eine Empfindung.
Mir schien sie aber so wahr und echt, weil ich so ja nur einmal mit einem Menschen zusammentraf.
Das war sie, Margret.

Aber, es war so, dass ich in der letzten Nacht, so oft ich halbwach wurde, feststellen musste, ich habe von \st{Grete} Margret getr\"{a}umt.
In heftiger Erregung durchlebte ich mir ihr die Scenen, die mich damals vor Jahren in unserem Charlottenburg \st{mit ihr durchmachte} durchgemacht habe.
Doch fand ich sie bereits verheirratet, dass hatte mein Unterbewusstsein mit hineinverarbeitet.
Aber sie hatte f\"{u}r mich genau wie damals diese nicht abweisende, aber ganz und gar von vornherein mit dem "noli me tangere" besagter Haltung.
Seltsamerweise duzte ich sie.
Ist aber unser "Sie" nicht verbundener als manches Du.
- Und doch habe ich sie damals am hellichten Tag im Park vom Charlottenburger Schloss genommen und auf den Mund gek\"{u}sst.
- In meinem Traum erfuhr ich ebenfalls noch von ihr das ihr Gatte Hauptmann sei.
Da bin ich nun auf jeden Fall gespannt, wass daran ist.
Wie sagt Ruth Hoffmann in ihrem Buch "Pauline aus {\Cross}burg": So schnell geheiratet, wie es nur ganz junge oder sehr reife Menschen tun.

Ich k\"{o}nnte mir nicht denken, da Margret mir schreibt, auch ihr Mann sei Soldat, dass sie nicht einen Offizier geheiratet hat.
(Ich behaupte das und weiss es wohlgesetzt zu begr\"{u}nden.)

Am Morgen dann, noch vor dem Fr\"{u}hst\"{u}ck, - f\"{u}r einen Prisoner nicht unbedeutend - errinerte ich mich des Datums: 26. III. 42 - Margrets 22. Geburtstag, der 2. den sie als junge Frau verlebt.
Warum besch\"{a}ftigt sie mich so sehr?

\clearpage
