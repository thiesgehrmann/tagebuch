\subsection{25.8.1942}
%%\mktitle

Dieses Heft beginne Ich heute als Sch\"{o}nschreibheft auf den Rat meines guten Vaters.
Nachdem sie zu Hause seit unserer Verschiffung nach Kanada ein halbes Jahr ohne Nachricht von mir gewesen sind, hat er mir als Trost mit das Sch\"{o}nschreiben ans Hertz gelegt.
Er soll es nicht umsonst getan haben.
Jeden Tag eine Seite Sc\"{o}nschrift wil ich versprechen.
Ich habe ja reichlich Zeit daf\"{u}r und, seit ich meinen eigenen Tisch habe, auch Gelegenheit dazu.
Man kann annehmen, dass k\"{u}nftig nur noch die lateinische Schrift gelehrt wird.
Ich bedauere dies f\"{u}r meinen Teil sehr, aber da ich mir einen verwaltungsm\"{a}sigen Beruf selbst gew\"{a}lht habe, muss ich mich auch in dieser Hinsicht abfinden.
Wenn ich wieder zu Hause bin, dann werde ich f\"{u}r mich sowieso in meinen eigensten Briefen doch nur die Deutsche Schrift gebrauchen.
Nun muss ich mir \"{u}berlegen, welche haupts\"{a}chligsten Fehler meine Schrift aufweist und worin das Merkmahl einer guten Handschrift besteht.
Das Heft reicht f\"{u}r ein halbes Jahr aus.

\clearpage
