\def\day{26.8.42}
\mktitle

Die Stimme des Alten riss Adelheit wieder hart in die Gegenwart zur\"{u}ck:
"Wenn ich begraben bin, solt ihr keine eiserne Platte \"{u}ber mich decken.
Namen und Buchstaben n\"{u}tzen nichts in der Ewigkeit.
Lasst Gras und - Blumen dort wachsen, wo ich liege, und macht ein h\"{o}lzernes Kreuz und setzt oben drauf.
Wenn das zerf\"{a}llt, dann weiss niemand mehr etwas von mir.
Das Holtz daf\"{u}r k\"{o}nnt ihr aus dem Wald hinter der n\"{o}rdlichen Umfriedung der Weidepl\"{a}tze holen, bei dem weissgrauen Stein; denn das war die letzte Stelle, an der ich gesessen und den Wald mir nahe gef\"{u}hlt habe."

Seine Stimme klang bei den letzten worten ein ganz klein wenig unsicher, als Adelheit ihn aber ansah, blinkte nicht eine Spur von Feuchtichkeit in seinen Augen, sondern nur der starre Fieberglan{\ss}, und er sass mit der gleichen festen Haltung in den Kissen.
Sie hatte erz\"{a}hlen h\"{o}ren, dass Bauernfrauen noch im Todeskampfe Kleider und Schmuck unter ihre T\"{o}chter verteilen k\"{o}nten - ruhig, wie eine alt\"{a}gliche besch\"{a}ftigung.
aus und Schluss.

\clearpage
