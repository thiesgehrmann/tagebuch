\subsection{28.8.1942}
%\mktitle

So das Signum rechts oben, von D\"{u}rers Hand mit dem selben St\"{u}ck Kohle hingeschrieben, mit dem die Zeichnung gefertigt wurde, F\"{u}nfhundert Jahre schaut die Mutter D\"{u}rers stummen Blickes aus der Zerbrechligkeit und aus der Zerbrechligkeit ihres zerm\"{u}rbten lebens in die nahe Tr\"{o}stung der ewigkeit.
Dieser Blick aber ist die Gewalt und Ewigkeit des Bildes selbts.
Was ist sonst gross daran?
Eine alte, zerarbeitete Frau, die Stirn voller Sorgenfalten, Mund und Wangen ein gefallen, die Lippen eingefallen und verdorrt fast blossgelegt scheinen die Sehnen an dem hageren Halse.
Das \"{a}rmliche Kopftuch h\"{a}ngt wie es will, und l\"{a}sst das karge, streng zur\"{u}ckgestrichene Haar nur \"{u}ber der Stirne frei.
Barbara Holger hat sie geheissen, die D\"{u}rers mutter war.
Der 40j\"{a}hrige Vater hatte sie, die 16 Jahre alt war und "eine h\"{u}bsche grade Jungfrau", als Die Tochter seines N\"{u}rnberger Meisters 1467 geheiratet.
Zwischen Arbeit und Sorgen gebar sie ihm 18 Kinder.
Albrecht war das dritte{\color{red} [sic] }.
Wo sind Sch\"{o}nheit und Jungfr\"{a}ulichkeit geblieben?
Unerbittlich hat der Sohn den Kohlestifft gef\"{u}hrt, undbestechlich wiedergegeben, was vor Augen war.
Aber er war der Sohn.

\vfill

{\color{red} [ Nun f\"{a}ngt er an nur die rechten Seiten zu beschreiben, und dreht am ende das Heft herum, und beschreibt wiederum nur die rechten Seiten] }

\clearpage
