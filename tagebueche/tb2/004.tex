\subsection{29.8.1942}
%\mktitle

Aber es war die Mutter.
Aber er war der Meister in seiner Kunst.
Aber alle Kunst ist allerletzt Herzschlag des Blutes und weiss mehr als um das, was vor Augen ist.
Und so ward dieses Bildnis der Mutter ein Suchen nach dem was bleibt, nach dem, was von Ewigkeit in uns war und selber ewig ist, und ein St\"{u}ck Gottes und nach ihm verlangt.
Wir Deutsche heissen es die Seele.
Ich weiss kein Bild, das mehr diesen Namen verdient als das der Mutter D\"{u}rers.
Zwar hat sich die Seele ganz in das Augen gerettet, h\"{a}ngt ganz in dem ferngerichteten Blick, als wollte sie albereits fortfliegen, als sei sie im K\"{o}rper gar nicht eigentlich mehr zu Haus.
Darum musst Du dieser Frau und Mutter ins Auge schauen, lieber Betrachter.
Du musst Ihr lange und and\"{a}chtig ins Auge schauen, um das zu finden was an Seele darin ist: Siehe ich m\"{u}hte mich, ich sorgte mich, empfing und gebar, weiss um Lachen und Weinen, um S\"{u}nde unr Reuhe - siehe es ist alles dahin!
Innsbruck ich muss Dich lassen, Ich fahr dahin mein Strassen, in fremde Land dahin, mein Freud ist{\color{red} [sic] }


\clearpage
