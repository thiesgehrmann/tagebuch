\def\day{31.8.1942}
\mktitle

Der Freigeist von Sans-couci{\color{red} [sic]} hatte das siebenj\"{a}hrige Schicksal des grossen Krieges erfahren, sein R\"{u}cken war krumm von der Last so vieler Schlachten, die Z\"{a}hne waren ihm ausgefallen und eisgrau starrte das Haar an den trockenen Schl\"{a}fen, aber der zahnlose Mund hatte den beissenden Spott verloren, und keiner war S{\color{red}[sic]} sicher, dass sich der Witz des K\"{o}nigs nicht An ihm versuchte.
So hatte er einmal den General Zieten auf Karfreitag zur Tafel geladen, aber der alte Husar hatte ihn schuldiger Ehrfurcht um Urlaub gebeten: er k\"{o}nne und werde nicht kommen, weil er zum Abeldmahl ginge.
Als sie zum n\"{a}chsten Mal wieder in Sanscouci sassen, alle die lockeren Geister, die, von den Kerzen, Des K\"{o}nigs beleuchtet, das leckere Mahl \"{u}berm\"{u}tig genossen als Zieten ihm gegen\"{u}bersass, statt zur Seite die Ungnade zu sp\"{u}ren reizte den K\"{o}nig sein Teufel der Bosheit gegen den Alten: "wie ist ihm, Zieten", rief er \"{u}ber die Tafel und lies die Augen der sp\"{o}ttischen Frage schon das Schellenspiel l\"{a}uten, "wie ist ihm das Mahl am Karfreitag bekommen? Hat er den {\color{red} [sic] }

\clearpage
