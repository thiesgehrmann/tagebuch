\def\day{1.9.1942}
\mktitle

Friedrich von Bodelschwingh erschien nicht un\"{a}hnlich einem westf\"{a}lischen Bauern in seiner K\"{o}rperlichen Bildung, aber mit auffallend breiter Stirn und hochgew\"{o}lbtem Sch\"{a}del, mit tief liegenden Augen, die klug betrachtend, nicht gutm\"{u}tig aber unendlich g\"{u}tig blickten, mit dem Mund eines Redners, dem man es aber doch anmerkt, dass er auch schweigen k\"{o}nne.
Auf mancherlei Weise aber erwekte er die Vorstellung eines Herrschers und Herren, der in Liebe gebietet und gerechtes Gericht halten kanwie seine V\"{a}ter unter der Linde zu Dortmund.
Mir war es besonders bemerkenswert, wie er in sich Eigenschaften vereinigte, die gerade in ihrer Gegens\"{a}tzlichkeit einen ganzen Menschen darstellten.
Aufrechte W\"{u}rde und echte Demut, eiserne Tatkraft und zartestes Erbarmen, durchdringende Menschenkenntniss und doch auch jene Einfalt, Unk\"{u}mmertheit und heiter spielende Laune, die das Kind und den K\"{u}nster {\color{red} [sic] } auszeichnen.
Heinrich Wolfgang Seidel.
Es war einige Jahre vor dem Weltkrieg.
Ein tr\"{u}ber Wintertag liess den {\color{red} [sic] }

\clearpage
