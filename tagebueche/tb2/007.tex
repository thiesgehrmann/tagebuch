\subsection{3.8.1942}
%\mktitle

Beethoven.
Wesensmerkmale Deutschen Seelentums sind: innige Naturliebe, tapferer Trotz im Ringen mit den widerw\"{a}rtigen Gewalten des Schicksals und ehrf\"{u}rchtige Hingabe an das Geheimniss der g\"{o}ttlichen Wirklichkeit.
An Beethoven der zu den gr\"{o}ssten Erscheinungen unseres Volkes geh\"{o}rt, l\"{a}sst sich dieser Dreiklang des Herzens in besonders lebendiger weiser veranschaulichen.
Beethoven hat Zeitlebens eine heisse Liebe zur Natur besessen.
Wir haben eine F\"{u}lle von Ãusserungen des Meisters, die uns diese Hingabe ahnen lassen.
So schreibt er einmal, nachdem er die heisse Sommerwohnung in Wien mit l\"{a}ndlicher Umgebung vertauscht hat:
"Wie froh bin ich, in Geb\"{u}schen und W\"{a}ldern, unter baumen, Kr\"{a}utern und felsen wandeln zu k\"{o}nnen.
Kein Mensch kann das Land so lieben wie ich.
Ich bin selig, gl\"{u}cklig in Walde!
Ist es doch, als ob jeder Baum zu mir spr\"{a}ch: Heilig, Heilig ... O welche Herlichkeit!
Wenn Wenn {\color{red} [sic] } ich am Abend den Himmel staunend betrachte und das Heer der Lichtk\"{o}rper, dann schwingt sich mein Geist \"{u}ber die so vielen Millionen Gestirne hin zur Urquelle, aus welcher alles Erschaffen erstr\"{o}mt."

Heil.

\clearpage
