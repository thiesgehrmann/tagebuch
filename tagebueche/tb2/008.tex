\def\day{4.9.1942}
\mktitle

So hell wie in Basel trat des Paracelsus Lebensschicksal nie wieder ins Licht der Geschichte.
K\"{o}nnte man ihn hier nicht auch mit Faust vergleichen?
Wirklich deutlich, leuchtend und im Ged\"{a}chtnis haftend sind die ersten Szenen: Faust als Forscher, Faust zu Ostern, Faust der Liebende.
Alles Sp\"{a}tere bleibt mehr schattenhaft.
Paracelsus scheint das Erlebnis der Liebe versagt gewesen zu sein.
Sein einziges authentisches word, das R\"{u}ckschl\"{u}sse erlaubt, ist ein Nebensatz aus einer kurtzen Betrachtung \"{u}ber seine Schicksale.
Er lautet: "... dieweil ich der Venus kein Zubitter bin."
Ob sich das Wort auf die Liebe \"{u}berhaubt oder nur auf die vulg\"{a}re Ausschweifung bezieht, bleibt dunkel.
So liegt auch also auf diesem f\"{u}r das innerste Schicksal des Menschen - man denke im Gegensatz nur an Goethe - so wichtige Gebiet v\"{o}lliges Schweigen und ein tiefer geheimnisvoller Schleier.
Dieses faustische Element fehlt also ganz im Bilde des Paracelsus oder wir wissen wenigstens nichts davon.
Um so faustischer wirkt dieses "Momentbild" Basel.
Es zeigt ihn als allumfassenden Geist, als ewigen K\"{a}mpfer, als rastlosen Sucher.

\clearpage
