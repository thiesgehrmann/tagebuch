\subsection{5.9.42}
%\mktitle

O Musika, du{\color{red} [sic] } edle Kunst!
Wer weiss wie ich zum  Trauung  Weltanschauung  die Bebauung Trauung  alliierten {\color{red} [sic] }, wer weiss wie ich zum Lehrer f\"{u}r das Laienmusizieren geworden bin, der kann mich am leichtesten verstehen.
Als jugendlicher Wandervogel spielte ich die Laute und wir sangen die Lieder des Zupfgeigenhansel dazu.
Das ist eine Liedersammlung, in der alte Volkslieder von Hans Br\"{a}uer und seinem Kreis zusammengestellt wurden, aus Liebe zur Deutschen Heimat und Landschafft.
Die Liebe zu diesen Liedern und zu meiner Laute bewog mich, ins Instrumentenmacher zu werden um all die Instrumente noch neben die Laute zu stellen, die uns helfen konnten, urv\"{a}terliches Erbgut wieder lebendig zu machen.
In Markneukirchen, der Stadt, in der in jedem Hause Instrumente gemacht werden, erlernte ich das Handwerk und gr\"{u}ndete auch meine Werkstadt.
Hier baue ich Mit meinen Gehilfen und einem meiner S\"{o}hne Instrumente, von denen vor meiner Arbeit nur ein paar Musikwissenschaftler wussten und die doch aus der heutigen Deutschen Hausmusik nicht mehr fortzudenken sind: Blockfl\"{o}te


\clearpage
