\def\day{3. Oktober 1942 *}
\mktitle

Ich habe es mir vorgenommen, meine Schreib\"{u}bungen regelm\"{a}ssig fortzuf\"{u}ren.
Es sollte eigentlich die erste Morgenstunden{\color{red} [sic] } nach dem Fr\"{u}hst\"{u}ck sein.
Das ist aber nicht so einfach, denn schon allein zweimal in der Woche ist ein f\"{u}rchterliches "Reinschiff" wobei das Wasser k\"{u}belweiss durch die elende Baracke gegossen wird.
Ist das Wette nur einigermassen ertr\"{a}glich, so flieht von allein schon ein jeder ins Freie.
Damit ist der halbe Vormittag vorbei.
Augenblicklich schreibe ich nach dem Mittagessen, das heute wie jeden Sonnabend aus einem Blechnapf voll Eintopf besteht.
W\"{a}hrend des Essens kam R. Lindemann darauf zuberichten, was f\"{u}r eine gute Entdeckung er mit einem hier k\"{u}rtzlich eingetroffenen Band des Volksbrockhaus gemacht hat.
Es war auf jeden Fall erfreulich zu sehen, wie er mit richtigem klaren Blick eine der vornehmsten Zukunftsaufgaben eines Unteroffiziers als Ausbilder erkannte und mir in einem kurtzen Gespr\"{a}ch umriss.
Und ich fragte ihm nur kurtz, wie es damit wohl bei den ewigen Kartenspielern und Radaubr\"{u}dern stehen w\"{u}rde!

\clearpage
