\subsection{8. Oktober 1942 *}
%\mktitle

In einer normalen Denkweise gedacht, kann man es nicht verstehen, wieso es m\"{o}glich ist, dass unsere Post einmal schnell, einmal langsam l\"{a}uft.
So kommt es, dass mir wieder mehrere Briefe aus der Zwischenzeit fehlen.
Und daraus ergibt es sich, dass mir so manche Frage offenbleibt.
Ich weiss, es sind viele Briefe geschrieben worden.
Verlange ich den nun zuviel, wenn ich von ihr, die sie zu Hause ist in dem h\"{a}uslichen Frieden des Elternhauses umsorgt leben darf, auf jeden einzelnen Punkt meiner Zeilen eingehen soll?
Bem\"{u}he ich mich doch schon, in meiner auf 76 Zeilen beschr\"{a}nkten Monatspost dies in bezug {\color{red} [sic] } auf die wesentlichsten Punkte ihrer Briefe zu tun.
Aber, es ist so, wie sie mir einmal schrieb, all dies geh\"{o}rt eben mit zu den Enbehrungen, die uns die Zeit auferlegt.
Und an uns ist es dabei, vertrauend einander in festem Glauben die Treue zu halten.
Na, an mir soll und kann es nicht liegen, werde ich doch vom kanadischen Staat so sorgf\"{a}ltig besch\"{u}tzt, dass mir doch beim besten Willen nichts derartiges begegnen kann.

\clearpage
