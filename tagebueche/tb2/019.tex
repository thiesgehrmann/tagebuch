\def\day{13. Oktober 1942 *}
\mktitle

Herrlich l\"{a}uchtende Herbsttage sind hereingebrochen.
Sie passten so gar nicht zu den Spannungen, die die letzten Tage mit dem Versuch, in zahlen 83 gefangene{\color{red} [sic] } in einem umz\"{a}unten Lager nochmals zu fangen, hervorgerufen haben.
Einen kurzen Bericht dar\"{u}ber an anderer Stelle.
Ich warte auf Post.
Es ist gewiss nicht leicht, wenn man, wie es auf mich zutrifft, es eben gewohnt gewesen ist, in seinen Briefen, so wohl denen, die ich schreibe, wie auch denen, die ich empfange, ein tats\"{a}chlich gelebtes Leben zu f\"{u}ren.
Ich weiss es, so ist es f\"{u}r mich gewesen.
Darum trifft mich auch die jetzige Freiheitsberaubung - soeben habe ich meinen Sitz ge\"{a}ndert, so dass ich nun mit aufrechtem Oberk\"{o}rper am Tisch sitzen kann - vielleicht h\"{a}rter in bezug auf die Post, als einen dem es v\"{o}llig gleichg\"{u}ltig ist, ob er nun wenig oder \"{u}berhaupt keine Briefe von ihm nahestehenden Menschen erh\"{a}lt.
- Zwei B\"{u}cher holte ich mir heute: Alois Senefelder \"{u}ber die Kunst des Steindruckes und ein umfassendes Werk \"{u}ber Prag.
Das erste{\color{red} [sic] } gibt eine umfassende Auserbeitung, dass zweite{\color{red} [sic] } soll mein Geschenk an Margret und ihren Gatten sein. -

\clearpage
