\subsection{27. Oktober 42   abends.}
%\mktitle

Nun nehme ich zum zweitenmal heute die Feder zur Hand.
Nach dem Abendbrot habe ich gelesen.
Kurz vorher gab es eine kleinde unbedeutende aber nicht uninteressante Diskussion mit E. Schmidt {\color{red} [K.G. 26 von Erich Schmidt, Uffz., Magdeburg, Grosse Diesdorferstrasse 36b] }: Er k\"{o}nne keine solchen dicken B\"{u}cher lesen.
Ein paar Worte hin und her, doch wir bekommen keinen Streit.
- Ich habe dann weitergelesen.
Mich w\"{u}hlt dieses Buch auf, es ergreift mich zutiefst,
Zu wahr ist dieses Gestalt der Christa von Vermien, als dass ich solch Buch noch als Unterhaltung hinnehmen k\"{o}nnte.
Dann f\"{u}hle ich etwas, das ich nicht ausdr\"{u}cken mag, ich k\"{o}nnte es schon, doch w\"{u}rden die Worte nur blass entkleidend wirken.
Es ist das gleiche{\color{red} [sic] }, wie oft schon.
Es ist mir deutlich in Erinnerung an damals, als ich den Abend mit der Edeltraut durch die Felder gegangen war, Margret zum letzten Mal gesehen hatte, da, als mich eine ungeheure Traurigkeit des Alleinseins, ja des Alleinseinm\"{u}ssens \"{u}berfiel.
Und in diesem Augenblick frage ich mich: Warum hat mich nie dieses Gef\"{u}hl befangen, wenn ich von jenem geliebten M\"{a}del schied, das so wunderbar un  mir ihre Liebe und sich selbst dargebracht hat?
Weil es nie ein Abschied wurde.
Ja, ich war es ja selbst, der irh mit all meinen Leebenkr\"{a}ften{\color{red} [sic] } die Schwere des ersten Trennens nehmen wollte um jeden Preis und auch genommen habe auf unserem zweiten gang durch das nun herbstliche Gl\"{u}cksburg.
Es war mir gelungen, sie glaubend zu halten wie ich ihre m\"{a}dchenhafte Hand mit meiner oftmals ganz umschloss.
Und so ist sie ihren Weg gegangen, hat aus sich die Gl\"{a}ubichkeit ihrer ebenerwachten Jugend und Sch\"{o}nheit in stolzer Kraft zu unwandelbarem Willen ihres Herzens werden lassen, um nun zu erf\"{u}llen, was ich in der Zeit unseres Gl\"{u}ckes schon einmal vor sie stellte: Ueber{\color{red} [sic] } die Not des Krieges, \"{u}ber Meere hinweg, w\"{a}hrend langer Jahre ungewissen Alleinseins in ihrem hohen Glauben stolz und stark sich selbst getreu bleiben, den Weg ihrer Liebe zu gehen in meinder Liebe zu ihr, zu ihrem Herzen allein, das mir Heimat wurde, zu dem einzig auf einer weiten Welt allein ich werde heimkehren und koennen{\color{red} [sic] }, um ihr dann zu sagen: Du mene Hilde, Mutter meiner Kinder, \ul{Dich} liebe ich!

\clearpage
