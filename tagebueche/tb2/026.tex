\subsection{11. November 1942 *}
%\mktitle

Inzwischen sind sechs Tage vergangen.
Am sechsten war mein Geburtstag.
Meine Freunde und engeren Kameraden brachten mir ein sch\"{o}ne \"{U}berraschung noch am fr\"{u}hen Morgen vor der "roll call".
Ich habe mich riesig gefreut.
Im Verlauf des Tages kam mancher gute Bekannte, der sich des Tages erinnerte und meiner gern gedachte.
F\"{u}r den darauffolgenden Sonntag habe ich mir den \"{u}blichen Geburtstagskuchen mit Bohnenkaffee bestellt, was mir bei einem urgem\"{u}tlichen Plauderst\"{u}ndchen ohne Anstrengungen mit zehn Mann gemeinsam verzehrten.
Der 9. November kam.
Nach einer kurzen w\"{u}rdigen Gedenkstunde fand im Speisesaal ein Vortrag statt umrahmt von Marschmusik, wie das ja auch daheim in Deutschland \"{u}blich ist bei derartigen Vortr\"{a}gen.
Es war ene erhebende Stunde und ein Gl\"{u}ck in den jetzigen Tagen mit ihrem Wirrwarr von Nachrichten und sich \"{u}berst\"{u}rzenden Meldungen, solch einen Vortrag in der Gefangenschaft lauschen zu k\"{o}nnen, - Kunstschriftfedern habe ich heute bekommen.
Wenn ich meine Schriftz\"{u}ge anschaue so sind sie nicht sch\"{o}n und nicht schlecht, aber wie gesagt: \"{U}bung macht den Meister!


\clearpage
