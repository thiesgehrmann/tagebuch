\subsection{12. November 1942 *}
%\mktitle

Ganz schnell und pl\"{o}tzlich ist es gekommen.
Ein paar Tage lang hat es stets ein wenig geschneit, nur ein bisschen.
Die Erde sah trotzdem braun und schwarz aus.
Schnell hat sich der leichte weisse Schimmer unter den 16 000 Paar Stiefelsohlen zertreten.
Der Boden war ja auch noch viel zu warm.
Aber dann hat es einen ganzen Tag lang bei gelinder K\"{a}lte in dichten Flocken und tanzenden Wirbeln geschneit.
Immer dichter kamen die kleinen Kunststernchen herunter, immer zusammenh\"{a}ngender wurde das weite weisse Winterkleid der Erde, immer lautloser verhallten die Schritte und irgendwelches Stapfen auf dem dichter und dichter werdenden Schneeteppich.
Nun liegt das ganze Land ausgebreitet und zugedeckt da.
Wenn ich es auch nicht sehen darf, so denke ich doch, wie es w\"{a}re, das scheinende und webende Leuchten verschneiter Fluren und Wege auf einem Gang durch den Winterwald sehen zu d\"{u}rfen, wie wenn es gar unser M\"{a}rchenwald w\"{a}re und wie wenn wir zwei nur einmal wieder eine Weihnachts- oder Neujahrsnacht miteinander friedvoll still dahingehen d\"{u}rften: Es w\"{a}re mir sch\"{o}ner als alles jetzt.

\clearpage
