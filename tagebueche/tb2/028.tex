\def\day{13. November 1942 *}
\mktitle

Seit einigen Tagen tobt ein Schneesturm \"{u}ber die Felder, in denen mitten unser Lager mit seinem \"{a}rmlichen H\"{u}tten steht.
Noch ist der Anblick neu und ungewohnt.
Bis an die erh\"{o}hten Eing\"{a}nge lecken die Scheewehen gleich ziehenden Wogen herauf.
Sie liegen still, wenn man f\"{u}r den Augenblick hinschaut, aber ganz fein, leise und t\"{o}dlich f\"{u}r den ohne ein Dach \"{u}ber dem Kopf mit einem warmen Herd darin ziehen sie sich entlang.
Bis \"{u}ber Meterl\"{a}nge reichen wie gewaltige lange gl\"{a}serne Schwerter die Eiszapfen von den nicht \"{u}berm\"{a}ssig hohen H\"{u}ttend\"{a}chern in bizarre Gebilde ver\"{a}stelt herab.
Durch die Gassen, die von den L\"{a}ngswenden{\color{red} [sic] } der H\"{u}tten gebildet werden, fegt heulend und wirbelnd der eisige Nord, der \"{u}ber weite ebene Fl\"{a}chen direkt von der Hudson-Bai ungebrochen herunterst\"{u}rmt und in einem fantastischen wlden Tanz die weissen Flocken vom Himmel reisst oder sie peitschend in wagerechter Linie sie jedem, der sich auf die Strasse wagt, scharf in Gesicht und Augen schneidet.
Wer drinnen am Tisch vor dem Fenster sitzt, kann sich an der leuchtenden Helle der Schneefl\"{a}chen und den glitzernden Kristallen der Eisblumen erfreuen

\clearpage
