\subsection{20. November 1942 *}
%\mktitle

Das Jahr neigt sich zu seinem Ende.
Traurig und \"{o}d, trostlos und sterbend steht die Natur um das gleichnishaftebild des mir so masslos verhassten Stacheldrahtverhaues.
L\"{a}ngst schon ist das helle weisse Leuchten der vom Himmel gefallenen Totendecke unter Zur\"{u}cklassung der Schmutzdecke vergangen.
Und, so erscheint es mir, gibt sich auch unser Leben dar, allt\"{a}glich, nur zu allt\"{a}glich.
Das glatte Gesicht, dass Zucht und Disziplin, Scheu vor einem Fremden und Furcht vor der Verantwortung vielen auferlegt hatte, dass ist \st{nun} hier in der nun schon jahrelangen "Gew\"{o}hnung" abgefallen, wo es nur Maske war.
Charakter ist nicht gefragt, Anstand und R\"{u}cksicht stehen niedrig im Kurse.
Pflicht und Verpflichtung vor der eigenen Person, Ja wenn schon nicht Person, weil keine Pers\"{o}nlichkeit vorhanden, den wenigsten{\color{red} [sic] } vor dem Dienstgrad sind l\"{a}ngst ungekannt.
Man lebt ja so sch\"{o}n unter sich.
Und so l\"{a}uft ein Leben dahin, das ja kar{\color{red} [sic] } keins mehr ist.
- Eine Weihnachtskarte durfe gestern geschrieben werden.
Der Entwurf war sehr gut, die Herstellung des Druckes auf gut Deutsch: saum\"{a}ssig.
Ich schrieb sie heim, weil ich nun weiss, dass sie sich freue.

\clearpage
