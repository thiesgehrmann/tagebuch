\subsection{3. Dezember 1942}
%\mktitle

Das Jahr 1942 geht auch seinem Ende zu.
Wo ist es geblieben.
Das einzige, was lohnenswert w\"{a}hre festzustellen, k\"{o}nnte sein: Es ist dahin, unwiederbringlich dahin.
Aber was soll das alles, davon wollte ich ja gar nicht schreiben.
Zum Ende November kriegte ich noch ein paar sch\"{o}ne Briefe.
Wirklich sehr, sehr sch\"{o}ne Briefe waren es.
Viel Neues brachten sie.
So die Mitteilung von ihrer neuen T\"{a}tigkeit, ihr feinsinniger Brief, der mir von der Auff\"{u}hrung der "Uta von Naumburg" erz\"{a}hlte.
Wie wunderbar klingt es f\"{u}r mich, wenn sie sich in einer solchen Zeit wie jetzt w\"{u}nscht, mit mir nach Naumburg zu fahren.
Dreimal habe ich fr\"{u}her schon vor diesem unverg\"{a}nglich sch\"{o}nem{\color{red} [sic] } Bildwerk gestanden.
War es nicht einmal am sch\"{o}nsten, als die Strahlen der sinkenden Sonne lang in den Westchor hineinfielen?
War ich nicht damals mit Elisabeth dort gewesen?
Wo mag sie jetzt sein?
Ob es ihr Gl\"{u}ck warm, jenen grossen Holl\"{a}nder zu heiraten?
- Ich bin gl\"{u}cklich \"{u}ber meine Post.
Sie hat mich froh gemacht.
Sch\"{o}n ist es, wenn man von Eltern und Braut Briefe bekommt.
Es ist schon viel hier!
Wenn nur nicht so viele Briefe zwischen fehlen w\"{u}rden.
Doch bin ich.

\clearpage
