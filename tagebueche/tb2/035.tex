\subsection{13. Dezember 1942}
%\mktitle

Zehn Tage sind schon wieder vergangen.
Die Zeit rast, wenn ich die Stunde betrachte, und sie schleicht wie eine Schnecke dahin, wenn ich auf Wochen uder gar Monate schaue.
Was habe ich in dieser Zeit geschafft?
An erster Stelle muss ich das Buch Prag nennen.
"Prag, Kunst, Kultur und Geschichte."
Ich habe es mir von ihr zu Weihnachten gew\"{u}nscht.
Ob sie es noch bekommen hat?
Ich habe es als ausgezeichnet gefunden.
Leider kann ich es nicht auslesen weil unsere B\"{u}chereileitung die B\"{u}cher ab 15. Dezember einziehen muss.
Die B\"{u}cher sollen "\"{u}berholt" werden.
Na, Kommentar \"{u}berfl\"{u}ssig.
Man hat ja auch entschieden, dass man einem Maupassant, der \ul{den} Deutschen Offizier aus dem franz\"{o}sisch-deutschen Krieg 1870/71 als "Schweinehund" darstellt, solch einen Fehler schon verzeihen m\"{u}sse, da er eben ein sooo{\color{red} [sic] } gottbegnadeter Schriftsteller sei!
Auch schon ganz nett.
Und so ein schmutziger Schmierfink, der "Karl und Anna" einst verbrechen durfte, bleibt eben unbehindert dem leseeifrigen und -freudigen Publikum erhalten.
Doch nicht mehr davon.
- Die letzten Brief von meiner Braut brachte mir wieder sehr viel Freude.
Ich weiss auch, noch ganz anders k\"{o}nnte ich ihre Briefe aufnehmen wenn ich nicht unter solchen Menschen zu leben gezwungen w\"{a}re.
Das ist es!

\clearpage
