\def\day{15. Dezember 1942}
\mktitle

Gestern haben wir zu Meinhard Mildes Geburstag.{\color{red} [sic] } den \"{u}blichen Geburtstagskaffee in einer "kleinen" Gesellschaft von "nur" 13 Mann getrunken.
- Am Rande kleine Aphorismen: man sollte nicht meinen, mit welchem geringen Gut an allgemeinem Wissen und Bildung und noch weit geringerem Fachwissen einer heutzutage Oberfeldwebel werden kann!
- Mich schert das alles noch wenig mehr.
Ich habe meine Zielsetzung ausserhalb solcher Konkurrenz gestellt und - siehe da, ich lebe mit einem Male wieder.
Ich f\"{u}hre ein h\"{o}chst lebendiges Leben, wenn ich hier inmitten einer vielz\"{a}hligen Menschenmenge sitze und ein Buch wie "Prag" lese.
Ich f\"{u}hle mich tief vom Leben ergriffen, wenn mir zu diesem Buch ein Brief von Frau Magret kommt.
Und dankbar f\"{u}hle ich mich innig einem lebensvol pulsenden{\color{red} [sic] } Streben hingegeben und verpflichtet wenn mit meine Braut einen sehr feinempfundenen Brief \"{u}ber die Gestalt der "Uta von Naumburg" zu schreiben weiss.
Ob ich ihr daf\"{u}r einmal etwas \"{u}ber meinen hiesigen Tagesablauf mitteilen soll?
Nein, man kan gewiss nicht sagen, dass ich unt\"{a}tig bin.
Und ich kann dank meiner lieben Eltern auf breiter solider{\color{red} [sic] } Grundlage bauen.

\clearpage
