\subsection{22. 12. 1942. *}
%\mktitle

Vor einem Jahre waren wir in Liverpool um unsere Verschleppungsfahrt nach Canada anzutreten.
Vor dem schrieb ich nach Hause dass ich mich freuen werden zu weihnachten noch eine Zigarette zu haben.
Ich hatte Zigaretten in H\"{u}lle und F\"{u}lle, aber dass Weihnachtsfest sah uns eingepfercht im Atlantik schwimmen.
Nur geraucht habe ich in den 10 Tagen der Seefahrt.
Laufend Pfeife, Zigarette oder eine "Gedrehte".
Eins war aber fest und sicher gel\"{o}st: Es gab kein Weihnachten f\"{u}r uns.
In diesem Jahre ist nun das bevorstehende Weihnachten wieder bedeutend problematischer: Wir haben zu viel um ganz arm zu sein, und zu wenig um unsere j\"{a}mmerliche Armut zu vergessen.
Du{\color{red} [sic] } ich am Heiligen Abend Heizw\"{a}chter bin, wird der Weihnachtsabend f\"{u}r mich ausgef\"{u}llt und ich hoffe, dass er mir vielleicht eine stille Stunde in meiner Ecke bescheren wird, wenn alle anderen im Speisesaal bei der geplanten Weihnachtsfeier sind.
Drei kleine, ganz kleine Weihnachtskerzen sind die einzigen, die mir diesmal verblieben sind.
In dem "l\"{u}tten" Holztannenbaum, den sie mir schickte, werden sie mir ganz leis{\color{red} [sic] } und heimlich dann erz\"{a}hlen von Deutscher Weihnacht fern, fern in der Heimat.

\clearpage
