\subsection{Am Heiligen Abend 1942 {\color{red} [24.12.1942] }}
%\mktitle

Im Kriegsgefangenenlager 23, Monteith/Ont. Can.
Es war fr\"{u}her eine liebe Gewohnheit von mir, den Heiligen Abend, nachdem meine Eltern sich zur Ruhe gelegt hatten, bis in die fr\"{u}hen Morgenstunden auszudehnen.
Es war ja die Weihenacht.
Und sie war wohl anders als alle anderen N\"{a}chte je sein konnten.
Ich entsinne mich, dass schon als Junge fr\"{u}hzeitig grade in dieser Nacht meine Gedanken weit wanderten, sich z\"{a}rtlich liebend, aber doch fest und f\"{u}hrend mit jener einzigen Geliebten verbanden.
Wir lange Jahre ist sie mir unbekannt geblieben!
Kenne ich sie nun?
Kaum das ich diese herrliche Gewissheit erf\"{u}hlen und mir glaubend machen konnte, riss mich das Schicksal hart von ihrer Seite.
Es ist uns nur noch der Glaube geblieben.
Solch Glaube steht in der Zeit unserer Generation unter harten Pr\"{u}fungen.
Mir sind die H\"{a}nde im Schreiben gebunden, geschweige das ich mit einer Gabe das geliebte Herz erfreuen k\"{o}nnte.
Und von ihr, die sie die Freiheit hat, bekomme ich nichts dank der unerh\"{o}rt "pr\"{a}zisen Arbeit" englischer und kandadischer Zensurstellen.
Die Weihnachtspakete, ja die habe ich in diesem Jahr und sie sind auch mit all ihrer Liebe vor mir.
(Wer wollte es mir aber verargen, dass die k\"{o}stlichen Pl\"{a}tzchen lang schon ihren Bestimmungsweg fanden?)
Ich denke auch an mein erstes Weihnachtspaket von ihr, das damals genau zum Mittagstisch ankam.
Der Posthilfsbote erhielt einen guten Schnaps zum Aufw\"{a}rmen und meine gute alte Grossmutter schaute still erfreut und staunend ihrem Enkel in seiner Freude zu.
Doch, dies war einmal.
War es wirklich?
nein{\color{red} [sic] }, es ist heute noch, weil es immer in mir leben wird.
- Heute sitze ich wohl wie jeden anderen Tag an meinen{\color{red} [sic] } selbstgebauten Tisch - man hat uns keinen gegeben -in meinem gekauften Liegestuhl - auch die gab man uns nicht - und begehe still f\"{u}r mich meinen Heiligen Abend.
Einmal bin ich allein mit meinen Gedanken, denn das ganze Lager ist zur "sogenannten" Lagerweichnachtsfeier im Speisesaal.
Und anschliessend habe ich Feuerwache bis in die Nacht um 4 Uhr.
Ein teil davon ist noch freiwillig \"{u}bernommen.
So geh\"{o}rt auch hier in diesem ungl\"{u}ckligen Jahr die Weihenacht mir, meine Gedanken weilen auch heute bei der einzigen Geliebten und ich will es glauben: Ich kenne sie.

\clearpage
