\subsection{11. Januar 1943}
%\mktitle

"Ich bin sehr froh dar\"{u}ber, dass ich den Mann, mit ich dem ich meine sch\"{o}nsten Jugendjahre verbracht habe, so liebe dass ich ihn heiraten kan."
Dies hat sie mir geschrieben, und vielleicht, ja warscheinlich weiss sie gar nicht, was sie mir damit gegeben hat.
Das ist es, wohin ich sie ja habe bringen wollen, seit ich hier in Gefangenschaft gekommen bin.
Die offen und klar ausgesprochenen Worte von sind ein neues, ein Anfang, der kommen musste, wenn das Bestand haben sollte, was in einem unvergesslich sch\"{o}nen Sommer begann und nach drei kurzen Jahren so j\"{a}h unterbrochen und brutal auseinandergerissen wurde.
Ihre Worte sind nichts anderes, als das uneingeschrenkte Bekentniss: Ja, Hans, ich weiss es, die Zeit der ersten jungen Liebe ist vergangen, jetzt ist jeder wieder f\"{u}r sich in seine n\"{u}chterne Umwelt gestellt - wobei sie eine hat und ich nicht - allerdings mit anderen Augen: wissender und dadurch dem Leben verlangender gegen\"{u}ber.
Aber sieh dar\"{u}ber hinaus lebt soviel von Dir in mir, bist Du mir vor all denen die zu gewinnen mir ein Leichtes w\"{a}hre, so viel Wert, dass ich Dir ganz ergeben bin und daraus die Kraft finde, auf Dich zu warten.

\clearpage
