\subsection{15. Januar 1943}
%\mktitle

Ich hatte schon das Gef\"{u}hl gehabt, dass mein Sonderbrief vor dem letzten von ihr angekommen sein musste.
Wirklich brachte mir ihr vorgestriger Brief eine Best\"{a}tigung meiner Annahme.
Doch nicht nur das alein.
Auch ihre feste Entscheidung: Heiraten - ja, Ferntrauung - nein.
Ich bin sehr gl\"{u}cklich \"{u}ber diese offene Erkl\"{a}rung und ihre Gr\"{u}nde daf\"{u}r, die sie mit einer so selbstsicheren Gewissheit mir geben kann.
K\"{o}nnte ich ihr nur einen "Brief" daraufhin schreiben, d.h. w\"{a}hre ich nicht nur auf 24 oder gar nur eine 7 Zeilen-Karte angewiesen.
Ob ich etwas zu dem Druck der Verlobungskarte zu w\"{u}nschen h\"{a}tte?
Soll ich hinschreiben, dass man hinter meinen Namen "Funkmaat d.R. z.Zt. in Canada" setzen soll?
Ich m\"{o}chte es schon, denn ich kann mich nun einmal nicht mit den Axiomen und Lebensmaximen derer zufriedengeben, nach denen ich meine Erfahrungen bitter sammeln musste.
Sei es nun gegen\"{u}ber der Frau, der Gesinnung, der Verantwortung, der Pflicht, der Religiosit\"{a}t oder gar in dem Lebensformen oder der Standesehre \"{u}berhaupt.

\clearpage
