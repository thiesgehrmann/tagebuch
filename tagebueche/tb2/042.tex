\subsection{16. Januar 1943}
%\mktitle

Ich k\"{o}nnte eigentlich innerlich sehr zufrieden sein mit dem, was mir die letzten Briefe von ihr und meinen Eltern gebracht haben.
Ich bin es ja auch, nur darf man es hier in dieser h\"{a}mischen und misg\"{u}nstigen Umgebung, die so sicher sein will, dabei aber nur sich selbsts\"{u}chtig sieht, und mehr Angst davor hat, durchschaut zu werden und jene, die es k\"{o}nnen, erbarmungslos und erb\"{a}rmlich verfolgen will, nicht zeigen.
So begn\"{u}gte ich mich auch damit, den staunenden Gaffern im Gesch\"{a}fstzimmer zu erkl\"{a}ren, ich gebe{\color{red} [sic] } meinen Antrag desshalb unausgef\"{u}llt zur\"{u}ck weil seine Einreichung von der Heimat aus abgelehnt worden sei.
Was brauchen jene davon zu wissen, die mir raten wollen, ich h\"{a}tte doch nur vorteile von einer Ferntrauung, allein schon das erh\"{o}hte gehalt.
Ich werde es mir besser erh\"{o}hen als Funkmaat d.R.
- Die erste volle Woche meiner planm\"{a}ssigen Arbeit mit dreimal Sport am abend ist vergangen und ich kan sagen, dass mich die restlose Aufteilung meiner Zeit befriedigt und das tote Warten vertreibt, dass sich so l\"{a}hmend auf alle T\"{a}tigkeit hier legt.

\clearpage
