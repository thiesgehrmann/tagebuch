\def\day{26. Januar 1943.}
\mktitle

Gestern bekam ich einen Brief von den Eltern.
\"{U}ber die Freude habe ich ganz vergessen auf den Umschlag zu schauen.
Als ich es soeben wie ich grade meine Gedanken f\"{u}r einen ausf\"{u}hrlichen Brief an Roland zusammenfassen wollte, tat, stand jenes Wort tats\"{a}chlich darauf.
Das ist nun das ganze Resultat.
Habe ich denn nicht gestrebt und micht nicht bem\"{u}ht?
Soll ich denn wirklich zu dusselig sein und in falscher Einsch\"{a}tzung meiner selbst zu hochfordernde Pl\"{a}ne gesteckt haben?
Nein, dieser Schlag ist gewiss nicht leicht; es ist die Vernichtung jedes selbstbewussten und sicheren Gef\"{u}hls, \"{u}ber seine Arbeit die innere Zufriedenheit zu empfinden, wenn - ja es gibt einen Ausweg: aber dieser wiederstrebt mir, eben weil er einmal so bequem ist und zum anderen die seichte, allgemeine und vielleicht geforderte L\"{o}sung mit sich bringt.
Bin ich denn tats\"{a}chlich so weit wie grad jene, deren Leben mir in den ersten Jahren unter den Einwirkungen und Folgen des Alkohols und der Hurerei zum grauenhaften Beispiel geworden ist?
Ich nicht!


\clearpage
