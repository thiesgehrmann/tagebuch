\def\day{7. Februar 1943}
\mktitle

Endlich gab es die Weihnachtsgeschenke des F\"{u}hrers.
Nachdem was ich geh\"{o}rt habe, sollen als Weihnachtsgabe ein oder zwei Millionen Reichsmark gesandt worden sein, mit denen es ohne Weiteres m\"{o}glich gewesen w\"{a}re, den gesamten kanadischen Geschenkartikelmarkt leer zu kaufen.
Daher ist auch die verz\"{o}gerung gekommen, weil es dem Sonderbeauftragten der Schweizer Délegation nicht m\"{o}glich war die erforderliche Anzahl Gegenst\"{a}nde zu erwerben.
In meinem Paket, die \"{u}brigens \"{a}usserlich alle glech aussahen, habe ich unter dem Buchstaben "R" eine Gold-Silberring bekommen.
Alle \"{u}brigen Gegenst\"{a}nde gleichartig: 1 Handtuch, 1 Taschentuch, 1 St\"{u}ck Seife, 5 Rasierklingen, 1 Tafel Schokolade, 1 Paar Sokken.
200 Zigaretten - bei deren Auswahl man wohl bedacht gewesen ist, dem Tabakh\"{a}ndler gef\"{a}llig zu sein - hatte es bereits vorher einmal gegeben.
Aber ich werde mich h\"{u}ten diese Marke "Lungenreisser" zu rauchen.
Daf\"{u}r habe ich schon einen guten und dankbaren Abnehmer in Meinhard Milde.
Das sch\"{o}nste Geschenk bleibt also die Weihnachtskarte vom F\"{u}hrer.

\clearpage
