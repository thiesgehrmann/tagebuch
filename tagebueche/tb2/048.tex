\def\day{12. Februar 1943.}
\mktitle

Ich hatte Heizwache heut nacht.
Das ist eine g\"{u}nstige Gelegenheit, ohne weiteren "Zeitverlust" am Tage die leidige W\"{a}sche zu erledigen.
Es ist furchtbar kalt gewsen.
Ein eisiger Sturm fegte die ganze Nacht \"{u}ber unsere armseligen Baracken hin.
Die beiden eisernen \"{O}fen standen st\"{a}ndig in Glut und trotzdem war es nur m\"{o}glich, eine m\"{a}ssige Temperatur im Raum zu halten.
Nach dieser Nacht ist aber ein pr\"{a}chtig in strahlender Wintersonne leuchtender Morgen aufgegangen.
Solange es im Raum nach dem Fr\"{u}hstuck noch nicht hell genug war, habe ich auf meiner Koje bei Licht gelesen.
"Ekkehard und Uta", ein Roman aus dem Hohenstaufenverlag \"{u}ber die historischen Gestalten der Markgrafschaften im elften Jahrhundert.
- Mich hat eine furchtbare Unruhe gepackt.
Nichts mag ich mir heute vorzunehmen.
Mein Herz ist wund und zerissen.
Ich m\"{u}sste Schreiben.
Wo bleibt die Post von Weihnachten?
Was sie mir wohl bringt?
Da sind liebe Mensche daheim, die mir ganz vertrauen, sie die zu mir aufschauen will!
Und ich, kann ich noch an mich selbst glauben?

\clearpage
