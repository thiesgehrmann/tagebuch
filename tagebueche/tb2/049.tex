\def\day{14. Februar 1943}
\mktitle

Eine \"{u}berm\"{a}ssige K\"{a}lte ist hereingebrohen.
Man kann bei einem Gang von einer Baracke zur anderen schon Gesicht und Ohren erfrieren.
Am Vormittag habe ich nur gefroren, jetzt am Nachmittag werde ich wieder einmal zu Meinhard gehen. Was hier bei uns in der Ecke los ist, kann ich beim besten Willen nicht sagen. E. Schm. {\color{red} [Erich Schmidt] } spielt die gekr\"{a}nkte Leberwurst und der andere, na das ist ein Fall f\"{u}r sich!
Ich werde jetzt nur noch in mir selbst gefestigt durch solches Gebaren meiner Mitmenschen.
Hoffentlich komme ich nur recht bald dazu, in meinem eigenen Heim in keiner Weise von denen abh\"{a}ngig zu sein.
Dann, das \st{W} weiss ich ganz fest und bestimmt, werde ich jedenfalls in der Lage sein, mein Leben so einzurichten, wie ich es ihr einst gelobte: "Ein Leben zu f\"{u}hren, wahrhaft k\"{o}niglich, nicht nach Geld und Gut gemessen".
- Seit zehn Tagen warte ich vergeblich auf Post.
Weder von ihr noch von meinen Eltern.
Wie kann ich da die Ruhe und Kraft finden, nach Haus das zu schreiben, was ich mir vorgenommen habe, im Besonderen meine W\"{u}nsche und Freude \"{u}ber den Brief meiner Schwester?

\clearpage
