\def\day{16. Februar 1943}
\mktitle

Wieder sind zwei Tage vergangen, die mir keine Post gebracht haben.
Herrgott, ist das ein Zustand.
Da haben sie nun zu Haus ein Fest begangen, das nicht nur sch\"{o}n ist und Freude bringt, das grade am sch\"{o}nsten und innigsten aller Deutschen F.{\color{red} [Feste] }, zur Weihnacht, seine besondere Weihe erh\"{a}lt und vor allen Dingen f\"{u}r uns ein frohes und sinnvolles Ereignis bedeutet.
Na, ich f\"{u}r meinen Teil sitze hier.
Ersten habe ich gar nichts davon gewusst und geahnt, zum zweiten erfuhr ich den Plan von ihr und meinen Eltern aus Briefen von November,die mich im Januar Mitte und Ende erreichten.
Und jetzt warte ich.
Ich hatte mir ja mit meiner Arbeit grade vorgenommen, nicht auf Post zu warten.
Aber kann ich denn nun anders?
Tag und Nacht steht dieser Gedanke vor mir, wie mag es gewesen sein.
Wem wird sie ein Aussage gesandt haben?
Darf ich darauf rechnen, dass sie auch meine Freunde bedacht, nicht nur die \ul{ganze} Verwandtschaft, wie ich es wollte, und ob sie soweit dachte, wenigstens dem Vorsteher meiner Dienststelle eine Anzeige zu senden?
Darf ich darauf hoffen, dass man dann auch mir ein Paar Zeilen dazu schreiben wird oder sich begn\"{u}gt, ihr eine Karte oder Blumen zu senden?
\clearpage
