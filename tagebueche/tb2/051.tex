\subsection{18. Februar 1943}
%\mktitle

Bei mir in der Ecke ist Zustand X eingerissen.
Aber es verlohnt tats\"{a}chlich nicht, auch nur ein weiteres Wort dar\"{u}ber zu verlieren.
- Ein wertvolles Buch habe ich gelesen.
"Ekkehard und Uta".
Was g\"{a}be ich darum, k\"{o}nnte ich dieses Buch in ihre H\"{a}nde legen.
Aber wie wahrscheinlich aus meinem letzten Brief hervorgegangen ist, kann ich annehmen, dass nicht ein einziger Weihnachtsgruss zu Hause angekommen ist.
Ich kann es nicht \"{a}ndern.
Es ist aber wahnsinnig schwer, dazu eben verdammt zu sein auf nichts mehr Einfluss nehmen zu d\"{u}rfen.
Doppelt schwer f\"{u}r mich, da ich es gewohnt war, mir selbst mein Leben zu gestalten.
Was kann es den so sehr treffen, der sowieso bisher stets vom Leben geschoben, wenn nicht gar gedr\"{a}ngt worden ist?
Doch es hilft alles klagen{\color{red} [sic] } nichts; es gibt nur eine L\"{o}sung: hindurch.
Ich gehe diesen Weg in meiner Arbeit mit meinen besten Freunden, mit meinen B\"{u}chern.
Jetzt habe ich die Dithmarscher, danach steht ein Schillerroman in Aussicht und anschliessend werde ich mir zwei Kolbenheyer holen: Monsalvach und das gottgelobte Herz.
Ich freue mich.

\clearpage
