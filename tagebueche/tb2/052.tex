\subsection{20. Februar 1943}
%\mktitle

Es ist ein Gl\"{u}ck f\"{u}r uns, dass wir doch ab und zu - und auch f\"{u}r uns grade im richtigen Augenblick wie in der Heimat - an den erhebenden Wendepunkten der Geschicht dieses Krieges miterleben d\"{u}rfen und daran teilhaben k\"{o}nnen.
So gestern wieder der gewaltige Vortrag von "Alex" im Speisesaal (xoxoxoxoxox){\color{red} [Kurzschriftnotat] }.
Ich m\"{o}chte jetzt freudig  mit den Worten des Dichters sagen: "Es muss Fr\"{u}hling werden", und in diesem Fr\"{u}hjahr muss sich nach den jetzigen Vorbereitungen Entscheidendes ereignen.
Ich will ja so gerne hoffen, schon allein darum, dass sie auch wirklich den Mann einmal bekommet, den wert ist.
Denn wozu hat mich schon dieses Leben hier gebracht?
- Ein sehr feiner Satz ihres letzten Briefes hat mich hoch erfreut, wie sie auf die Worte "einer jungen Frau eines meiner jetzigen Kameraden" reagiert hat.
Ja, darin liegt wohl doch der Punkt, an dem sich die Geister scheiden.
Wenn ich mnur noch bald die Briefe vom Weihnachtsfest h\"{a}tte.
Ich weiss ja tats\"{a}chlich nicht, was ich in diesem Monat auf das bis jetzt f\"{u}r mich leider nur als Fragment bestehende Vorhaben der Bekanntgabe unserer Verlobung schreiben soll. -

\clearpage
