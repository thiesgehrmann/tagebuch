\subsection{12. M\"{a}rz 1943.}
%\mktitle

So bin ich nun gar nicht zum Schreiben gekommen.
Meine post hat seit dem letzten Sonnabend mich vollauf besch\"{a}ftigt.
Mein grosser Antwortbrief ist fort, auch je eine Karte an unsere Eltern.
Es ist nur traurig, dass ich alles schreiben muss.
K\"{o}nnte ich doch ein Wort nur zu ihr reden.
Am Mittwoch h\"{o}rte ich mit Meinhard zusammen - wenn auch nur von Schallplatten - Anton Bruckners 4. Symphonie, die Romantische.
Eine nette kurze Einf\"{u}hrung dazu, dann die Musik im dunklen Speisesaal, w\"{a}hrend draussen im D\"{a}mmerlicht der Tag verging.
Wie armselig sind wir doch, da wir unsere Tage zu ihrem meisten Teil bei k\"{u}nstlichem Licht erleben m\"{u}ssen!
Jener erste Fr\"{u}hlingsabend auf der Reede von Brest kam mir in Erinnerung, da ich es wieder erleben durfte, den sonnigen Tag in der alles befriedenden D\"{a}mmerung bis zum Hin\"{u}bergleiten in dunkle Nacht ausklingen zu sehen, bis ich selbst einschlief.
Und heute morgen griff ich froh nach dem gr\"{u}nen Band, den Frau Margret mir gesandt hat.
"Herz, wag's auch Du" ruft er mir immer zu, sooft ich ihn der Hand halte.
Kann das kleine Kreuz von ihr sein?

\clearpage
