\subsection{21. M\"{a}rz 1943}
%\mktitle

Fr\"{u}hlingsanfang, Tag der Fr\"{u}hjahrstag und Nacht-Gleiche.
Welch sch\"{o}ner Tag ist es fr\"{u}her gewesen.
Noch nie war es mir verg\"{o}nnt, diesen Tag bei ihr zu sein, nicht einmal unseren gl\"{u}cklichen Tag - wie sie ihn in einen Brief so recht treffend benannte - ja nicht einmal einen ganzen Jahresablauf, und sei er auch nur in seinen Feierstunden und Festtagen, durften wir zwei miteinander verleben.
Vierzehn ganze Tage uns t\"{a}glich ein paar Stunden zu sehen, das war aber auch schon das H\"{o}chste was sich uns je bot.
Das ist unsere Jugend.
Ich will nicht klagen, denn ich weiss es, - weil stets so gewollt - grade darum haben uns die kurzen Tage sch\"{o}neres und gr\"{o}sseres Erleben geschenkt, dass{\color{red} [sic] } nun die Erinnerung daran Jahre der Trennung zu \"{U}berbr\"{u}cken vermag und in ihrer Reinheit und Tiefe der zuneigung zu einander doch uns beiden die Jugendjahre leuchtend gl\"{u}ckselig \"{u}berstrahlt.
Vor zwei Jahren konnte ich ihr noch Blumen, leuchtende Freude, an diesem Tage senden.
Einen Kartengruss schrieb ich im vorigen Jahr und in diesem Jahr ist nichts mehr mir m\"{o}glich.
Und dennoch, das band unserer Liebe wird es tragen!

\clearpage
