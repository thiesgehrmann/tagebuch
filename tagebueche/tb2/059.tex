\def\day{24. M\"{a}rz 1943}
\mktitle

Fast kalendergem\"{a}ss auf den Tag hat sich der Fr\"{u}hling eingestellt.
Seit drei Tagen scheint eine kr\"{a}ftige M\"{a}rzsonne auf die meterhohen Berge von Schnee und Eis, die sich im Lager hocht\"{u}rmen.
L\"{a}ngst schon rauscht der kleine Bach unter der verharschten Schneedecke und jedes kleinste Rinnsal strebt jeder Vertiefung nach, einmal in ein breites Flussbett zu m\"{u}nden und dem ewigen Meere zuzufliessen.
Mich hat es weh ergriffen, wie ich gestern abend noch im w\"{a}rmenden Sonnenschein auf einem kurzen Rundgang in den wenigen Tagen die Welt ver\"{a}ndert fand.
Schwer duftend war die Luft geschw\"{a}ngert, Baum und Strauch hatten, unmerklich fast, einen helleren lichten Schein angenommen, der golden-r\"{o}dlich bis hellgelb-gr\"{u}nlich im Sonnenlicht aufbrach, das erwachende Leben regte sich in der Natur.
- All das kann ich allein im \"{a}ussersten Winkel des kleinen Lagers einmal still erschauen, dass es mir Erlebnis wird, das mir das Herz zu sprengen scheinen will.
Es zerstiebt und zerf\"{a}llt beim ersten rohen Laut der mir schon weit noch von den Menschen in ihren engen Baracken herschallt.


\clearpage
