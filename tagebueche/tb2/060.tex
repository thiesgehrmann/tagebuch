\def\day{29. M\"{a}z 1943}
\mktitle

L\"{a}ngst schon ist in dieser h\"{a}sslichen Gegend der kurze Fr\"{u}hlingsrausch vorbei (etwa nur 200 M. s\"{u}dlich der Hudson-Bai) {\color{red} [sic] }
Schnee ist wieder gefallen, Frost herrscht Tag und Nacht.
Und die eingefallenen Reste einst herrlicher Winterpracht stehen verschmutzt traurig umher.
- Aber einen grossen Brief will ich ihr nach Haus schreiben.
Der Stoff ist gesammelt, kurz notiert, der ganze Brief schon im Kopf festgesetzt.
Es kommt nur noch die Niederschrift und dan das grosse Kunstst\"{u}ck, durch sorgsames Abfeilen, Vereinfachen, Umstellen und Zusammenstreichen die Gedanken, die ich \"{u}ber den Ozean vermitteln will, in 24 Zeilen hineinzubringen.
Ich glaube, es werden wieder 48 Zeilen, dh.{\color{red} [sic] } ich werde mit aufeinanderfolgenden Daten, beide Briefe des Monats April inhaltlich zusammenlegen.
W\"{a}hrend ich mir meine eigenen Briefe heute morgen durchlas, sah ich doch, das sie gar manches, aus meinen Briefen oft \"{u}bergeht in der Beantwortung.
Sie hat doch Raum und Gelegenheit daf\"{u}r; ich \ul{muss} mich ja so sehr einschr\"{a}nken.
Das kommt wohl eben doch daher, dass nun mal in einem so blonden K\"{o}pfchen mit hellen blauen Augen die Gedanken \"{u}ber das Herz gehen.
Und das ist in grosser Not!
\clearpage
