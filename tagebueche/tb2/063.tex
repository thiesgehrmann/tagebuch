\def\day{10. April 1943}
\mktitle

Yum dritten Male j+ahrt sich heute schon der Tag an dem Deutsche Zerst\"{o}rer Narvik verteidigten und unsterblichen Ruhm gewannen.
Lang her scheint es schon in unserer schnellebigen Zeit.
L\"{a}ngst ist das heldenhafte Epos de Verteidiger von Stalingrad danebengetreten.
Und doch war es eine der k\"{u}hnsten Taten vor drei Jahren, die den Wikingergeist \st{aus} vergangener Jahrhunderte\st{n} leuchtend aus dumpfer Vergessenheit riss, strahlend nordisches Helden- und K\"{a}mpfertum zu erf\"{u}llen.
- Da ist ein Brief von ihr gekommen, der klingt zuerst ganz wunderbar.
Er ist schon gestern gekommen, am Abend.
Er war handschriftlich abgefasst und in seiner von einem gequ\"{a}lten Herzen geschriebenen Not f\"{u}r mich so ergreifend, wie ich wohl tats\"{a}chlich in diesem reuhen leben nicht oft ihrer gedacht habe.
Was habe ich diesen wenigen Stunden alles durchgedacht.
All meine Briefe habe ich sofort nachgelesen, die danach, nach dieser ungl\"{u}ckseligen karte, noch bald ankommen m\"{u}ssten.
Schwer wird ihr noch mein M\"{a}rzbrief ankommen, aber auch das muss \"{u}berwunden werden.
Und es wird, wenn es mir gelingt, in meinem Brief am 1. Mai sie zu \"{u}berzeugen, dass jene andere die sie meint - ich kenne gut und diese will mir eine Freundin sein - f\"{u}r mich nicht eben "meine Hilde" sein kann.


\clearpage
