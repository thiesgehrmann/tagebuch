\def\day{19. April 1943}
\mktitle

Es ist am Vorabned des F\"{u}hrers Geburtstag.
Das Lager hat zu dem Tag ger\"{u}stet, die Fenster strahlen klar, man kann mit einm Blick durch zehn Baracken sehen, denn sie sind haarscharf ausgerichtet.
Der erste warme Tag hat es auch erlaubt, einiges im Garten, d.h. dem Rasen und dem kleinen Vorraum vor der Baracke, zu tun.
Es sieht sauber aus.
Zur richtigen Frische fehlt allerdings noch ein t\"{u}chtiger Regen.
Dann  muss bald der Rasen kommen.
Der Fr\"{u}hling kann seinen Einzug halten.
Er wird aber nicht f\"{u}r uns Gefangene kommen.
Doch was macht das!
Der Umzug ist eingeschlafen, das Ger\"{u}cht vom Austausch lebte auch nicht lange.
Der erste Abschnitt im Russischen wurde heute mit den Osterferien beendet.
Damit werde auch ich meine Arbeit auf eine neue Grundlage stellen.
Der Zusammenklang der f\"{u}nf Sprachen hat mich doch in einem Sinne befriedigt: es ist auf jeden Fall durchgef\"{u}hrt worden.
(Wenn man auch einige Unterbrechungen und Verschiebungen auf die Umst\"{a}nde beziehen muss.)
Das Russisch wird beibehalten!
Daf\"{u}r ist der plan vorausschauend gefasst worden.
Die Ostertage bringen eine gr\"{u}ndliche Bereinigung der schwebenden Arbeiten.
und danach gehe ich an die Vorbereitung der Reifepr\"{u}fung, zu der mir die Aufforderung and die Arbeitsgem. mit Neuser, Prim, Pix, Weissenborn {\color{red} [gruppenfoto{\_}c266] } grad gerufen kommt.


\clearpage
