\def\day{20. April 1943}
\mktitle

Strahlender Sonnenschein leitete den Geburtstag des F\"{u}hrers ein.
Am Morgen fand ein Aufmarsch aller Kriegsgefangenen in voller Uniform auf dem Sportplatz statt.
Er war haupts\"{a}chlich daf\"{u}r um den Kanadiern ein eindrucksvolles Bild unseres geschlossenen Auftretens zu geben, wonach sie sich dann ja jedesmal ein entsprechendes Bild unserer inneren Haltung machen k\"{o}nnen.
Gleich Anschliessend{\color{red} [sic] } daran hielt der Lagerf\"{u}hrer eine Ansprache im Speisesaal.
Sie war kernig und aufr\"{u}ttelnd f\"{u}r die, welche in der jetzigen scheinbar totliegenden Wartezeit des Krieges und bei dem \"{u}bertriebenen Siegesgeschrei der Alliierten  \"{u}ber Afrika vielleicht den Kopf h\"{a}ngen lassen wollten.
Ein grosses Marschpotpourri anfangs und die Nationalhymnen im Ausklang umrahmten w\"{u}rdig die Feierstunde.
Leider habe ich heute wieder Wache.
Da kann ich aber gut lesen.
Das Buch "Prag", Werde{\color{red} [sic] } ich weiterlesen.
Ausserdem habe ich ja noch das Geschichtsbuch der Eltern da.
Und f\"{u}r die Morgenstunde m\"{o}chte ich mir gern das englische "Outlines" wieder vornehmen.
Wenn bloss die Wohnverh\"{a}ltnisse nicht so beengt w\"{a}ren.
Wie werde ich es begr\"{u}ssen  und gl\"{u}cklich sein, einmal ein eigenes Heim haben zu d\"{u}rfen, das ich jetzt schon sehs{\color{red} [sic] } Jahre entbehre.

\clearpage
