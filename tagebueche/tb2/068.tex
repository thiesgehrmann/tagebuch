\def\day{21. April 1943}
\mktitle

Leider hat sich mir bisher in meinem Leben noch keine gute Gelegenheit geboten, dass ich Beethovens "Neunte" in einem grossen Konzert einmal h\"{o}ren konnte.
Ich h\"{a}tte wohl gern zugegriffen.
So habe ich aber bis jetzt peinlich jeden Umstand vermieden, der mich gezwungen h\"{a}tte, dieses Musikwerk in einer technischen  Wiedergabe wie Radio oder Schallplatte anh\"{o}ren zu m\"{u}ssen.
Den reinen unverf\"{a}lschten Klang des Instruments, den die Meisterschaft des K\"{u}nstlers zu m\"{a}chtigen Sinfonien hervorbringt, wollte ich einer erhebenden Stunde in mich aufnehmen, wenn ich mich reif daf\"{u}r w\"{u}sste.
Wohl f\"{u}hle ich solche Reife durch das Erlebnis der Jahre in mir, ja grade die Auseinangersetzung in mir selbst mit der Gefangenschaft hat mich aufnahmebereit gemacht, ein so gewaltiges musikalisches Werk in dramatischem Erleben mir zum unverg\"{a}nglichen Eindruck werden zu lassen.
Darum lebt in mir immer noch stark mein Knabenwunsch.
Aber um so froher bin ich auch, das es mir gelungen ist, trotz der Schallplatten und der rohen Umgebung in all ihrer Zucht- und Erziehungslosigkeit mich davor abzuschliessen und allein dem Klang hingegeben den letzten Satz der Neunten Sinfonie von Beethoven in mich unverlierbar aufzunehmen.
(Er wurde mit einer Rede genau wie daheim hier zum Vortrag gebracht)

\clearpage
