\subsection{23. April 1943}
%\mktitle

Ich hatte mich auf die Arbeitsgemeinschaft in dem Kreis von Neuser, Pix, Prim, Weissenborn eigentlich gefreut.
H\"{a}tte ich sie doch als Tat begr\"{u}sst.
Nun hat man mich aber aus dreierlei Gr\"{u}nden ausgebotet.
H.W. besitzt nun auch jenes gl\"{u}ckselig-ungl\"{u}ckselige Talent, einem Menschen schonend etwas beizubringen, in dem sehr geschickt - letzten Endes doch mit der T\"{u}r ins Haus f\"{a}llt.
Na, ich kann ihm pers\"{o}nlich nicht b\"{o}se sein, bestand doch f\"{u}r niemanden eine Verpflichtung.
Denn es handelte sich doch bisher eben nur um einen gefassten Gedanken.
Aber so sieht es aus, wenn man abgeschoben wird.
Bleibt f\"{u}r mich \"{u}brig, wie bisher, nur jetzt verst\"{a}rkt weiter allein zu arbeiten.
Und warum sollte ich dar\"{u}ber gr\"{u}beln und weiter nachdenken, bin ich doch immer so am besten gefahren.
Wenn dan wirklich eine Schule kommt, k{\color{red} [sic] } geht die Arbeit sowieso auf unpers\"{o}nlicher Grundlage weiter.
Bis dahin werde ich meinen Plan im bisherigen Rahmen vortsetzen, nur nicht Intensiv Sprachen{\color{red} [sic] }, sondern im Hinblick auf die Reifepr\"{u}fung.
- F\"{u}r heute Abend habe ich wirklich einmal etwas vor: Wie vor bald 15 Jahren den I. Teil werde ich heute in einer ungest\"{o}rten Nachtwache von abends 8 Uhr bis morgens 7 Uhr den II. Teil von Goethes Faust lesen.

\clearpage
