\subsection{24. April 1943}
%\mktitle

Ich habe mir selbst Wort gehalten: Grade bis heute morgen zur Abl\"{o}sung hatte ich zu lesen.
Eine Nacht dem Schicksal abgerungen.
Und der Trag\"{o}die \st{ersten} 2. Teil habe ich aus einem Fluss in mich aufnehmen k\"{o}nnen.
Jawohl, auch der zweite Teil hat mich gewaltig ergriffen, trotz aller scheu, mit der ich all die Jahre gewartet habe.
Aber Vielleicht{\color{red} [sic]}, so m\"{o}chte ich sagen, ist unsere Generation die letzte, die Goethe noch ohne ein n\"{a}heres Studium oder Forschen in diesem Sseinen{\color{red}[sic] } Lebenswerk ganz verstehen.
Denn wenn auch einzig das Deutsche Volk gerufen wird, so Steht es doch Fest in der Form der \st{mythischen} griechischen Mythologie.
Und damit ist der letzte Durchbruch zu unserem eigenen Wesen doch noch nicht gelungen.
- Viel B\"{u}cher lese ich in diesen Tagen.
Das geht noch bis Ostern.
Nach dem Fest m\"{u}ssen greifbare Vorbereitungen f\"{u}r den Lehrgang zur Reifepr\"{u}fung getroffen werden, d.i. praktisch Teil I. u. II. des algemeinbildenden Aufbau auszugsweise durcharbeiten.
- Post ist gekommen.
Ein zweiter Brief von ihr in vier Wochen.
Und was tut sie?
Widerruft alles ihres handschriftlichen Briefes.
Nie wieder, sagt sie selbst.
Ist eben doch trotz allem mein lieber kleiner - Peter.

\clearpage
