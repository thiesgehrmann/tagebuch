\def\day{am 1. Mai 1943}
\mktitle

Wenn ich doch heute den einen einzigen Tag in einem stillen Frieden ganz f\"{u}r mich bei meiner Arbeit mit meinen Gedanken verbringen k\"{o}nnte.
Aber eine sollche stille Stunde sch\"{o}pferischer Schaffenspause gibt es das ganze lange Jahr \"{u}ber nicht.
Und wenn ich nur in meinem Tun an diesem Tage alles zu ihrem Gedenken vollbringen k\"{o}nnte, - d\"{u}rft ich es mir anmerken lassen in dieser rauhen und rohen Umwelt?
- es w\"{a}re ja Gl\"{u}cks genug, indem Gewissen ihrer zuneigung in ganzen Liebe.
ja, ich wollte mich bescheiden mit meinem Los, war es doch f\"{u}r sie, dass ich mit all den vielen Anderen hinauszog, an der Stelle, die mir der Befehl wies, meine Pflicht zu erf\"{u}llen.
Warum kam es nicht bis zum Letzten, frage ich mich noch manchmal.
Doch ha hilft jetzt nichts mehr.
Das Leben f\"{o}rdert sein Recht.
F\"{u}r mich ist es ja dazu nicht einmal das gegenw\"{a}rtige Leben; alle T\"{a}tigkeit muss ich auf Zuk\"{u}nftiges beziehen, dem \st{dazu} ausserdem in der un\"{u}bersehbaren Ungewissheit fasst alles Zielstrebende fehlt.
Es ist kein "sich bew\"{a}ren" an einer Aufgabe, nur im rein Abstrakten bleibt alle Arbeit auf sich bezogen.
Kein greifbares Ziel steht da, kein hoffnungsvolle{\color{red} [sic] } Streben.
Nur die Vernunft heischt immer wieder Arbeit an sich.
- 2 Briefe liess der Zensor in diesem Monat von ihr durch.

\clearpage
