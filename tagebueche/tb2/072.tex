\def\day{2. Mai 1943}
\mktitle

Die Post kommt, wenn man am wenigsten daran denkt.
So hatte doch gestern abend, als bis zum Kino nichts gekommen war, niemand mehr damit gerechnet.
Am wenigsten ich.
Doch fand ich drei liebe Briefe vor.
Ja, ich muss es schon gestehen, auch der dritte der nicht von ihr kam, war es, den er schloss wieder mit "lieben Gr\"{u}ssen".
Doch dazu kann ich nichts auch kann ich keine liebe Gr\"{u}sse erwidern dorthin.
Das musste nur einmal heraus.
Aber warum sollte ich mich nicht mitfreuen an so lebhaft geschilderten "Mutterfreuden".
- Ach und ihre Briefe!
Wie sind sie lieb und was haben sie alles froh zu berichten.
So von einem selbstgen\"{a}hten Schlafanzug, der "nicht zur einfach so rauf und runter", vom "Hans im Gl\"{u}ck, der es n\"{o}tig h\"{a}tte, \"{u}ber ihrem Bett zu h\"{a}ngen", - so sollte es ja nicht ganz heissen -  von Film, Theater, B\"{u}chern, der Westerlander Brautfahrt und Nannys Gl\"{u}ck.
Ja, die Jahre sind dahin, ein hartes Opfer.
Wir wollen zufrieden sein, wenn es uns verg\"{o}nnt ist, unser Leben den Kindern zu widmen und sie zu freien gl\"{u}cklicheren Menschen heranzuziehen.
Freude genug, das allein in der Erinnerung die kurze Spanne der Jugendjahre sich zu einem geschlossenen Erlebnis der Erf\"{u}llung sch\"{o}nster Tr\"{a}ume dichtet, wo alles gut und recht war, was geschehen ist.

\clearpage
