\subsection{7. Mai 1943}
%\mktitle

Sieben Tage sind im Mai schon dahingegangen.
Angangs war das Wetter sch\"{o}n, die Sonne lockte das junge Gr\"{u}n.
Doch der Regen fehlte.
Er kam.
Mit einem heftigen n\"{a}chtlichen Gewitter brach es los.
Ich war grade dr\"{u}ben auf meiner letzten Nachtwache.
Sie war furchtbar.
Das machte machte{\color{red} [sic] } wohl die schw\"{u}le gewittergeschw\"{a}ngerte (Witterung) Luft.
So war ich gestern am Tage \st{den} wie zerschlagen u{\color{red} [sic] } f\"{u}hlte eine grosse Mattigkeit im ganzen K\"{o}rper.
Die feuchtwarme Luft mit abwechselnden Niederschl\"{a}gen hielt an, auch heute noch.
Ich lag auf meiner Koje und lass in dem Geschichtsbuch von meinen Eltern.
Endlich habe ich wieder ein Werk, das mich ganz in seinen Bann zwingt und fesselt.
Ich bin froh dar\"{u}ber.
- Zur Post: Drei Briefe von ihr.
Der eine mit dem entz\"{u}ckenden Brief-"Kopf".
Eine gl\"{a}nzende Idee.
Sie f\"{a}ngt an, das Heim zu richten.
F\"{u}r mich ist es die Frage, wem schicke {\color{red} [ich] } meinen 2. Maibrief?
Ihr, ihren Eltern oder meinen Eltern?
F\"{u}nf B\"{u}cher muss {\color{red} [ich] } mir ihr durchsprechen.
- Da ist noch etwas: Ist es nicht so, das grade diese lange Trennung mir ihren K\"{a}mpfen un einander - wollen wir froh sein, das es um eins zu k\"{a}mpfen wert ist! - uns schon weiter f\"{u}hrte, als manche junge Ehe je kam?!

\clearpage
