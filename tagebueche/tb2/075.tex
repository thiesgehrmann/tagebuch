\def\day{14. mai 1943}
\mktitle

Soll mir doch nun keiner sagen, bei den Prisonern g\"{a}be nichts zu erleben.
Da ist doch vorgestern einer "escaped" - so ganz einfach von einer Aussenarbeitsgruppe fortgegangen (Hans Hilpert) - und grade gestern, als ein General zur Inspektion gekommen war, musste doch wieder einer fehlen.
So was ist doch wirklich \"{a}rgerlich, besonders f\"{u}r den "guten Onkel Kommandeur".
{\color{red} [6 Worte in Klammern sind geschw\"{a}rzt] }
Nun haben sie uns jedoch den guten hans schon gestern wiedergebracht und - seltsam - auch die Z\"{a}hlung stimmte gestern nachmittag schon wieder.
- Aber heute morgen war der H\"{o}hepunkt sportlicher Leistungen: man hat mich als linker L\"{a}ufer zu einem Fussballspiel aufgestellt.
Ich weiss nicht, ob es nun grad eine zweite oder etwa dritte oder vierte Manschaft war, jedenfalls hat es eine Menge Spass gegeben.
Und als eine Windhose \"{u}ber den Platz gefegt war, suchten 22 Mann den Ball.
Den hatte es bis an den Lagerzaun mitgenommen.
- Daf\"{u}r hatte ich auch einen lieben Brief von ihren Geburtstag, den ich schon lang erwartete.
Post und Blumen sind p\"{u}nktlich auf den Tag angekommen.
Aber welch ein Ungl\"{u}ck, nun hat sie Margret mit Johanna, "{\color{red} [sic]der Frau, "die ich liebte" und den Froschledernen [Handschuhe] } durcheinandergeworfen.
Wie hab ich herzlich gelacht; das kann ich auch gut, denn es ist auch nichts Ernstes weiter dabei.
Schluss!
\clearpage
