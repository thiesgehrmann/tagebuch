\def\day{16. Mai 1943}
\mktitle

Ein Sonntag ist Heut, jedoch kein sonnentag.
Seit dem Morgengrauen rauscht ein ununterbrochener Landregen hernieder, der alles draussen in ein diesiges, d\"{a}mmerndes Grau taucht.
Und doch ist es kein Sonntag daheim wie alle Anderen.
Es ist heut Muttertag.
Wohl brauchte man keines besonderen Tages, um der M\"{u}tter zu bedenken, und doch ist es gut, sich an einem Tage in Jahr zu besinnen, was wir alle einer Mutter, der nie vergessenden m\"{u}tterlichen Liebe zu danken haben.
Nun, ich m\"{o}chte ja herzlich hoffen, dass meiner Mutter mein Brief, indem ich zu ihnen \"{u}ber Herta schrieb, die rechte Freude brachte, die ich ihr damit gew\"{u}nscht habe.
Der Brief muss ja wohl jetzt zu Hause sein.
Eine Mutter ist noch, der ich so aus vollem Herzen dankbar sein muss, ist sie es doch, die "ihr" das Leben schenkte und ihr all ihre Sch\"{o}nheit, Anmut und Lieblichkeit mitgab, die mich nun einmal zusammen mit dem hellen, klaren K\"{o}pfchen, das so kompromisslos seinen eigenen rechten Weg durchsetzt, ganz in ihren Bann gezogen hat, dass ich sie nie mehr lassen kann.
Aber wie soll ich das sagen, dass sie es h\"{o}ren?
Wann darf ich es tun?
Ach, da muss jetzt noch mein M\"{a}rzbrief kommen, der ja wohl zu Weihnachten noch eben ging, aber nun fast all seine Berechtigung verlogen hat.

\vfill
{\color{red} [Das Schreibheft wird nun gewendet und wiederum nur alle rechten Seiten beschrieben] }

\clearpage
