\def\day{17. Mai 1943 *}
\mktitle

Vor zwei Tagen ist etwas geschehen, ein Geringf\"{u}giges an sich, doch gewaltig unter unseren Verh\"{a}ltnissen.
Die Meisten schenkten schon damals den grossen sch\"{o}nen Gem\"{a}lden leider nur allzuwenig beachtung, als sie uns vom Deutschen Roten Kreuz geschickt worden waren und an den W\"{a}nden unseres Speisesaales aufgeh\"{a}ngt wurden.
Jetzt ist der Speisesaal, nachdem er in hellgelber Farbe g\"{a}nzlich neu gestrichen wurde, durch die Auschm\"{u}ckung der B\"{u}hnenwand vor der K\"{u}che zu einem grossartigen Gemeinschaftsraum gestaltet worden.
\"{U}ber einem grossen F\"{u}hrerbild (Kopf) steht leuchtend das Hoheitszeichen des Grossdeutschen Reiches.
Rechts und links vom F\"{u}hrerbild h\"{a}ngen ebenfalls in silberbronziertem Holz wie das Hoheitszeichen, gerahmt zwei F\"{u}hrerworte: "Wer leben will, der k\"{a}mpfe also, denn wer nicht \st{leben} k\"{a}mpfen will in dieser Welt des ewigen Ringens, verdient das Leben nicht!"
Und das andere: "Im Soldaten werden Nationen als zu leicht befunden und zum Untergang bestimmt, oder als Wert gefunden, neues Leben zu tragen."
Solche Ausgestaltung hier, ist schon eine Tat an sich!

\clearpage
