\def\day{22. Mai 1943}
\mktitle

Oh, nein, ich vergesse nichts; so auch nicht welche Tage in der letzten Woche vergangen sind.
Lebendig und unverg\"{a}nglich stehen sie vor meinem inneren Auge mit all ihrer Zartheit und der aus dem einen Abend sich voll erschliessenden - eben weil mit einander errungen - so selbstsicheren Unbeschwertheit einer gluckseligen Jugend.
Aber auch in ihrem Erleben einer ersten wahren Liebe von beiden inmitten eines der sch\"{o}nsten Deutschen Lande, dort oben an der F\"{o}rde, wo im Sp\"{a}tfr\"{u}hjahr und Vorsommer die N\"{a}chte nicht mehr dunkel werden, es immer klar von beiden Meeren \"{u}ber das Land weht.
- Am Montag soll ja unsere Schule mit der Aufnahmepr\"{u}fung beginnen.
Ganz f\"{u}r mich begr\"{u}sse ich das als eine Tat.
Es soll also wieder planvoll gearbeitet werden.
Das Ziel freut mich, und wenn es erreicht wird, ist ein Gewinn offensichtlich.
Ob es mir zeitlich und auch sonst m\"{o}glich sein wird, den Russisch-Unterricht weiter mitzumachen?
Ich m\"{o}chte ihn ungern aufgeben.
In den vergangenen f\"{u}nf Monaten machte er mir die meiste Freude.
Wir haben eben gearbeitet.
- Am letzten Mitwoch h\"{o}rte ich ein Brahms-Konzert von Schallplatten.
Ich finde Musik eben nur sch\"{o}n u. befasse mich nicht n\"{a}her mit einer Analyse.

\clearpage
