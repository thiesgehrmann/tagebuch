\subsection{27. Mai 1943}
%\mktitle

Diese Woche hat Abwechslung gebracht.
Montag nachmittag 3 Stunden Mathematikpr\"{u}fung (ich konnte mich eines L\"{a}chelns nicht ganz erwehren: denn alles kann man nicht wissen, aber wohl kann einem mit Gottes und anderer Leute Hilfe im rechten Augenblick das Richtige finden!), Dienstag fr\"{u}h 4 Stunden Deutschen Aufsatz (ganz ungewohnt solch eine anstrengende geistige Arbeit), Mittwoch fr\"{u}h 2 Std. Franz\"{o}sisch und am Abend 2 Stunden Englisch.
Sehr haben mir die einleitenden Worte des Professors gefallen: "Wir werden Sie vor eine kurze Pr\"{u}fung stellen, dass Sie nachweisen, ob Sie auch tats\"{a}chlich die Reife besitzen, nach einem so hohen Ziel zu langen."
Das war eins.
Und dann im Nachsatz: {\color{red} ["] }Es ist selbstverst\"{a}ndlich, dass Sie selbst\"{a}ndig arbeiten.
Wer es nicht tut und sich erwischen l\"{a}sst, hat bewiesen, dass er diese Reife nicht besitzt."
Ist so etwas nicht k\"{o}stlich?
Leider hat er uns alle nun auf eine Folter gespannt: Wen er bis zum Sonntag den 30. Mai, nicht zu sich gerufen hat, der nimmt am Jahrgang teil und findet sich bei der Besprechung des Lehrplanes am Sonntag nachmittag ein.

\clearpage
