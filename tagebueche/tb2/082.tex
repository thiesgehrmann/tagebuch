\def\day{30. Mai 1943}
\mktitle

Weil gestern seit Tagen mal wieder die Sonne sich hervorwagte, haben wir gleich den Tennisplatz fertiggemacht.
Meinhard kam hinzu mit der Neuigkeit seiner Bestallung zum "Landwirtschaftsminister".
Na, ich werde zur Schule gehen, denn die Aufnahmepr\"{u}fung ist glatt abgelegt worden.
(Ich h\"{a}tte mich anders aber auch schwer gewundert.)
Gestern und Heut \"{u}berschlugen sich die Ereignisse noch einmal.
H\"{u}bsch eins nach dem anderen.
Unter herzlich befreienden{\color{red} [sic] } Lachen habe ich das Buch von der geteilten Wohnung gelesen
Einfach nett.
Zum Abend ein Grossfilm mit "Marlene": The Flame of Neworleans.
Ihr Spiel gef\"{a}llt mir immer wieder, kann's nicht anders sagen.
Der Sonntag ist auch ausgef\"{u}llt.
Drei gute grosse Fuss- und Handballspiele finden am Vor- u Nachmittag statt.
Vor dem Essen ein kurzer Besuch mit Erich bei Meinhard.
Zum Nachmittagsgeburtstagskaffee hat mich Kurt Techen eingeladen.
Alle Unteroffiziere der "Egerland" hat er eingeladen.
Ein guter Gedanke.
Und zum Abend soll ein Theaterst\"{u}ck steigen.
ch glaube nicht, dass ich hingehe, denn ich brauche eine Zeit innerer Sammlung f\"{u}r morgen.
Fast geht es mir grad wie in der Schulzeit, wo ein seltsames Angstgef\"{u}hl mich stets nach den Ferien die Schule f\"{u}rchten liess.
Ich wusste nich warum, aber es hat mich wohl doch immer vor der Untersch\"{a}tzung bewahrt.

\clearpage
