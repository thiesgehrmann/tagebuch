\def\day{14. Juni 1943.}
\mktitle

Jetzt wird das Leben langsam aber sicher ganz furchtbar hier.
Abgesehen vom Wetter, das von Stunde zu Stunde wechselt vom hellsten Sonnenschein mit 30° bis zum Wolkenbruch um 0°, gehen die menschen vollkommen parallell damit.
Die primitivsten einfachen Formen, ja fast nur Urinstinkten wird gelebt.
Der umgang ist mir seit Jahren schon verhasst; die gehen darin auf und ich kann bei Gott kaum noch Gegenkr\"{a}fte dagegen aufbringen. Fr\"{u}her konnte ich sie aus meinem wirklich gelebten Leben nehmen, das mich immer wieder mit Menschen zusammenf\"{u}hrte.
Wo andere von vornherein versagten oder - selbst wenigsten so schlau - verzichteten, da konnte ich anfangen.
Aber das war einmal.
Wie so vieles, was einmal war.
Da ist das Deutsche Opernhaus, wo ich heute nacht ganz deutlich Karten f\"{u}r die Bohème kaufte.
Ich hielt sie in meinen h\"{a}nden mit all der Vorfreude auf den Theaterabend, die er mir in meiner Heimatstadt nur immer zu bringen vermochte.
Klang da nicht im Unterton die "Aïda" mit. oder war des "Rosenkavalier", "Figaros Hochzeit"?
Oder waren es Erinnerungen, die immer noch mitschwingen?
Ein gemeinsamer Schulw{\color{red} [e] }g mit einem kleinen noch ganz jungen M\"{a}dchen.
15 Jahre z\"{a}hlte sie.
Immer war sie f\"{u}r mich da, w\"{a}hrend ich durch Deutschland zog.
Bis ich es f\"{u}hlte, dass sie wartete, als sie es nicht mehr durfte.
Und trotzdem will sie es nie vergessen.
Warum nicht?

\clearpage
