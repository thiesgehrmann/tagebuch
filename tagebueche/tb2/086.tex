\def\day{28.6. 1943.}
\mktitle

Rasend vergeht die zeit.
Vor drei Tagen kam ihr Brief.
Zuerst glaubte ich, dass ich ihn bestimmt \"{u}berwinden werde.
In meinem Brief, den ich schon rein gef\"{u}hlsm\"{a}ssig bis zu ihrem Brief hinausschob, scheint es mir auch noch gelungen.
Aber heute zum Abend, da ist mir, als ginge das letzte F\"{u}nkchen Hoffnung auf alles in der Welt aus mir heraus.
Ist  doch sie mein Glaube und mein ganzes Hoffen.
Und nun will sie sich g\"{a}nzlich aus unserer Gemeinschaft stellen, weil ich wohl zu hart in meinem Brief gewesen bin.
Ja, der Brief ist geschrieben.
Aber ich muss t\"{a}glich hier einen fast aussichtslosen Kampf wie jeder der 1600 Mann im lager um nichts ausstehen, dazu ein st\"{a}ndiger innerer Kampf um das Schicksal, mit dem Schicksal \"{u}ber dem oft genug ein stummes warum{\color{red} [sic] } sich einschleichen will.
Es muss auch runtergek\"{a}mpf{\color{red} [t] } werden.
Wie kann ich da mich noch innerlich gegen sie stellen?
Auch weiss ich es ganz genau, wenn sie es nicht will, ich kann ihr nach \st{fast} den verlorenen jahren jetzt nichts bieten, womit ich sie halten k\"{o}nnte, falls sie eine innerliche Losl\"{o}sung vollzogen haben sollte.
Und das eines Briefes wegen?!? -
Ach ich sehe heute komaum noch einen Weg.
Der Tag war h\"{a}sslich.
Ich will schlafen.
Wenn doch nur bald etwas gesch\"{a}he!!!

\clearpage
