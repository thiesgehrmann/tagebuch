\subsection{13. Juli 1943.}
%\mktitle

Die Zeit verrennt{\color{red} [sic] }, aber sie verrinnt nicht mehr.
Wohl bin ich wegen der Schule siet \"{u}ber 2 Wochen nicht mehr zum Schreiben gekommen.
Eine Menge w\"{a}re ja zu berichten.
So dass am 30, juni noch sechs Briefe gekommen sind.
Davon drei von ihr, zwei noch aus April - ich hatte doch meine Freude zu sehen, wie gut es war, dass sie meinen M\"{a}rzbrief den Eltern gezeigt, aber noch weit mehr; dass sie mir schreiben konnte, es selbst gef\"{u}hlt zu haben, warum ich diesen Brief ihr wohl nur so schreiben konnte.
Ja, so ist auch dies \"{u}berwunden.
Und ich bin mir nicht b\"{o}se, dass ich es wagte, eine solche Frage brieflich \"{u}ber Tausende von Seemeilen an das Schicksahl zu stellen.
Sie ist gel\"{o}st und das ist besser so.
Seitdem warte ich auf den am 8.5. erw\"{a}hnten Maibrief von Ersten, der mir angesagt ist.
Bis jetzt ist noch nichts gekommen.
ob er mir das grosse Bild, das von ihr versprochen ist, bringen wird?
Das ist leider nicht ganz klar aus ihren und den Briefen ihrer Eltern hervorgegangen.
- Jetzt h\"{a}tte ich ja noch kein einziges Wort \"{u}ber meine Ernennung zum Inspektor apl. {\color{red} [ausserplanm\"{a}ssig] } niedergeschrieben.
Ich war ja erst so erstaunt, dass ich es gar nicht fassen mochte, aber nun ist mir der Gedanke ganz vertraut geworden.
So ganz nebenbei m\"{o}chte ich ja einmal doch den spass auskosten, hier noch oder auf oder nach der Heimfahrt ganz kurz eine Gastrolle zu geben.

\clearpage
