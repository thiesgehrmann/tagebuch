\subsection{15. Juli 1943.}
%\mktitle

Eine postkarte habe ich heute geschrieben.
Eine ganz gew\"{o}hnliche postkarte mit einem Gruss an den Vorsteher und die Herren der Standortverwaltung M\"{u}rwik.
Aber weil diese karte an meinen Dienstvorgesetzten doch so halb amtlichen Charakter hat, habe ich bei dieser Gelegenheit zum erstenmal meine Unterschrift mit meinem neuen Dienstgrad versehen.
ich bilde mir gewiss nichts ein, aber wenn ich doch ab und an ein staunendes Gesicht eines ... sehe, na, da habe ich eine kleine Genugtuung.
Doch beiseite damit.
Jetzt sind es genau 2 Wochen, dass ich auf den einen Brief vom 1. Mai warte.
Ich m\"{o}chte schreiben, ihr unendlich viel erz\"{a}hlen, ein klein wenig von hier, den Kirschen, den Hitzetagen oder so.
Aber viel, viel mehr von einem urlaub an der See, von diesen Tagen unseres fast ungetr\"{u}bten reinen Liebesgl\"{u}ckes, - das eine Bild von ihr am Strande ersehne ich so sehr - auch von einem ganz einfachen Gang nur durch die Strassen und Gassen der Stadt an der F\"{o}rde, oder nur von einem Gang durch den Wald, auf dem wir beide gehen, ohne ein festgestecktes Ziel vor Augen zu haben, nur das Leben des Waldes einfangen wollen und sich freuen, freuen an der Welt, an ihrer Sch\"{o}nheit und an unserem Jungsein.
Nichts weiter.

\clearpage
