\subsection{24. August 1943}
%\mktitle

In der Zwischenzeit lag un die Schule, Jawohl{\color{red} [sic] }, die Schule, f\"{u}r die ich tats\"{a}chlich gelernt habe und so ganz eigentlich auch nur f\"{u}r die Vierteljahresabschlussarbeiten.
Man kann auch ganz einfach sagen, dass man diese Abschlusswoche vor den vierzehn Ferientagen nicht so ohne weiteres \"{u}berstanden h\"{a}tte, wenn da nicht ein volles Vierteljahr systematischer Schularbeit voraufgegangen w\"{a}re.
Der \st{g}Geist w\"{a}re wohl ganz einfach zu tr\"{a}ge gewesen.
Nun hat man ja auch die drei Mann ganz einfach vom A-Lehrgang abkommandiert.
Ob das wirklich mit der Leistung f\"{u}r die Gemeinschaft begr\"{u}ndet ist?
Ich habe mich ja einmal mit einer F\"{u}hrerin aus dem RAD. stundenlang \"{u}ber das Thema Gemainschaft und Pers\"{o}nlichkeit \st{entha}unterhalten.
- Da ist nun sie irgendwo am Strand gewesen, ganz allein hat wohl zwei Wochen lang im Sande gelegen, faul hingestreckt, und ihren ranken, schlanken M\"{a}dchenk\"{o}rper den Sonnenstrahlen hingegeben.
Irgendwo in den D\"{u}nen \st{A} an der Ostsee oder an der Nordsee.
Wik{\color{red} [sic] } auf F\"{o}hr, St. Peter oder Fehmarn.
Ich g\"{o}nne ihr diese Sonnenzeit von ganzem Herzen.
Ist sie doch grad so arm wie ich in ihren jugen M\"{a}dchenjahren.
Ob sie mir von dieser Zeit einmal t\"{a}glich einen Brief schreiben wird?
Ich glaube es ja nicht, das ist ihr eben nicht gegeben.
Ich verstehe es.

\clearpage
