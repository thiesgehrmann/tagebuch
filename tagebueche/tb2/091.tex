\subsection{5. September 1943}
%\mktitle

Verregnet ist der letzte Sonntag der Sportwoche, dahin mit einem Schlage der ganze Zauber, den herbstlich goldene Sonnenstrahlen selbst auf unsere armlichen{\color{red} [sic] } Gefangenenbaracken zu malen wussten.
Gottes Segen trifft arm und reich.
Aber nicht verflogen sind mir jene holden Gedanken, die ich in meinem Brief einfing.
Ich glaube, ich werde ihn absenden.
- Es ist nach dem Essen jetzt, alles fast pflegt der Mittagsruhe auf unserer H\"{u}tte, eine himmlische Stille herrscht.
Da habe ich mein Bild von ihr hervorgeholt, habe es lange und lange angeschaut.
Ich wusste gleich, als ich es aus dem Paket entnahm, dies Bild werde ich oft anschauen m\"{u}ssen, dis ich es ganz in seinem tiefen, so lebenden Wesen werde erfassen k\"{o}nnen und bis es mir dann auch ganz und gar gefallen wird.
Nun bin ich schon bedeutend n\"{a}her daran.
Es spricht stark zu mir.
Mein Herz jubelt, ja sie ist es!
Nicht spurlos sind die Trennugsjahre an ihr vor\"{u}bergegangen und grade weil dieses Bild aus der Zeit des gr\"{o}ssten seelischen Kampfes ihres Alleinseins kam, den Briefe hervorriefen und an dem weniger starke Seelen gewiss zerschellt w\"{a}ren, zeigt es mir klarer noch als ihre lieben Briefe, zu welcher H\"{o}he des Glaubens und der Liebe der Sieg in diesem Kampf uns f\"{u}hrte.
Ja, ich liebe Dich, so wie Du bisst und verehre den wissenden Stolz und die frauliche Anmut in der jugendlichen Sch\"{o}nheit deines Bildes, das ich im Herzen trage.

\clearpage
