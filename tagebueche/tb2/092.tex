\subsection{7. September 1943}
%\mktitle

Und heute war Umzugstag.
Ich bin nicht umgezogen.
Man kann es jetzt bequem aushalten in unserer Nock mit Hein Finger, k\"{u}rzlich Stuben\"{a}ltester, Erich Schmidt, der Nachfolger Meinhard Mildes, und Erwin Klinke, der k\"{u}rzlich f\"{u}r P... eingezogen ist.
Wir haben schon wieder einige nette Nieselnachmittage verlebt.
Ausserdem hat uns der Umzug eine ganze Dielenbreite mehr platz gebracht.
Es gab auch gleichzeitig eine g\"{u}nstige Gelegenheit, einige inzwischen notwendig geworden Reorganisationen vorzunehmen, als da gewisse Schwierigkeiten mit der Fussbank des Liegestuhls auftraten.
Sie sind behoben.
Nun soll ich f\"{u}r zwei Bilder, die grossen, Leisten f\"{u}r Rahmen vor Karl Bekker bekommen und Stoff werde ich auch organi\st{e}sieren, damit wir uns unsere Ecke f\"{u}r die Winterzeit etwas behaglich ausstatten k\"{o}nnen.
So verbleiben von der zus\"{a}tzlichen Ferienwoche noch ein paar Tage, um nach solchen Umstellungen die Arbeit f\"{u}r die Schule aufzunehmen.
Wenn es auch fraglich ist, ob der geregelte Schulbetrieb sobald aufgenommen wird, so werde ich doch f\"{u}r mich das begonnene Arbeitspensum f\"{u}r die Reifepr\"{u}fung weiterf\"{u}hren.
- Ach, wenn sich nur einmal hinschauen k\"{o}nnte, mit welchen K\"{a}mpfen wir unser Leben hier um das kleinste Ding f\"{u}hren m\"{u}ssen und wie einen so ein kleiner Gewinst{\color{red} [sic] } erfreuen kann. Es ist ja auch kaum zu glauben.

\clearpage
