\def\day{10. Oktober 1943}
\mktitle

Nun habe ich lange ausgesetzt mit dem Schreiben.
Es ist eine Menge geschehen inzwischen.
Da war erst der Kampf mit dem Kommandeur, den wir gewannen.
F\"{u}r fast acht Tage getraute sich der Tommy nicht, eine Z\"{a}hlung zu unternehmen.
So kam es durch diesen Tumult das sich der Schulbegin bis zum 29. September verz\"{o}gerte.
Als sie dan begonnen hatte - und der unterricht lief gut an; man merkte es deutlich, das die Klasse auf ein Vierteljahr gemeinsame Arbeit zur\"{u}ckblicken konnte - war der ganze Lehrgang pl\"{o}tzlich gef\"{a}hrdet wegen der Dummheit und Kurzsichtigkeit einiger Unentschlossener.
Dank der Entschlossenheit Erhard Trenklers steht der Lehrgang.
Ich war ganz froh, hinterher noch einmal mit dem Professor (Fw. Wulf) dar\"{u}ber sprechen zu k\"{o}nnen.
Inzwischen hat sich ein neues Drama angebahnt mit meinem Untermieter.
Auch so ein Vertreter der Wahrheitsager, die alles so reden "Wie es wirklich ist".
Nun, ihm ist nicht viel zu helfen.
Er muss eben noch viel Lehrgeld zahlen.
Vielleicht wird er auch nie klug werden, der er ist von Natur aus nicht mit einem zu kritischem Geist begabt.
- Und sie schickt mir ein halbes Testament, "wenn ihr etwas zustossen solte. Ich komme ja zur\"{u}ck."
Ich bin best\"{u}rzt.
Sieht es so aus in Deutschland?
Was hilft's: wir m\"{u}ssen siegen!

\clearpage
