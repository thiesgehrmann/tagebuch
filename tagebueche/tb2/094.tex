\subsection{12. Oktober 1943.}
%\mktitle

Der Dienstag ist der schulfreie Tag des Abiturientenlehrgangs.
Er ist f\"{u}r Lagergemeinschaftsarbeiten gedacht, so wie es sich im Sommer ja auch in sch\"{o}ner Weise \st{sich} gezeigt hat.
So h\"{a}tte ich heute morgen gut Zeit gehabt, einige freie Arbeiten zu erledigen.
Jedoch lockte zum ersten mich das ganz besonders sch\"{o}ne Wetter heraus aus der staubigen Baracke und zum andern hatte ich eine grosse inere Unruhe in mir, die mir ein geistiges Arbeiten jeglicher Art unm\"{o}glich machte.
Da ist der Trubel der abziehenden Holzf\"{a}ller, - ganz abgesehen von dem komischen Verhalten eines gewissen Herren - kommt die zweifelhafte Haltung des Lehrganges mit der wahrscheinlichen M\"{o}glichkeit einer zusammenlebung der verbleibenden Barackenbelegschaften hinzu.
Und ausserdem liegt mir immer noch ihr letzter Brief im Sinn, der mir so deutlich von den grossen Opfern sprach.
Aber nicht nur das allein.
Es liegt in ihren schlichten S\"{a}tzen eine solche Selbstaufgabe in ein wohl unvermeidliches Geschick, das sie schon in Gedanken ein Opfer auf sich nimmt, das wohl m\"{o}glich sein k\"{o}nnte an das ich hier aber gar nicht denke, noch denken mag.
Ich muss sie von diesen Gedanken abbringen.
Wie, das ist die Frage.
Ob es mir gelingt?
Ich w\"{u}nschte es.

\clearpage
