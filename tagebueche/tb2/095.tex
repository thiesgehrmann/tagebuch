\subsection{17. Oktober 1943}
%\mktitle

Der erste Schnee ist gefallen.
Da werde ich gliech nacher ein paar Brat\"{a}pfel auf dem Ofen schmoren.
Denn welch' gl\"{u}ckselige Erinnerung h\"{a}ngt daran.
Traumland, Kinderland!
Mit all seinen Freuden und kleinen Schmerzen.
Ich erinnere mich ganz deutlich des grossen Jubels beim ersten Schnee, wenn uns die Mutter weckte.
Ein ganz neues helles Licht, fiel von den umliegenden D\"{a}chern in das Zimmer und auf den Strassen versanken die l\"{a}ute{\color{red} [sic] } Ger\"{a}usche des Verkehrs in dem weiten weichen weissen Teppich, der auch in der Grossstadt seine m\"{a}rchenhafte Wirkung nicht verfehlte.
- Als ich Gefreiter wurde, n\"{a}hte ich mir den Winkel auf den \"{A}rmel und ging stolz zu meinem M\"{a}dchen.
Ich freute mich ganz einfach dar\"{u}ber.
Als ich Obergefreiter werden sollte, \"{u}berlegte sich meine oberste Dienststelle dies sehr, sehr lange, was mit dem K\"{u}stendienst zu tun sei und teilte mir nach 10 Wochen auch schon den Bescheid mit.
Meine Bef\"{o}rderung zum Maaten erreichte mich im besetzten Gebiet.
Ich nahm sie zur Kenntniss und \"{a}nderte den Absender.
Wie andere tagelang dar\"{u}ber feiern konnte, verstand ich schon nicht mehr.
Was tut nicht einer, wenn er Angestellter wird, ja Beamter und noch dau Inspektor.
Ich bin's seit 2½ jahren und kann allenfalls zu einem Vertrauten davon sprechen.
Darf ich mich denn nichd auch ein bischen \"{u}ber etwas freuen?
Ich t\"{a}t es ja auch so gerne.

\clearpage
