\def\day{19. 10. 1943}
\mktitle

Ich schaff nichts zur Zeit.
Auf der einen Seite ist seit dem gestigen Tage eine frohe Erregtheit in mir, ich muss fortw\"{a}hrend an den letzten Brief von ihr denken und habe so meine eigenen Gedanken dazu.
Zumal \"{u}ber den einen Satz: "... aber Du kannst Dich nicht davon frei machen und so f\"{a}llt es mir auch immer schwehr!"
Kann ich ihr dem in diesen vorgeschriebenen Gefangenenformulier von meiner Gef\"{u}hlen schreiben, und warum kann ich ihr nicht einen ganz herzinnigen Brief schreiben?
Und wenn ich es selbst k\"{o}nnte, verm\"{o}chte dem solch ein Brief ihr eine Befriedigung zu geben?
Steht denn nun nach 2½ Jahren nich auch bei ihr viel zu gross die Frage auf: "Wann endlich wird er kommen k\"{o}nnen, mich gl\"{u}cklich zu machen, mich mit seiner Liebe zu erf\"{u}llen, an die ich all meine Hingabe verschwenden d\"{u}rfte, ohne nach einem Was und Wie zu fragen, nur getragen von dem seelischen und k\"{o}rperlichen Wollen, im Augenblick aufzugehen und in allen Wonnen in seinen Armen unterzugehen?
Wer kann diese Frage beantworten, und die weit wird unser Schicksal einmal davon abh\"{a}ngen?
- Ich reisse mich t\"{a}glich zusammen.
Da sind die Nachrichten,
Was ist wahr, was ist L\"{u}ge?
Wie sieht es imn Osten aus?
Wann kommt der Schlag, auf den wir alle sehnsuchtsvoll warten?
Fragen \"{u}ber Fragen und selten eine Antwort.
Allein der Glaube muss mich f\"{u}hren.
Ich will mich ihm willig anvertrauen.
Der Sieg muss unser sein.
- Und ich will f\"{u}r meine Schule hier arbeiten.

\clearpage
