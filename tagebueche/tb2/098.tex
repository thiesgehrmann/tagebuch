\def\day{24. Oktober 1943}
\mktitle

Ein seltsames Erlebnis hatte ich gestern abend.
Eigentlich ja nur seltsam, weil ich nun wieder einmal nicht im Umgang mit Menschen aufpassen konnte oder weil ich eben doch wieder glauben sollte.
Dabei h\"{a}tte ich es mir doch von vornherein selbst sagen k\"{o}nnen, konnte ich sie doch im Umgang mit anderen Menschen gen\"{u}gend beobachten.
Es ist vor allen Dingen wieder einmal mehr die alte Tatsache: man glaubt sich zu kennen, man trifft sich zu sehr im Alt\"{a}glichen und schon ist die sich selbst auferlegte R\"{u}cksichtnahme und Zur\"{u}ckhaltung dahin, geschmolzen an der W\"{a}rme der allt\"{a}glichen Klein-Reibungen wie das Eis an der  Fr\"{u}hlingssonne.
Nun ist das ein Lehrfall mehr f\"{u}r mich und ich trage keine{\color{red} [sic] } allzu starken Verlust daran.
- Jetzt ist es zwei Stunden sp\"{a}ter.
Die Sonne hat sich hervorgewagt und ich habe einen Spaziergang - sprich Rundendrehen - von genau denn zwei Stunden hinter mir.
In einem \st{Mathematik} angeregten Gespr\"{a}ch mit unseren Mathematiklehrer, dem Gefreiten Henrichsen, \"{u}ber unseren Abiturientenlehrgang, und da vor allem \"{u}ber die angesetzte Dauer, hat sich in mir auch so einiges gel\"{o}st, was eine kleine Verbitterung in mir hervorrufen wollte.
Ich freue mich dar\"{u}ber und lache mit dem Sonnenschein der tr\"{u}ben Gedanken.
Arbeite Du, Hans, still f\"{u}r Dich allein.
Du wirst immer sowieso als Mensch allein bleiben m\"{u}ssen.
Sei froh und im Stillen gl\"{u}cklich, dass dir ein Mensch seine ganze Liebe hingeben will.
Sie wird dich immer dadurch allein ganz verstehen k\"{o}nnen.

\clearpage
