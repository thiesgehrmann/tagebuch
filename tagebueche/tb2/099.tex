\subsection{9. 11. 1943.}
%\mktitle

Nun habe ich sehr lange nichts mehr geschrieben und seit noch zwei Tagen fr\"{u}her habe ich die letzte Post bekommen.
Es ist zu vertrackt.
Grade auf den n\"{a}chsten Brief von ihr, der nach dem 30. August geschrieben werden sollte und somit das Ergebnis ihres Besuches in M\"{u}rwik bringen m\"{u}sste, grade auf diesen Brief muss ich wieder unverh\"{a}ltnism\"{a}ssig lange warten.
Dann kam mein Geburtstag.
Ich hatte ihr an diesem Tage schreiben wollen, wie ich es wohl von ihr auch gedacht h\"{a}tte.
Es ist so vieles dazwischengekommen.
Wenn das nun bei mir geschehen kann, um wieviel mehr ist dazu die M\"{o}glichkeit gegeben, das es ihr ebenso geht.
Ich werde darum nicht mehr im Stillen darum rechten, obwohl ich es mir immer w\"{u}nschen werde.
So will ich noch berichten, wie es in diesem Jahre war.
Hein Finger, mit dem ich auch auf der neuen Baracke in einer "gem\"{u}tlichen" - soweit es die Umst\"{a}nde nat\"{u}rlich erlauben - Ecke zusamme{\color{red} [n] }wohne, war der erste Gratulant, der mich aufmerksam am Morgen \"{u}berraschte.
Dann war ich zwei Stunden in der Schule, zwei schw\"{a}nzte ich einfach, und am Nachmittag fanden sich alle sorecht nett ein.
Mit dem letzten Rest des Tees konnte ich ihnen grad noch einen Willkommentrunk bieten.
Am Sonntag drauf gab es mit 14 Mann hoch eine kleine Nieselrunde.
Es war etwas anderes als alle Tage und bei etwas Kuchen und Tee wurde der Nachmittag verplaudert.
Zigaretten hatten wir in menge.
Gestern war der Umzug.
Ich hatte eine Wut dabei in mir dar\"{u}ber und heute tun mir auch alle Knochen weh.

\clearpage
