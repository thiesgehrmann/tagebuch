\def\day{14. November 1943.}
\mktitle

Der Winter ist da.
Der Schnee bleibt liegen, der Frost klirrt.
Fast ohne \"{u}bergang sind es mit einem Male 18°C unter Null geworden.
Der Schulbetrieb l\"{a}uft wahrhaftig auf Hochturen.
Dazu sind noch die 2 weiteren Unterrichtsstunden gekommen, in dennen ich Franz\"{o}sisch in der neuen Klasse gebe.
Aber das geht ohne weiteres ganz gut.
Seit Tagen steht wieder einmal das graue Gespenst Gefangenschaft ganz riesengro{\ss} vor mir.
Ich hatte Geburtstag.
Bisher war es mir noch immer gelungen, ihr zu diesem Tag einen lieben Gru{\ss} zukommen zu lassen, so da{\ss} sie das Gef\"{u}hl meines ganz besonders lieben Gedenken hatte.
Ich darf das alles nicht mehr haben.
Doch darf ich klagen?
Nein, aber ich kann mich nicht des Gef\"{u}hls erwehren, dass sich seit einiger Zeit schon anstelle aller Klagen ein{\color{red} [sic] } grosse ironisch kaltl\"{a}chelnde Resignation und ein bei{\ss}ender Spott breitgemacht haben.
Das w\"{a}re nun etwas, was man lebhaft bek\"{a}mpfen m\"{u}{\ss}te.
Kann man das auch noch in einem Leben, wo man nicht einmal das Geringste mehr als eine selbst\"{a}ndige eigene Handlung vornehmen darf?
O, ich bin keineswegs an einem Ende, aber was wird sie sagen, wenn sie diese Ver\"{a}nderung an mir bemerken wird?
- Und alles kommt formlos, tonlos, leblos in einem neuen Jahr der Gefangenschaft wieder: Weihnachten, der kurze Sommer, die Geburtstage; aber im R\"{u}ckblick ist nichts Gewesenes mehr da, dass den Gedanken lieb verweilen liesse.
7 Jahre schon.
Wie lange noch?

\clearpage
