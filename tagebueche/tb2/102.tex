\subsection{5. Dezember 1943.}
%\mktitle

Es ist furchtbar kalt heute.
Am Morgen waren es -30°C.
Jetzt zum Mittag ist es etwas milder geworden.
Der Schnee liebt blendend wei{\ss} drau{\ss}en und glitzert in der winterlichen Mittagssonne.
Und was das Sch\"{o}nste am ganzen Tage ist, das ist mein Herz voller innigster Freude.
(Nur darf ich es keinem anmerken lassen!)
Denn nach 6 Wochen ist Post f\"{u}r mich gekommen.
10 Briefe auf einmal.
Ich lie{\ss} das Kino fahren, machte mir en Glas Tee zurecht und \st{rauch} las meine Briefe.
2 von den Eltern aus Dresden.
Wie war ich Froh das sie dort etwas Ruhe u. Friedlichkeit hatten.
1 war von Mutti aus Flensburg.
Aber die 7 anderen alle von Ihr.
Oh, ich k\"{o}nnte jubeln.
Schon allein \"{u}ber die Bilder.
Bei dem Brief, der mir den gew\"{u}nschten Bescheid aus M\"{u}rwik brachte - es ist mir gleich, wie es nun steht, ich wollte nur Klarheit haben; - mu{\ss}te ich mir erst eine neue Zigarette anz\"{u}nden.
Vor Staunen kam ich nicht weiter, als das Foto herausgefallen war.
Zwei entz\"{u}ckende seidenbestrumpfte Beine die sich nicht zu sch\"{a}men brauchen, da{\ss} sie zeigen wie sie \"{u}ber der weichen Rundung des Knies bis zum Strumpfansatz weitergehen, fragten s\"{u}{\ss} versch\"{a}mt, ob ich w\"{u}sste zu wem sie geh\"{o}ren m\"{o}chten.
Nein, dieser Schlingel, ich m\"{o}chte ihn an den Ohren zausen ob einer solchen "frechen" R\"{u}pelei.
Aber, ich f\"{u}rchte, wenn ich da w\"{a}hre, w\"{u}rde dieser kleine Racker nur ganz still mich anschauen und darauf warten, bis all seine tr\"{u}ben und jetzt so sehnsuchtsvollen Gedanken in meinen st\"{u}rmischen Liebkosungen traumverloren versinken k\"{o}nnten!
Wann, wann endlich?

\clearpage
