\def\day{21. Dezember 1943.}
\mktitle

Ich h\"{a}tte aufschreiben k\"{o}nnen, wie wir den zweiten Adventssonntag begingen.
Aber ich wei{\ss} nicht, aus welchem Grunde man unaufgefordert den einen mitbrachte.
Als er da war, mu{\ss}te ich gute Mine zum b\"{o}sen Spiel machen, jedoch meine Stimmung war dahin.
Gelten denn die einfachsten Gesetze anst\"{a}ndigen Handels nicht mehr?
Ich w\"{u}nschte mir ja nur eine bescheidene Lage, die es mir erlaubte, all dieses armselige, elende Pack abtun zu k\"{o}nnen.
Dum bin ich ja, da{\ss} ich immer noch teile mit anderen.
Die bekommen nichts, haben nichts und "fressen" gedankenlos hinein, was mit Sorgfalt und Liebe f\"{u}r mich gebacken und verpackt worden ist.
Henry W. machte mir so richtig Freude allein an jenem Abend.
Leider mu{\ss}te ich all die andern guten Sachen aus einfachem Hunger zwischendurch essen.
Ein Weihnachtspacket kam von den Eltern.
Es war lieb verpackt und ein Geschenk darin ausgew\"{a}hlt, das mich sonnst herzlich erfreut h\"{a}tte.
Hier schnitt es mir das Herz ab.
- Die Arbeitswoche ist beendet.
Jetzt sind Schulferien.
Und wieder einmal soll Weihnachten kommen.
Es ist aber keins mehr f\"{u}r mich.
Ich kann mich nicht selbst bel\"{u}gen.
Wer hier noch ein Gef\"{u}hl f\"{u}r eine Weihnachtsfeier aufbringen kann, ist nach meiner Ansicht auch unter Wilden oder Negern zu Hause.
- Geld wird es auch kaum geben, eine "Weihnachtsfreude" mehr.
Die Sorge um die Angeh\"{o}rigen und Lieben in der Heimat wird immer gr\"{o}{\ss}er.
Es gibt nur eine L\"{o}sung f\"{u}r all die Probleme hier: Heimfahrt ins Reich nach einem siegreichen Ende.

\clearpage
