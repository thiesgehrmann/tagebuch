\def\day{24. Dezember 1943.}
\mktitle

Auch dieser Tag ist wieder gekommen.
Ich hatte meine Briefe schreiben wollen.
Es ist bis jetzt beim Wollen geblieben.
Das Abendbrot haben wir hinter uns.
Seit Wochen wieder mal eine Mahlzeit, nach der man ges\"{a}ttigt war.
Das w\"{a}re ja alles nicht so schlimm, wenn man nicht so viele Gedanken h\"{a}tte.
Ich will aber keinen davon niederschreiben, keinen einzigen.
Denn sie sind es nicht wert.
Sie fallen mit der Gefangenschaft.
Und wann endet diese?
- Was haben sie heute daheim gemacht?
Ich bin in Gedanken alle letzten Heiligen Abende durchgegangen, alle hatten Freude gebracht, solange ich noch daheim war.
Auch hatte ich dann in meiner Dienstzeit das Gl\"{u}ck, zum Weihnachtsfest zu Haus zu sein.
Dir ersten Kriegsweihnachten kamen meine Eltern nach Flensburg und diese Tagen{\color{red} [sic] brachten ja f\"{u}r uns so richtig das er[s] }te junge Gl\"{u}ck.
Nie werd ich es vergessen.
Und im n\"{a}chsten Jahr noch einmal Galgenfrist mit Verl\"{a}ngerung.
Auf Umwegen kam ich heim zu ihr.
N\"{a}chstes Fr\"{u}hjahr in Gefangenschaft, Weihnachten eingepfercht auf der Verschleppungsfahrt nach Kanada.
Aber das konnte ein Soldat ab.
Wenn man auch bei Tag und Nacht auf den Sprung zum letzten gro{\ss}en Schritt im Leben bereit seit mu{\ss}te, aber war das etwas Neues f\"{u}r einen Soldaten?
Vielleicht gar Untragbares?
Das kam erst mit den n\"{a}chsten zwei Jahren und steht heute immer noch wachsend, un\"{u}berwindlich vor mir: Verdammt zu sein zur Unt\"{a}tigkeit, darbend zusehen zu m\"{u}ssen.
Und dabei sehen zu m\"{u}ssen wie es von Jahr zu Jahr sich \"{a}ndert, die Menschen immer mehr so werden, die sie wirklich sind.
Und kein Ende - keine M\"{o}glichkeit, frei davon zu sein!

\clearpage
