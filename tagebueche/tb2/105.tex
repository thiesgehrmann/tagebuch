\def\day{25. Dezember 1943.}
\mktitle

Der Speisezettel seit gestern abend ist wirklich gewaltig, wenn man bedenkt, da{\ss} es f\"{u}r Kriegsgefangene ist.
Jedoch mu{\ss} man sich stets sagen, da{\ss} den wesentlichsten Anteil unsere eigenen Leute mit ihrem Flei{\ss} in der K\"{u}che daran haben.
Ich will die Mahlzeiten beschreiben: Gestern abend: Kartoffelsalat in Mayonaise mit W\"{u}rstchen.
Heute fr\"{u}h: Kaffee, Wei{\ss}brot mit Butter u. Konfit\"{u}re, 1 kl. Napfkuchen f\"{u}r 6 Mann; mittags: Puten mit Kartoffeln u. Rotkohl, Suppe, Schokoladenpudding; abends: Ein gro{\ss}er Teller Aufschnitt (jawohl!) f\"{u}r 3 Mann anstatt der sonst \"{u}blichen Zulagen.
Und gestern nach der gemeinsamen Feier am Heiligen Abend im Speisesaal erhielt jeder, der hin\"{u}bergegangen war - und sowas wei{\ss} man -, eine bunte T\"{u}te mit einem kleinen Stollen und Geb\"{a}ck mit S\"{u}{\ss}igkeiten und Obst.
Au{\ss}erdem gab es eine menge Zigaretten.
So kann es uns im Grunde wenig ausmachen, wenn uns der Tommz unser Geld nicht ausbezahlt. -
Zum Heiligen Abend selbst gab es einen guten Kaffee.
Ich habe mit Hein Finger in unserer Ecke zusammengesessen.
Auf dem tisch, den ich mit dem Holzgeschirr von den Eltern gedeckt hatte, und mit Weihnachtspapier und Tannengr\"{u}n geschm\"{u}ckt hatte, stellten wir unsere Bilder: von den Eltern, von seiner Frau und das gro{\ss}e von ihr.
Dann haben wir geplaudert.
Das elektrische Licht war aus auf der Stube.
Wir hatten auch Kerzen diesmal.
Es war anheimelnd und sch\"{o}n zu sehen, wie das Kerzenlicht den den armseligen Holzbaracken das H\"{a}{\ss}liche, Na{\color{red} [c] }kte zu nehmen vermochte.
Es war ein Heilig Abend noch in der Fremde, aber der Heimat n\"{a}her.

\clearpage
