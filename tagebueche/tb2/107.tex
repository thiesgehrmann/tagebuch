\subsection{31. Dezember 1943.}
%\mktitle

Meine hilde, Du innig geliebtes Wesen, Du, ich m\"{o}chte Dir schreiben, wenn ich k\"{o}nnte und es auch d\"{u}rfte, wie ich wollte.
Es ist der Altjahresabend.
Ich habe ihn sch\"{o}n verbracht, so sch\"{o}n wie es mir irgend m\"{o}glich ist in der Gefangenschaft.
Drau{\ss}en in Schnee und Sonne auf meinen Brettern.
Lange habe ich nicht mehr solche Freude gehabt, wie bei diesen Stunden.
Sie bringen so ein Frohsein mit sich, da{\ss} ich es fr\"{u}her kaum f\"{u}r m\"{o}glich gehalten h\"{a}tte.
Zum Abend sollen wir noch sechs Flaschen Bier bekommen; R\"{o}stbrot und K\"{a}se zum pr\"{a}peln gibt es dazu.
Aber ich werde uns unsere Ecke schon ein bi{\ss}chen gem\"{u}tlich machen.
Wem k\"{o}nnte denn ein Sylvesterabend f\"{u}r das kommende Jahr mehr Hoffnung versprechen als dem Kriegsgefangenen!
Wenn es eben noch nicht im n\"{a}chsten Jahr sein soll, so will ich es doch w\"{u}nschend gern glauben.
Es k\"{o}nnte ja sein.
Und welch ein Fr\"{u}hling wird dann f\"{u}r uns beide beginnen.
Sie, die so lange mein Peterlein hat sein m\"{u}ssen, soll es mir dann nur noch einmal ein einziges Mal sein und dann, ja dann w\"{u}nschte ich mir doch, da{\ss} sie ein so kluges Frauchen sein wird, \st{A} vor ihrem eigenen mann eben immer, immer noch ein\st{e} neues Reizvolles zu verbergen \st{zu} wissen, was sie ihm stets wieder begehrens- und eroberenswert zu sein hei{\ss}t.
Ja, das w\"{u}nsch' ich mir jatzt f\"{u}r alle Zukunft von ihr!-
Aber heut mu{\ss} ich noch allein sein und da werde ich mir als Sylvesterbesch\"{a}ftigung die alten Esquirebl\"{a}tter durchsehen.
O, wenn sie's w\"{u}sste.

\clearpage
