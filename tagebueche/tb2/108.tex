\def\day{4. Januar 1944.}
\mktitle

Meine 1. Eintragung im neuen Jahr.
Die Aussichten sind verheerend f\"{u}r 1944.
Das Essen wird immer weniger; an satt werden ist seit den Feiertagen gar nicht mehr zu denken.
Was au{\ss}erdem das neue Jahr so mit sich brachte ist ein pr\"{a}chtig aufbl\"{u}hender Wahnsinn im vierten Gefangenenschaftsjahr durchschnittlich.
Menschen kann man hier kennen lernen, und eins mu{\ss} man vor allem wieder lernen: sich einen Dreck um andere k\"{u}mmern.
Gaunereien, Gemeinheiten, Niedertr\"{a}chtigkeiten, das scheint die allgemeine Losung.
Alles was anst\"{a}ndige Gesinnung und echte Kameradschafteshandlungen sind, werden durch den Schmutz geholt.
Das ganze nennt sich eine Gemeinschaft von Unteroffizieren, von denen der eine Teil so borniert ist, sich in h\"{o}chster Weise von dem anderen abzuschliessen.
Ich bedauere sie; wie wird es denen schwer fallen, sich in Deutschland wieder zurechtzufinden.
Wahnsinn sage ich dazu!
Und denke an die Schilderung der Gefangenschaft aus dem Buch "Volk ohne Raum".
- Es ist sehr interessant, festzustellen, welchen geistigen Querschnitt das \"{u}brige Verh\"{a}ltnis der Dienstgrade abgibt und wie unterlegen der langdienende Soldat, also der aktive, gerade dem Reservisten ist.
Doch Schlu{\ss}.
Es ist hinnreichend bekannt.
- Hurra, wir d\"{u}rfen Geld nach Hause senden!
Ein himmelsschreiender Bl\"{o}dsinn, den man grade uns noch antun mu{\ss}.
- Am Neujahrtag habe ich mit Jcke{\color{red} [sic] } geplaudert.
Er las mir ein paar seiner Briefe \st{dur}vor.{\color{red} [sic] } und bat mich um meine Stellung dazu.
Es war eine recht erquickende Stunde.
Ich gab ihm dem brief von Edeltraut zu lesen.

\clearpage
