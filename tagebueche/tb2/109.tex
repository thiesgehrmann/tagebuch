\subsection{8. Januar 1944}
%\mktitle

Mein Lieber Hans!
Das geht mit Dir nicht mehr so weiter.
Du verlierst Dich in tr\"{a}umerischen Gedanken, st\"{a}ndig hast Du etwas auszusetzen, laufend haderst Du mit Deinem Schicksal.
Noch dazu klagst Du in Deinen Briefen an sie.
Ich will Dir mal ganz ehrlich meine Meinung sagen.
Am Montag f\"{a}ngt ja die Schule wieder an.
Gut, Du hast Dich selbst entschlossen, in der Naturwissenschaftlichen Klasse zu verbleiben.
(Dein pers\"{o}nliches Unbehagen und die Furcht, den guten Platz in der Klasse zu verlieren, kann ich verstehen!)
Also wirst Du regelm\"{a}{\ss}ig u. flei{\ss}ig das Latein weiterbetreiben f\"{u}r Dich.
Aber gleich am Montag wird rechtzeitig aufgestanden, damit Du am Morgen frisch und munter in den Tag gehst.
Den Morgengebet sei: "Ich will freudig arbeiten und viel lernen, meine Arbeit is ja ihr und unseren Kindern gewidmet.
Ich will ja doch sehen, ob bei diesen Anlagen nicht eine t\"{u}chtige Leistung zu erreichen ist, die den Inspektor zu etwas berechtigt."
Am Nachmittag werden mir dann keine M\"{u}digkeitsscheinungen gezeigt.
Du kannst 2 mal in der Woche raus zum Skilaufen, das gen\"{u}gt.
Kommt's Dich trotzdem an, dann raus und 2 Runden in Eis und Schnee gedreht.
Danach wieder an die Arbeit!
Und nicht mehr denken "ach es geht schon"!
Willst Du dann noch am Abend Deinen tr\"{u}ben Gedanken nachh\"{a}ngen, bei{\ss} die Z\"{a}hne zusammen!
Denke, eine Stunde in der Armen der Geliebten l\"{a}{\ss}t Dich alle Bitterkeit vergessen.
Und sie will ja nichts weiter \st{D}als dir Frau und Gattin und Mutter Deiner Kinder sein.
Wer hat denn noch so rein und treu das Liebste im Leben daheim?
- Dein besseres Ich.

\clearpage
