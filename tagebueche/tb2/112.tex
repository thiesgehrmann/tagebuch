\subsection{2. Februar 1944.}
%\mktitle

So, Du mein herziges Lieb, jetzt habe ich wieder einmal etwas H\"{u}bsches vor Dich hergestellt, und ich mu{\ss} sagen, da{\ss} mir gerade solche kleinen Arbeiten eine tiefe innere Befriedigung und Ruhe zu geben verm\"{o}gen.
Was ist es?
Das Kapuzchen{\color{red} [sic] } habe ich doch bekommen.
Es ist unzweifelhaft eines der liebenswertesten und entz\"{u}ckendsten Bilder, die je von Deinem lieblichen Gesicht auf einen Filstreifen gebannt wurden (wobei ich mir schmeichele, es ja selbst aufgenommen zu haben).
Dieses Kapuzchenk\"{o}pfchen{\color{red} [sic] } kam nun vergr\"{o}ssert fein ausgeschnitten als Briefkopf zu mir.
Was sollte ich mit {\color{red} [ihm] } tun, wenn es mir stets zur Freude mich nun immer \st{an} mit seinem heiteren L\"{a}cheln anschauen sollte?
Ich habe mir einen Hartholzklotz und zwei Glasscheiben besorgt und einen feinen Bildst\"{a}nder daraus gebaut.
Nichts weiter als einen M\"{a}chenkopf, dessen Locken unter einem Kapuzchen munter hervorgucken, ist auf dem Glas zu sehen.
Vor einem dunklen Hintergrund macht sich der St\"{a}nder tadellos.
Du schaust mich jetzt immer so fr\"{o}hlich an, Du Lieb.
Ich bin sehr froh grad \"{u}ber dieses kleine Bild von dir, weil es uns beiden ja eine Erinnerung ist.
- Wenn ich nun solche Gedanken hege, dann m\"{o}chte ich dir schreiben, aber ehe ich dazu komme, schiebt sich der l\"{a}hmende graue Alltag dazwischen und aus ist es, wenn ich nur zur Feder greife.
Das f\"{u}hlst Du nat\"{u}rlich dann auch.
Wie lange noch?

\clearpage
