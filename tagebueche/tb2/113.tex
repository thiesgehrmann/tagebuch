\def\day{6. Februar 1944}
\mktitle

Soeben habe ich die Vorlage zur Ferntrauung ausgef\"{u}llt und unterscrieben.
Als ich am Freitag, vor zwei Tagen also, 15 Briefe erhielt, brachte mir eine von ihr eine ganz sch\"{u}chterne Anfrage, wie ich \"{u}ber eine Ferntrauung denken w\"{u}rde.
Mutti rate ihr jetzt dazu, weil wir dann leichter eine Wohnung haben k\"{o}nnten, bzw. im gegebenen Falle die Wohnung, in der sie jetzt ein Zimmer genommen hat, \"{u}bernehmen k\"{o}nnten.
Mein Entschlu{\ss} war schnell gefa{\ss}t, aber nicht \"{u}bereilt.
In der seelischen u. geistigen Beziehung zwischen uns beiden kann die reine Amtshandlung der Trauung keine \"{A}nderung herbeif\"{u}hren.
Hochzeit machen wir doch erst bei meiner Heimkehr.
Und die kann uns viel froher sehen, wenn uns ein eigenes Heim dann gleich aufnimmt.
Meine Gedanken dar\"{u}ber hatte ich ihr l\"{a}ngst schon mitgeteilt.
Sie ist nun 2 Jahre \"{a}lter geworden inzwischen, Paul will sich verloben und bald heiraten, die aussicht auf eine Wohnung lockt.
- Ich glaube, all diese Dinge haben in ihr leis den Gedanken, bei meiner R\"{u}ckkehr verheiratet zu sein, geweckt und ihn ihr vertrauter gemacht.
Ihr Brief war nur eine zaghafte Frage.
Meine Antwort ist ein fester Entschlu{\ss}.
Wenn ich mir vorstelle, bei meiner Heimkehr die kurzen Tage bis zu unserer Hochzeit mit meiner mir schon angetrauten Frau noch als unsere Brautzeit zu verleben, so finde ich den Gedanken schon als k\"{o}stlichen Spa{\ss}.
- Das Gerede von Austausch l\"{a}{\ss}t mich v\"{o}llig k\"{u}hl dabei.
In diesem Krieg wird alles durch harte Tatsachen entschieden.

\clearpage
