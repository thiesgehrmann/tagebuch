\def\day{7. 2. 1944}
\mktitle

Die Ferntrauung ist eingereicht.
Von nun an ein Vierteljahr kann es dauern, d.h. im Mai k\"{o}nnte die Trauung volzogen sein, und ich k\"{o}nnte dann fr\"{u}hestens Ende Juli, Anfang August eine Nachricht dar\"{u}ber haben.
Ich werde mein Herz verschlie{\ss}en m\"{u}ssen.
Schweigen mu{\ss} es zu allen, was dann kommt, ganz still u. fein mu{\ss} es alles f\"{u}r such behalten, wor\"{u}ber es vor Gl\"{u}ck und Freude jauchzen m\"{o}chte.
- Dem heutigen Tag habe ich nun doch noch eine stille feine Stunde abjagen k\"{o}nnen, jetzt am Nachmittag.
Alle Briefe sind geschrieben, morgen fr\"{u}h gehen sie fort; am Abend will ich sie noch eintragen.
Ich w\"{u}nschte nur noch, es gel\"{a}nge, ihr ihren Wunsch der M\"{a}dchenjahre an dem Trauungstag als \"{U}beraschung von mir zu erf\"{u}llen.
Ein Zungenring{\color{red} [sic] } aus 14 kar\"{a}tigem Gold mit einem Brillianten, linsenrein, blauwei{\ss} in Chattongfassung soll es sein.
Den soll ihr Vater als meinen Gru{\ss} an den Finger stecken.
(Mag er ruhig 300,-M kosten, mir ist es das wert.)
Und dann will ich abwarten, ob sie die Wohnung erh\"{a}lt sp\"{a}ter und was mir mein Vorsteher aus M\"{u}rwik zu schreiben hat.-
Die Frage der Konfessionen ist noch zu erledigen.
Die Trauung kann nur geschlossen werden, wenn beide Partner sich von der alten Konfession gel\"{o}st haben.
Ich habe f\"{u}r beide in der Urkunde "gottgl\"{a}ubig" unter religi\"{o}sen Bekenntnis vermerkt.
Das ist mir die enzige aber auch unumst\"{o}{\ss}liche Bedingung.
Ich zweifle nicht.-

\clearpage
