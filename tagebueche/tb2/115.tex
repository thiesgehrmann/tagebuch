\def\day{15. Februar 1944.}
\mktitle

Eben bin ich wieder aus dem Krankenrevier auf die h\"{a}{\ss}liche Wohnbaracke zur\"{u}ckgekommen.
Das Gef\"{u}hl dabei in mir ist unbeschreiblich.
Heute mittag noch in der sauberen Krankenstube - dann pl\"{o}tzlich in der verru{\ss}ten H\"{u}tte.
2 Stunden habe ich an meinen Sachen staubgewischt.
Buchst\"{a}blich fingerhoch lag der Ru{\ss} u. Schmutz der \"{O}fen auf jeder Fl\"{a}che.
Und mein Nachbar, mit dem ich meine Ecke teile, der wischt ja bei sich selbst nicht einmal ein bi{\ss}chen Staub.
- Ja, ich hatte mit einer Halsentz\"{u}ndung bei ziemlich hohem Fieber in das Revier gehen m\"{u}ssen.
Dort war es aber ruhig, still, sauber, man konnte sich erholen.
Ich tat es auch gr\"{u}ndlich.
Zwei hei{\ss}e Wannenb\"{a}der habe ich genommen, die taten mir gut.
Heute mittag Heinz Wolf den kleinen Gr\"{u}tzbeutel genau \"{u}ber der Stirn herausgenommen.
Ich bin sehr froh, das ich diesen kleinen Sch\"{o}nheitsfehler los bin.
Forerst habe ich ihn gegen einen anderen, jedoch vor\"{u}bergehenden eintegauscht: Der Arzt mu{\ss}te mir nat\"{u}rlich dazu ein sch\"{o}nes B\"{u}schel der sch\"{o}nen langen Stirnhaare herausschneiden.
Es ist aber gut, das sie es nicht sieht.
- Der Dolmetscher unseres Lagers hatte mich aufgesucht, um mir mitzuteilen, das mich zur\"{u}ckgekommene Briefe Ottawa erwarten.
Sind es die, die ich im Februar schon schrieb?
- Die Umst\"{a}nde unter denen ich hier leben mu{\ss}, sind furchtbar.
Der Kleinkram dr\"{u}ckt mich furchtbar.
Wenn ich da an die Trauungsurkunde denke, schwindelt es mir.
Kann eine Frau denn zu solch einem Mann aufblicken?

\clearpage
