\def\day{17. Februar 1944.}
\mktitle

Gottlob, es waren nicht meine Februarbriefe, die zur\"{u}ckgekommen waren, sondern nur ein Brief an sie aus dem Januar.
Ich hatte Allgemeines u. Liebes hineingeschrieben, des weiteren etwas \"{u}ber Geld u. Essen.
Das war dan zu viel.
Folglich mu{\ss}te ich hinaus zum Rapport beim Kommandeur.
Seltsamerweise lie{\ss} man mich ganz allein aus den Lager gehen.
Das macht mir nun ja nichts aus.
Ich habe mich mit unserem "Stafi" angeregt unterhalten auf den Hin-u. R\"{u}ckwege, doch als mich der Kommandeur befragte, ob ich Englisch spr\"{a}che, antwortete ich ganz einfach: "No, sir{\color{red} [sic] }" und "Please will you tell \st{to} the interpreter", h\"{a}tte ich beinahe noch unvorsichtigerweise zugef\"{u}gt.
Das habe ich bei jeder Vernehmung so gehalten: Wenn man von mir etwas wissen will, so k\"{o}nnen sie sich gef\"{a}lligst meiner Muttersprache bedienen.
Denn wenn ich auch sehr gut englisch verstehen kann, so werde ich mich nie des Vorteils begeben, mich in den Wendungen und mir wohl gel\"{a}ufigen Ausdr\"{u}cken u. Sinnbedeutungen meiner Muttersprache verst\"{a}ndlich zu machen.-
Ich hoffe ja nur, das meine Februarbriefe ungehindert durchgehen werden.
Am 1. M\"{a}rz kann ich ja dann gleich auf die Notwendigkeit der von mir Angewanten kleinen Schrift f\"{u}r den besonderen Zweck eingehen.
- Ist ein kleines Erlebnis und nicht uninterressant{\color{red} [sic] }, so ein Postrapport beim "Herren Kommandeur".

\clearpage
