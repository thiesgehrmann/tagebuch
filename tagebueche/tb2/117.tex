\def\day{20. Februar 1944.}
\mktitle

Sonntag war heute.
Wieder ist einer der Gefangenschaftstage hingegangen.
Ich m\"{u}{\ss}te so viel arbeiten, lernen, aber es wird mir alles leid, wenn ich auch nur einen Gedanken von dem Thema abweichen lasse.
Die n\"{o}tigsten Schulaufgaben habe ich ja erledigt, doch ist da noch Mathematik zu machen, 2 Aufs\"{a}tze zu schreiben: B\"{u}cher k\"{o}nnte ich lesen, viele u. gute.
Aber laufend ist um mich herum Geschw\"{a}tz, Gerede, Krach, Lachen, L\"{a}rmen.
Dann der ungeheure Schmutz auf den Baracken.
Wandern meine Gedanken noch weiter, so gehen sie zu meinen Briefen und da ist es schon ganz aus.
Was h\"{a}tte ich ihr allein zu schreiben!
Ferntrauung wollen wir beide machen.
Kinen Menschen hat man, dem man es anvertrauen k\"{o}nnte, dem ich voll innerer Freude u. Dankbarkeit davon sprechen k\"{o}nnte.
Nein, drei Jahre waren mir Lehrzeit genug.
Man erz\"{a}hlt sich, man traut u. glaubt, bis der Tag kommt, der aus meist l\"{a}cherlichen oder h\"{a}{\ss}lichen Gr\"{u}nden die ganze Niedertr\"{a}chtigket des Mitmenschen offenbart, die oftmals - und mich schmerzt es doppelt, die zu erkennen, - selbst nur aus der eigenen  Not des Herzens heraufbeschworen wurde.
Es ist schon so, unser Leben zeigt \"{a}u{\ss}erlicj einige Bequemlichkeiten, es ist aber in Wahrheit schlimmer zu f\"{u}hren als das eines Zuchth\"{a}uslers.
Es ist so wahnsinnig schwehr, sich selbst zwingen zu m\"{u}ssen und dennoch bei gutem Willensich selbst so oft straucheln zu sehen.

\clearpage
