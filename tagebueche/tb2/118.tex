\subsection{27. II. 1944.}
%\mktitle

Ja, mein Lieb, Dein Geburtstag ist \st{zu} vor\"{u}bergegangen.
Ich habe ihn nicht vergessen, nein, gar viel habe ich Deiner gedacht neben meiner Arbeit.
- Da ist riesig viel zu tun.
Doch manchmal m\"{o}chte ich alles in die Ecke feuern (und das ist die richtige Einstellung wohl.)
So manchen Gedanken hege ich.
Ein Brief ist da von ihr gekommen.
Der erz\"{a}hlt von Helga.
Helga, ein klingender Name.
Wie die Tr\"{a}gerin aussieht?
Ach, ich m\"{o}chte es glauben, das sie ihren Namen recht tr\"{a}gt und eine Helga verk\"{o}rpert.
Was sagte ich damals zu der Edeltraut, um sie aufzurichten?
Deutschland braucht starke Menschen!
- Und schon verliere ich mich wieder in Gedanken.
- Der professor hat uns den Plan der letzten Schulzeit angesagt.
Eine Generalprobe zu Ostern, dann im September Pr\"{u}fung.
Ich habe mich entschlossen, die Lateinpr\"{u}fung mitzumachen.
Bin doch mal gespannt, inwieweit ich etwas schaffe!-
Von Vater kam ein Brief, der mir den letzten Kontoauszug best\"{a}tigte.
Also ist es wirklich so, da{\ss} ein Mann, der eine Familie gr\"{u}nden soll, mit 28 Jahren, wenn er gezwungenerma{\ss}en noch ledig ist und aus einem Gef\"{u}hl ethische Werte keine Ferntrauung macht, immerhin schon ein Geld verdienen kann, das jede lumpige Firma zur Zeit einem jungen M\"{a}dchen f\"{u}r beliebige schematische Dienste im Monat bezahlt.
Ich bin ersch\"{u}ttert dar\"{u}ber.
Ob ich wirklich noch einmal sagen darf, es hat doch gelohnt?
Dann doch lieber einen gutgezielten Treffer, und alles w\"{a}re vor\"{u}bergewesen.

\clearpage
