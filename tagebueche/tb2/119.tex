\subsection{6. III. 1944.}
%\mktitle

Ein Montagabend.
Ich habe keine Lust, auch nur das Geringste zu arbeiten.
Gestern am Sonntag habe ich schon den ganzen Tag gefeiert.
Eigentlich wollte ich ab  1. M\"{a}rz meine Post schreiben.
Aber wenn ich jetzt meine Lage \"{u}berdenke, zumal nachdem ich im Februar doch innerlich ziemlich froh u. beherzt jenen Schritt getan habe, so ist mir doch nach Vaters Brief mit der Mitteilung \"{u}ber mein Gehalt fast die ganze Lust genommen.
Spa{\ss} macht es mir ja auf keinen Fall, zu allem abseits stehen zu m\"{u}ssen, und der Gedanek, da{\ss} ein M\"{a}dchen f\"{u}r eine rein schematische Arbeit genau soviel und mehr verdienen kann, als ein junger Mann sich erwerben und erstreben durfte, bringt mich fast jedesmal zur Wut.
Was sind das f\"{u}r Zust\"{a}nde!
Ich k\"{o}nnte mir zur Zeit wei{\ss} Gott nicht vorstellen, wie es w\"{u}rde, wenn ich jetzt nach Hause k\"{a}me.
Denn mit den Jahren geht alle Freundschaft und Verbindung hin.
Gewi{\ss}, die Ferntrauung ist eingereicht.
Ich kenne mich gut: ein Wort des Bedenkens und alles bricht f\"{u}r mich zusammen.
Es ahnt gewi{\ss} keiner, aber ich bin weit genug, um mich mit einem Schlage von alles l\"{o}sen zu k\"{o}nnen, und nur darum, weil mir das Schicksahl langsam alles aus den H\"{a}nden wand.
Um ein Bettler vor dem Leben zu werden, will ich nicht gelebt haben.
H\"{a}tte mich doch eine Granate zerrissen, w\"{a}re mir dies alles erspart geblieben.

\clearpage
