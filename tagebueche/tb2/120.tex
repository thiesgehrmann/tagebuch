\def\day{8. M\"{a}rz 1944.}
\mktitle

Das gro{\ss}e R\"{a}tsel der letzten anderthalb Jahre ist nun endlich gel\"{o}st: am 15. M\"{a}rz wird das lager Monteith 23/i nach Madison Head {\color{red} [sic, Medicine Hat] } verlegt.
- Die Schule wird also wieder unterbrochen.
Heute schon ist der Unterricht abgesagt.
Dann beginnt das gro{\ss}e Packen.
Es ist nur immer wieder erstaunlich, was sich im laufe der Zeit angesammelt hat.
Ich werde mir ein genaues Verzeichnis meiner B\"{u}cher und der Wertsachen anlegen, das ist immerhin ganz interessant f\"{u}r sp\"{a}tere Zeiten einmal.
Meine Skier werde ich noch an das Seemannslager verkaufen.
10,-{\$} ist ein runder Preis.
Wenn wir dann erst im Zug sitzen haben wir f\"{u}r 4 Tage midnestens Ruhe.
So lange wird die Reise bestimmt Dauern{\color{red} [sic] }.
Was dann kommt?
Man mu{\ss} es ja an sich herantreten lassen.-
Ich will mich gleich hinsetzen, um meine Briefe und Karten zu schreiben.
Es wird diesmal rasch gehen, denn aus dieser Unruhenstimmung heraus wird es mir nicht gelingen, sch\"{o}ne Briefe zu schreiben.-
Es ist ein Leben wie Hund, sagt man in der Soldatensprache.
Kriegsgefangene sind wei{\ss} Gott es Herrgotts \"{a}rmste Sch\"{a}fchen auf seiner Erde.
Und das kommt nun jetzt alles zu der Zeit, wo ich die Ferntrauungsurkunge abgeschickt habe.
Wenn ich daran denke - nein, unter den hiesigen Umst\"{a}nden kann ich gar nicht daran denken.
Ich hatte mir mein Leben bei gott ein wenig anders vorgestellt.
Nichts schaffen d\"{u}rfen!!

\clearpage
