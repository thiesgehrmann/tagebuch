\subsection{Medicine Hat, 1. Mai 1944.}
%\mktitle

Ich wollte Dir schreiben, du geliebtes M\"{a}del.
Erst hatte ich noch ein paar Arbeiten f\"{u}r die Schule erledigt, dann wollte ich Dir einen Brief senden.
Denn heute ist ja unser Tag.
ich h\"{a}tte Gedanken, Ideen, Dir die sch\"{o}nsten Briefe zu schreiben.
Abe ein anderes brauche ich hier viel notwendiger dazu.
Als ich eben an meinem tisch, den ich mit viel List aus Monteith habe mitgehen lassen, Sa{\ss}, dicht vor dem Fenste, kamen mir die sonderbarsten Gedanken.
Da sa{\ss} ich nun und beguckte mir meine Schularbeiten.
Drau{\ss}en tobte ein Sandsturm in dieser gottverlassenen Gegend.
Fortw\"{a}hrend drang der feine  Sand durch alle Fensterritzen; Die Feder knirschte beim Schreiben, zwischen Z\"{a}hnen mahlte der Sand und wo ich eben gewishct hatte, lag nach 5 Minuten eine dichte mehlige Schicht.
Hier mu{\ss} man leben!
Ich dar nicht an dieses Lager denken.
{\color{red} [8 Worte sind hier geschw\"{a}rzt] }
bereitwilligst erf\"{u}llt und sich von dem Vorhaltungen machen l\"{a}{\ss}e{\color{red} [sic] }.
Nein, meine Hilde, ich h\"{a}tte Dir schreiben wollen, brenned gern, aber die Zeit geh\"{o}rt kaum mir, immer sind zu viele und zuvieles H\"{a}{\ss}liches und Gemeines darum herum.
Nimm mein Gedenken.
Meine Erinnerung dieses Tages ist mit zu rein u{\color{red} [.] } hoch, als da{\ss} ich sie in dieser Umgebung preisgeben k\"{o}nnte.
Verzeih es.

\clearpage
