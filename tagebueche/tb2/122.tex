\subsection{5. Mai 1944.}
%\mktitle

2 Stunden Geschichtsvortrag sind wieder herum.
ich f\"{u}hlte mich mitten in einem f\"{u}r Gefangenschaftsverh\"{a}ltnisse riesigen - und noch dazu f\"{u}r das neue Lager - geistigen Schaffen.
Es macht mir unb\"{a}ndige Freude und ein Urteil ist auf alle F\"{a}lle erreicht: Der geistige Niederbruch der langen jahre ist bestimmt \"{u}berwunden.
Und das ziel lockt und soll es sp\"{a}ter lohnen.-
Was so alles in einem Tage isch ereignen kann.
Kommt da gestern ein Brief von kurt Schneega{\ss}.
Ja, der alte junge wei{\ss} ja gar nciht, was er mir antut, wenn er so begeistert von seinem kleinen Frauchen schreibt.-
Am Abend ist dann Musik von Schallplatten in der H\"{u}tte.
Es ist wohl st\"{o}rend, wenn da ein Kamerad bezecht vom Bierabend kehrt.
Doch ist das nun Humor?
Der Mann verschluckt sich am eigenen Bier, da{\ss} er bald dran ers\"{a}uft und erstickt.
Nur schnelles Eingreifen mit Wiederbelebungsversuchen h\"{a}lt ihn am Leben.
Er wird ins Revier geschafft.
Erholt sich dort v\"{o}llig und kommt quietschfidel u. ahnungslos heute mittag aus dem Schlunz.
Das erste, was er unternimmt, ist, da{\ss} er treuherzig seinen Retter fragt: {\color{red} ["] }Na, H\"{a}schen, wie geht's Dir denn?"-
Kanada, das land der Weite.
Brennt da irgendwo die Pr\"{a}rie 150 Meilen fon hier, da{\ss} einen ganzen Tag lang Brandgeruch und Qualm in der Luft stehen durch den man keine 100m sehen kann und die Mittagssone wie ein Feuerball dunkelrot nur hindurchglimmt.
Was macht das hier aus?-

\clearpage
