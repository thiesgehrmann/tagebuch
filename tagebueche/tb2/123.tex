\subsection{14. Mai 1944.}
%\mktitle
Wir sind fast 2½ Jahre in Monteith gewesen.
Was uns unser Lagerf\"{u}hrer gewesen ist, haben die letzten sp\"{a}testens bei der Ankunft in diesem Lager erkannt.
Noch eins ist in Monteith gewesen, was mir f\"{u}r die Gefangenschaft von unsch\"{a}tzbarem Wert gewesen ist: Man kannte sich, man wu{\ss}te um seine gegenseitigen Schw\"{a}chen und kleinen Fehler.
Dar\"{u}ber hinaus ist man aber, ein jeder fast in einer geregelten zielstrebigen geistigen Besch\"{a}ftigung f\"{u}r die Zeit nach dem kriege, mitten in de aus Raumnot geborenen unruhe doch wieder zu einem stillschweigenden Einverst\"{a}ndnis und zu einer anerkennung der pers\"{o}nlichkeit gekommen, das dem einzelnen die notwendigen Rechte zur Selbsterhaltung zugestand.
Das war ein Gewinn.-
Wie anders ist es hier!
Wie selten unbek\"{u}mmert r\"{u}cksichtslos "latscht" hier die gro{\ss}e menge \"{u}ber die einfachsten Gestze aus dem Zusammenleben einer Gemeinschaft hinweg.
Sie lernen nichts oder planlos ein bi{\ss}chen, sie greifen gedankenlos in andere Rechtsgebiete.
Der andere ist ja auch blo{\ss} so ein "armes Schwein", wie sie es sich so oft denken.
- Aber um himmels willen keine lehre von anderen annehmen.
nei, warum denn?
Es geht ja alles so seinen Gang. -
Die helle Wut kann einem{\color{red} [sic] } st\"{a}ndig packen, wenn man an diese Gesellschaft denkt und wie sie uns empfangen haben.
Ich habe nie an Austausch geglaubt und glaube es auch nicht.
Aber hier bete ich morgens mittags u. abends zum herrn, da{\ss} es werde.
Ich will hier hinaus.

\clearpage
