\subsection{16. Mai 1944.}
%\mktitle

Gestern ist der Brief angekommen.
Er ist vom 27. Februar 44, also zwei Tage nach ihrem Geburtstag, und enth\"{a}lt jene Frage: "K\"{o}nntest du mir vielleicht auch verzeihen? w\"{a}re Deine liebe auch so gro{\ss}?"
Der bewegende Grundgedanke \"{u}berrascht mich nicht.
Es ist und bleibt immer wieder eine Folge meiner eigenen Feigheit damals am 5. Juni 1941: Eine Kugel h\"{a}tte mir alles weitere Elend erspart.
Schicksalhaft ist dabei der Verlust meines letzten Briefes aus England.
Jetzt aber will und mu{\ss} ich leben.
Deine meine gr\"{o}{\ss}ere Liebe. die Deutschen M\"{a}nnern eine ethische pflicht ist, hei{\ss}t mir nach wie vor Deutschland.
- Was der Brief in mir leider ausgel\"{o}st hat, ist eine h\"{o}chst elende k\"{o}rperliche Verfassung, die in ihren l\"{a}hmenden Auswirkungen meinen Seelenschmerz weit \"{u}bertrifft.
Ob ich je darauf antworte und was ich in den wenigen Zeilen schreiben kann, wei{\ss} ich jetzt noch nicht.
Eins wei{\ss} jedoch: Ich werde um meine liebe k\"{a}mpfen, ja, das tue ich.
Aber ach, wie weit kann denn ein Mann in diesem Kampf nur gehen!
Ihm sind nicht die rein triebhaften Kr\"{a}fte eine Frau gegeben.
- Wenn ich aus ihren zaghaften und wenig\st{en}
mutigen Andeutungen etwas entnehmen kann, so ist es dies: Ihr ist ein mann begegnet, dessen Wesen nicht ohne bedeutenden Einflu{\ss} auf sie geblieben ist.
Sie bringt anscheinend auch nciht den Mut und die Kraft auf, ihm zu wehren.
M\"{o}glich, da{\ss} eine wissende Frau sehend unter dem physisch-psychischen Bann eines Mannes sthen kann.

\clearpage
