\subsection{{\color{red} [16. Mai 1944, fortgesetzt] }}
%\mktitle

Dieser mann - sie nennt ihn verehrend und achtend "einen Herren", das nimmt mich f\"{u}r ihn ein.
Er schickt ihr \st{Brot} dunkelrote Nelken, was sie ihm erlaubt.
Ihre zeit f\"{u}r die verheiratete Freundin scheint er in Anspruch zu nehmen.
\st{Das} Vielleicht ist er auch "die Bekannten", mit "denen" sie ihn Gl\"{u}cksburg zum kaffee war, vielleicht, ich wei{\ss} es ja nciht.
Sie gibt den drei Jahren unserer Trennung die Schuld.
Den Gedanken daran - ob sie die Gef\"{u}hle der gesunden Kraft eines jungen Weibes meint? Warum diese Umschreibungen? - habe sie gewaltsam unterdr\"{u}cken m\"{u}ssen.
Jetzt f\"{u}hlt sie die Kraft daf\"{u}r entschwinden.
Ist mir alles noch verst\"{a}ndlich.
In dem h\"{o}chst ungleichen Gewicht von Erinnerung der Vergangenheit und dem Erleben der Gegenwart unterl\"{a}uft ihr sogar schon die Vermischung, so da{\ss} sie sich statt der gut drei gleich f\"{u}nf Jahre einsam verlassen und um ein Jugendgl\"{u}ck betrogen f\"{u}hlt.
- Ich kann und will sie nicht richten.
Ich liebe sie und m\"{u}{\ss}te sie hassen sonst.
Aber warum diese Frage?
Hat sie alles vergessen?
Ich gab ihr doch einen Ring mit meinem Namen.
Ich ahne es wohl, skie hat ihn seit der Verlobung zu Weihnachten nicht mehr getragen.
F\"{u}hlte sie nun die Liebe dieses Mannes meine Liebe zu ihr besiegen, dann soll sie sich pr\"{u}fen, und wenn es so ist, mir meinen Ring zur\"{u}cksenden.
Dann wird er sie wohl heiraten und sie auch mit ihm gl\"{u}cklich werden.
Falsch w\"{a}re da jede piet\"{a}tvolle R\"{u}cksicht auf mich und gar noch meine Lage. ich k\"{o}nnte doch nie das Leben anders zwingen.

\clearpage
