\def\day{{\color{red} [16. Mai 1944, fortgesetzt] }}
\mktitle

Sollte sie jedoch glauben, aus rein physichen Gr\"{u}nden das Alleinsein nicht mehr aushalten zu k\"{o}nnen, dann, zum Himmel, soll sie sich doch nehmen, was sie denkt, da{\ss} es ihr diene.
Der Weg ist einfach, aber er ist gef\"{a}hrlich.
An ihm stehen hohes Frauentum und Spielereien einer hure um haaresbreite voneinander getrennt.
Und ist der mann ein Kerl, so nimmt er dies nicht an.
Denn nie d\"{u}rfte er erfahren, wer sie sei, und das Gesetz des Handelns solch eine liebesnacht d\"{u}rfte ihr nciht aus der Hand gleiten.
Welcher Mann mag das ertragen?
Welche Frau w\"{a}re dazu v\"{o}llig im Stande?
Ist es denkbar, da{\ss} eine Frau ihr Herz nicht mitsprechen lie{\ss}e, wenn ihr K\"{o}rper liebt?
Das kann wohl keine Frau au{\ss}er einer Hure, da{\ss} sie sich mit einem manne vereint gleich einer reinen Abreagierung ihres Triebes, wie is wohl im Wesen eines Mannes eher liegt.
Zwar findet er auch nur die bl{\ss}e Reaktion, Befreidigung kann es hier nicht geben.
Der Mensch ist eine Dreiheit von K\"{o}per-Seele-Geist.
(Nie k\"{o}nnen mir je Gedanken an Untreue kommen, als ich mir in Paris nach der wahnsinnigen Nachtrevue ein M\"{a}dchen zu 200 frcs. f\"{u}r eine Nacht kaufte.)
also mu{\ss} sich die Frau mit einem Geheimnis vor ihrem Gatten belasten.
Der mann tut nicht unrecht, der sich sp\"{a}ter, sei es, wann es wolle von solcher Frau l\"{o}st.
Die Ehe mu{\ss} zerr\"{u}tten.
Angenommen es gelingt, das Geheimnis zu h\"{u}ten, welch ein Charakter!
Und ist sie die Frau, sich klug

\clearpage
