\def\day{{\color{red} [16. Mai 1944, fortgesetzt] }}
\mktitle

vor allen Folgen zu bewahren?
Wohl wurde sie durch mich zur Frau, aber mehr lehrte ich sie nicht.
- Hat sie mir ja doch einmal entgegengehalten: Du merkst es ja nicht mehr!
Und damit ist  ein dritter Gedanke verbunden.
Weill sie mich gar nur pr\"{u}fen?
Das niedertr\"{a}chtige Recht steht Liebenden nicht zu.
Es w\"{a}re der beste Weg mich augenblicklich ganz zu verlieren.-
Was will nun eignetlich?
Ich will es ihr ja gerne glauben, da{\ss} ihr armes Herz in gro{\ss}en n\"{o}ten ist, aber ich kann ihr im Augenblick nicht helfen.
Ich kann ihr auch nicht schreiben, ja, es war gut was du getan {\color{red} [hast] }.
Wie furchtbar war mir doch der unterbewu{\ss}te Gedanke heute nacht, als ich von jenem schrekclichen Gewitterschlag auffuhr und die zuckenden Blitze durch die armselige Baracke grellten: jetzt l\"{a}ge sie gar in den Armen eines anderen.
- Nein, das trage ich nicht!-
Eines ist allerdings durch ihren Brief entschieden: Sie hat mir die M\"{o}glichkeit genommen, in ihr elternhaus heimkehren zu d\"{u}rfen.
Nun mu{\ss} ich sie iregendwo einst wiedersehen, an einem fremden Ort unter fremden Menschen.
so hatte ich es mir ja einmal gew\"{u}nscht, bevor sie die Verlobung beging.
Es ist mir auch klar, da{\ss} sie die Ferntrauung daraufhin, auf ihre eigene Frage, nicht eingehen wird
.
Denn das w\"{a}re mir doch undenkbar von ihr aus.
Es w\"{a}re mein Fehrler, da{\ss} ich  sie schon als meine Frau sah.
Und blieb zuletzt: Doch an der Macht des Schicksals zerschellt?

\clearpage
