\def\day{17. Mai 1944 15\begin{math}^{hmt}\end{math}}
\mktitle

Nun habe ich an sie geschriebem.
Ich begreife zwar noch nichts, aber sie Zwingt{\color{red} [sic] } mich ja zu schreiben.
Meine schaffenskraft ist hin, etwas gr\"{u}ndliches f\"{u}r die Schule zu tun.
Ich bin v\"{o}llig ersch\"{o}pft und so elend.
Geraucht habe ich irrsinnig viel.
Was hilft's?
Kein Mensch zum Aussprechen.
Es mu{\ss} irgendwo hin.
- Und was ich ihr geschrieben habe?
Ich hege keinen Groll, keinen Zorn, kann nicht einmal hassen.
Ich mu{\ss} sie lieben.
Aber ihre Frage ertrage ich nicht.
Kann ihr doch kein Mensch einen Vorwurf machen, wenn sie nicht das Leben Zwingt{\color{red} [sic] }.
Habe ich sie je auch nur aushorchend nach langen Zeiten allenseins{\color{red} [sic] } nach etwas gefragt?
Ich h\"{a}tte es jetzt auch niemals getan.-
Heute ist jener Tag, den sie gl\"{u}cklich pries.
"Wenn Du mir Dein Vertrauen schenken willst", sagte ich ihr beim Abschied nach jenem ersten kleinen, aber so bedeutungsvollen Ringen um einander.
Ja, der Tag war auch der Wendepunkt.
Von ihm an gewann ich dieses M\"{a}dchen, das so ahnugsvoll an dem ersten Beginn ihres jungen Lebens stand, st\"{a}ndig immer lieber, ich gewann sie ganz.
Mir war es so heilig ernst darum.
Und wenn sie jetzt meint es ginge nicht, so soll sich doch pr\"{u}fen, zum Himmel.
Oder meinen Ring senden.
Warum diese Frage?
Ich begreife nichts.

\clearpage
