\subsection{18. Mai 1944 10$^{hmt}$}
%\mktitle

Nachdem ich gestern den Brief an sie geschrieben hatte, mu{\ss}te ich hinaus, um eine Runde zu drehen.
Es tat mir gut und ein guter Geist wohl gab es mir ein, den Brief noch einmal zu \"{u}berlegen.
Sollte, es denn keine M\"{o}glichkeit geben, ihr auf einem anderen Wege solche "dr\"{o}meligen" Gedanken auszutreiben?
Jawohl es ging.
Ich habe ihr eine Karte geschrieben, auf die hin sie nur noch mit Kopfe sch\"{u}tteln kann.
Ich will aber hoffen, das ihr aus \st{au} dieser Karte, die gar robust, aber nicht grob abgefa{\ss}t ist, auch klar wird, da{\ss} ich nach drei Jahren nicht mehr auf derartige Einf\"{a}lle von ihr eingehen kann.
Ich will es einmal so herum wagen, zumal es mir ein gutes Zeichen schien, da{\ss} nach der Runde der Brief am offenen Fenster verregnet und ganz gr\"{u}n eingelaufen war.
- In diesen Tagen mu{\ss} ja auch die Ferntrauungsurkunde zu Haus eintreffen.
Wenn sie ablehnt, dann ist nichts mehr zu machen.
Am wenigsten, falls irgendwelche konfessionellen gr\"{u}nde vorgebracht werden.
F\"{u}r irgendwelche komische{\color{red} [sic] } Gef\"{u}hle und Empfindungen h\"{a}tte ich auch wenig Verst\"{a}ndnis.
Nachdem sie alle zu Haus die Verlobung inszeniert haben, will ich wenigstens bei der Trauung ein W\"{o}rtchen mitreden.
Au{\ss}erdem k\"{o}nnte sie den kleinen Gehaltsunterschied von nur 100,- RM. kaum beibringen u. es t\"{a}te mir leid, ihr den Betrag sp\"{a}ter f\"{u}r die verlorene Zeit einmal vom Wirtschaftsgeld abzuziehen.

\clearpage
