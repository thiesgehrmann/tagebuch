\subsection{18 Mai 1944 18$^{15}$}
%\mktitle

Nach dem Abendessen wollte ich noch arbeiten.
Es war mir nicht m\"{o}glich.
Ich lief zur Post und holte meine Karte zur\"{u}ck.
Nichts geht fort.
Ich werde jetzt warten, bis eine Antwort auf die Ferntrauung kommt.
Ich glaube nicht mehr, da{\ss} sie sie eingeht.
Auch ist es mir nicht gegeben, in dem scherzenden Ton dar\"{u}berhinzugehen.
Ich habe noch an mich gehalten, ihr den Brief zu senden, obwohl nichts Verletzendes darin sein k\"{o}nnte.
Aber warum soll ich nicht aus meiner Lage heraus \"{u}berhaupt nur anscheinend wehren?-
Ganz ehrlich, es ist doch so: ein Mann, der sich ein Lebenswerk schaffen und aufbauen will, will einen Erben haben, S\"{o}hne, die seinen Namen tragen.
Da nimmt er sich eine Frau.
Er gibt ihr seine Liebe, darin sie kraft ihrer Veranlagung, sowohl k\"{o}rperlich wie seelisch - geistig allemal -, v\"{o}llig aufzugehen vermag.
Jetzt kommt das M\"{a}del daher.
und{\color{red} [sic] } stellt mit ihren Zweifel einfach den Ursprung dieser Kinder in Frage, das blo{\ss} weil der Mann im Daseinskampf seines Volkes eine gewisse Zeit nicht bei ihr sein kann.
Sie tut es als Verlobte, was w\"{u}rde sie als Frau hindern?
Nein, eine Frau ist kein Mann.
Ungleich mehr steht auf dem Spiele.
Es war zuviel.
Ich ertrage es nicht.
Wohl wollte ich um meine Liebe k\"{a}mpfen, aber wenn sich die Frau ihrer w\"{u}rde, H\"{u}terin des Volkes zu sein, selbst begeben will, ihrer Gef\"{u}hls\"{a}u{\ss}erungen wegen und ein Mi{\ss}verstehen einer Novelle?
Nein, mein kind es war ein Mann dort und keine Frau.-
Ich kann da nicht mehr mit.

\clearpage
