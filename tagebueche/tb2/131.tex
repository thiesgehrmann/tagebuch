\subsection{In der Mittagsruhe 20. Mai 1944}
%\mktitle

Ihre Frage l\"{a}{\ss}t mich nicht los.
(Dabei m\"{u}{\ss}te ich f\"{u}r die Schule arbeiten!)
Ich habe an michgehalten, nicht noch einmal, in den Fehler zu verfallen, ein so schweigendes Wort im Brief festzuhalten, der \"{u}ber ein Meer und fast ein Vierteljahr ganz andere Aufnahmebedingungen erf\"{a}hrt als die, unter denen er geschrieben wurde.
Das will ich aber f\"{u}r mich festhalten: Kann eine Frau dem Manne in ihrer Frage auch nur verglichen werden?
Nein.
Die germanischen V\"{o}lker \"{u}bergaben solche Frauen gnadenlos dem Tode.
Mit Recht.
\st{S\"{u}dli} Und sie eroberten die Welt.
S\"{u}dliche V\"{o}lker lebten anders; sie namen die {\color{red} [8.5cm geschw\"{a}rzt, vermutlich vom Schreiber] Offenbarungsreligion mit ihrer Verhei{\ss}ung im jenseits an, weil sie zu schwach waren, im diesseits[sic] } zu bestehen.
- Ist es nur eine Frage der Sitte und der Moral, die es untersagt, einer Frau die Gattinrechte anzuerkennen, die einem fremen Manne sich hingibt, sei es unter welchen Umst\"{a}nden es wolle?
Nein.
Da liegt ein tiefer ethischer Grundgedanke darin, nachdem die V\"{o}lker, welche naturhaft biologisch denken - und solch Denken ist f\"{u}r Mischrassen immer fatal - schon immer lebten.
Und was tun wir, die wir dabei sind, diese geheimsten Gesetze dank unserer Wissenschaften zu ergr\"{u}nden?
Wir sprechen mit geschlagenen Sinnen von "verzeihen"{\color{red} [sic] }.
Was tut den die Frau, die "ihrer Schw\"{a}che erliegt", anderes, als das sie mit dem blo{\ss}en Zweifel allein schon die Herkunft der Nachkommen ihres Gatten in Frage stellt.
Was sie heut tat, kann sie morgen wieder tun.
Und freund{\color{red} [sic] } bleibt ihr der andere.
Sie vergeht sich an den Lebensgesetzen ihres Volkes.
Sie ist es ja, die die Kinder gebiert und nicht der Mann.
Ein zweifel an ihrer aufgabe an der Reinerhaltung, ihrer Pflicht, d\"{u}rfte eine Braut, die noch dazu gl\"{u}cklich sein k\"{o}nnte im{\color{red} [sic] }

\clearpage
