\def\day{21. Mai 1944}
\mktitle

Mein Brief an meine Eltern ist fort.
Es tut mir sehr weh, Ihnen{\color{red} [sic] } dies schreiben zu m\"{u}ssen, aber eine Heimstatt mu{\ss} doch die geq\"{a}lte Seele finden.
Andere laufen der Kirche in die Arme, die ihre Herrschaft auf der Not derer aufbaut und Gott ist weit in der Gefangenschaft.
Wenn ich es vermochte, ihr zu schreiben, ich schriebe ihr folgende Worte: L. H., ich habe deinen Brief, den Du zwei Tage nach D.23. Geburtstag schriebst bekommen{\color{red} [sic] }
6 Jahre nach dem Tag, den Du selbst dankbar gl\"{u}cklich \st{preistest} priesest.
Alles was ich liebe, worann ich glaubte, was ich hoffte und worauf ich vertraute hasst Du mir mit Deinem Zweifel zerst\"{o}rt.
Dankbar Gedenke{\color{red} [sic] } ich da der Freundin, die mich Dir einst freigab und die mich doch in ihrem reinen Gk\"{u}ck mit Gattin und Kind \st{im} Erinnern verga{\ss}.
Du hast mich zum armen Bettler gemacht.
Doch kann ich Dich nicht hassen, ich liebe dich ja mit reinem und ganzem Herzen.
So la{\ss}{\color{red} [sic] } Dich noch einmal gr\"{u}{\ss}en von Deinem H. -
O, w\"{a}r es mir nur verg\"{o}nnt um meine Liebe zu k\"{a}mpfen!
Sollte sie nicht sehend und erkennend werden?
Aber wie kann ich es in der Gefangenschaft?-
Ich habe den Professor um 14 Tage Dispens gebeten.
Ich mu{\ss} ja hindurch.

\clearpage
