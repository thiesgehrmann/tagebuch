\subsection{7. Juni 1944.}
%\mktitle

Seit Montag stehen wir in der m\"{u}ndlichen Vorpr\"{u}fung.
Die letzten drei Tage vor der Pr\"{u}fung bin ich schon wieder hingegangen.
Es gab mir mut, da{\ss} ich von allen Seiten bereitwilligstes Verst\"{a}ndnis f\"{u}r meine Bitte fand, ohne da{\ss} einer der Betreffenden in mich dringen wollte.
- Alles Geschehen aber wird zur Zeit von dem Geschehen \"{u}berschattet, das sich an der franz\"{o}sischen Kanalk\"{u}ste seit gestern einleitete: Die amerikaner und Engl\"{a}nder haben den Angriff auf Europa begonnen.
Das soll der Anfang vom Ende sein.
Bitter genug, da{\ss} wir nur zusehen d\"{u}rfen.
Aber davon mu{\ss} die Entscheidung abh\"{a}ngen.
Und dann mu{\ss} in nicht allzuferner Zeit der Tag kommen, der mich die Heimat wiedersehen l\"{a}sst.
Ob ich da ein Heim finden werden, da{\ss} au{\ss}er meinen{\color{red} [sic] } Elternhaus mir ganz allein offen stehen sollte mit einem so herzig geliebten M\"{a}dchen darin, die mir treu geblieben ist.-
Da kommt der Kline.
Ich sage, Du, wenn ich Dich einmal mit Christel treffe, ... - Aus, sagt er mir, die hat sich verlobt!
-Der andere macht Ferntrauung, kriegt seine Frau ein Kind.
- Die dritte l\"{o}st die Verlobung mit einem Kg., heiratet, ihr mann f\"{a}llt; nun schreibt sie, ob der alte sie nicht wieder heiraten m\"{o}chte.
- Ich glaube unbesehen alles, was mir auf diesem Gebiet berichtet wird.
"Gut und engelrein ist die Deutsche Frau." sang Herr Walther.
- Das augenblickliche Recht sch\"{u}tz die Frau zu allgemein.
Man sollte in jedem einzelnen Falle das Verwerfliche des Charakters und der Gesinnung heraustellen und die Frau dann bestrafen.


\clearpage
