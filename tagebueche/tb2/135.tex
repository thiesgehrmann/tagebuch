\subsection{10. Juni 1944}
%\mktitle

Ich glaubte, ich k\"{o}nnte heute schon wieder meine Briefe vornehmen, aber gefehlt.
Noch viel zu fr\"{u}h.
Da habe ich nun den R\"{u}ckschlag.
mein kopf ist schwer, mein hinr schwimmt, bieten kann ich nichts.
Nureinen Wunsch, nichts denken m\"{u}ssen.
und das jetzt mitten in der Pr\"{u}fung.
Es ist ja auch gleich.
Wenn nur die wahnsinnige Schwere der Welt nicht w\"{a}re.
- Ich wollte schreiben.
Ich kriege keinen Brief fertig, der nicht aus tief verwundetem Herzen k\"{a}me.
Drei Tage mu{\ss} ich noch warten, bis die Pr\"{u}fung vor\"{u}ber ist.
Dann will ich meinen Eltern wieder schreiben.
Sie d\"{u}rfen nicht ohne Post ein.-
Was soll ich noch niederschreiben?
Ich sitze normal an meinem Tisch.
Keiner der Umstehenden merkt mir etwas an.
Aber in meinem kopf geht es schwer wie ein M\"{u}hlrad um.
Der Kleine hat ja recht, wir m\"{u}ssen alles tragen.
Alle Gr\"{u}nde sind nicht entschuldbar vor der einen Verpflichtung, das Leben zu tragen, was immer es uns auch bringen mag!
Ausharren bis zum Letzten.
Viel ist es ja nicht mehr.-
Was k\"{o}nnte in diesen Tagen f\"{u}r eine Stimmung sein im Lager.
Doch h\"{o}rt und sieht man nicht.
So schleicht sich das leben weiter hin.
Dahin, dahin ist so Vieles!


\clearpage
