\def\day{11. Juni 1944.}
\mktitle

Ich sagte mir gestern, nachdem ich f\"{u}r die \st{Chemiep} Physikpr\"{u}fung nichts vorbereitet hatte, wenn Du heute dumm auff\"{a}llst, mu{\ss}t Du morgen Chemie arbeiten.
Ich bin nicht dumm aufgefallen, also ... habe ich auch nichts f\"{u}r Chemie getan.
Englisch u. Franz\"{o}sisch erledigt sich am Dienstag u. Mittwoch, dann hat die liebe Seele Ruh.
Ferien gibt es aber nicht.
Nur drei Stunden am Tage.
Gen\"{u}gt auch, wenn es so hei{\ss} wird wie heute.
- Ich las meine Briefe an sie aus April u. Mai nach.
Herrgott sie k\"{o}nnte sich doch freuen.
Und dann liegt mein Brief vom 17. Mai noch hier.
Ich werde das Datum \"{a}ndern und ihn am 17. Juni aufgeben, wenn nicht noch wichtige Briefe eintreffen und meinen Entschlu{\ss} \"{a}ndern sollten.
Der Brief ist klar und gut, vor allem dazu angetan, sie wach zu r\"{u}tteln, sollte sie getr\"{a}umt haben, und ihr die Grenzen aufzuzeigen.
Ja, sie mu{\ss} hn haben.
Es mu{\ss} eben doch die Auseinandersetzung folgen, die ich nicht im Sinne hatte, solange ich \st{selbst} voll vertrauen\st{d} konnte.
Aber wenn sie mir selbs mit der zweifelnden Frage entgegentritt?
Vielleicht ist auch dies ein noch notwendiger Schritt?
Wie schrieb sie doch?
Das Leben lege einem nur soviel auf, wie eins zu ertragen vermag.
Ich w\"{u}nschte mir ja nur, sie k\"{o}nne sich mit dem Brief zu wahrer reiner G\"{u}te durchringen.
Deutschland braucht edle Menschen.

\clearpage
