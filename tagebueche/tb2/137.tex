\def\day{16. Juni 1944.}
\mktitle
Gestern war die Schulklasse zum Besuch auf der Farm.-
Es sind andere Briefe gekommen.
ich werden den betreffenden{\color{red} [sic] } nicht am 17. Juni absenden.
Ich werde ihn \"{u}berhaupt nicht absenden.
Ich wollte schreiben, wollte gern schreiben.
Aber es wird mir alles so schwer.
Wenn ich mich nur von meiner Umwelt freimachen k\"{o}nnte.
So grau und bleiern, wie heut der Himmel ist, so ist es die ganze Zeit schon in mir.
Ihre Gedanken, die sie wohl haben mochte - warum sollte sie einer wehren? - die tat \ul{sie} ja bald ab und schon drei Tage sp\"{a}ter erkl\"{a}rte sie es f\"{u}r unm\"{o}glich, dergleichen im Brief niederzuschreiben.
Mir war es jedoch \"{u}berlassen, in der Zwischenzeit der beiden Briefe, aus der so mittlerweile vier volle Wochen geworden waren, ihre "Gedanken" bis zur letzten Konsequenz durchzudenken.
Was ist nun das Richtige, aus dem Wust dieser Gedankeng\"{a}nge herauszuw\"{a}hlen und ihr in Briefen hinzuschreiben?
Ich kann hier Briefe hassen lernen.-
Im tiefsten Wesen l\"{a}{\ss}t mich auch alle Freude an dem kleinen M\"{a}dchen kalt, an ihrem zimmer, an allem.
Was soll ich da noch zwischen!
Freuen und lachen{\color{red} [sic] } verlernt, den Glauben an sie dahin.
Das leben l\"{a}uft weiter, mir bringt es nichts mehr f\"{u}r einen, der sehenden Auges zuschauen mu{\ss}, wie es ihm gegen jeden Willen in den H\"{a}nden zerrinnt und das Wenige, das er zu halten vermag, ihm vom Schicksal rausgeschlagen wird.
Wozu das alles ? -

\clearpage
