\subsection{18. Juni 1944}
%\mktitle

Ich habe nun einen Brief heute an sie geschrieben.
Anfangs ging es ja so, wie ich es wollte, aber nachher kam alles anders, bestimmt so, wie es kommen mu{\ss}te.
Soll er fortgehen.
Tue ich denn jemandem Unrecht, wenn ich auch einmal schreibe, wie es mir ums Herz ist?
Sollen sie sich zu Hause einmal \"{u}berlegen.
Vater meinte, ob ich zu seinem 60. Geburtstag dabei sein werde.
Ach, es w\"{a}re sch\"{o}n.
- Ob ich einen Freund habe, fragte sie.
Ja, habe ich einen Freund?
Ich glaube schon, da{\ss} ich einem Menschen Freund sein k\"{o}nnte.
Geben kann ich.
Nur kann ich nicht mit denen mit, denen Freundschaft und Kameradschaft nicht \"{u}ber die gewissenlose Kumpelei hinausgehen.
- Da ist der Hans Kunert.
Er behauptet, ich h\"{a}tte ihn beleidigt.
Ich biete ihm Rechenschaft an.
Er schl\"{a}gt sie mir ab.
ich bitte einen ihm befreundeten \"{a}lteren Portepeetr\"{a}ger, eine Aussprache zu vermitteln.
Er erkl\"{a}rt dem, die ganze Sache sei nicht so tragisch, er werde das schon einr\"{a}nken.
Nach acht Tagen, nach beendeter pr\"{u}fung, erfolgt nicht.
Das ist seine von ihm so geforderte Haltung eines Unteroffiziers.
Na ja, es ist so vieles dahin.
Kann ruhig noch ein bi{\ss}chen mehr hingehen.-
Hans, bist Du es noch, der da so spricht?
Willst Du gar nicht mehr glauben und hoffen?-
Ich las vorgestern noch all ihre Briefe, seit sie von der Wohnung und der Ferntrauung schrieb.
Ich mu{\ss} ja warten, warten u. warten.
Worauf?-
\clearpage
