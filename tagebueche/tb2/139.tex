\def\day{23. 6. 1944}
\mktitle

"Aber vielleicht wird Dir manches klarer, wenn ich Dir sage, das Herr Amelung mir mitteilte, da{\ss} Du sofort bei Deiner R\"{u}ckkehr als au{\ss}erplanm\"{a}{\ss}iger Inspektor eingestzt wirst."
Dieser Satz stand in ihrem Brief vm 12. M\"{a}rz 1944.
Was folgte, war eine jener k\"{o}stlichen Schelte und Zurechtweisung, die mir so recht von Herzen wohl tun und die ich auch restlos hinnehmen und ertragen kann.
Jener Satz jedaoch scheint f\"{u}r mein weiteres Arbeiten und Lernen von ausschlaggebender Bedeutung zu sein.
Wie der lehrplan es vorgesehen hatte, habe ich jetzt ein volles Jahr zur Vorbereitung auf die Reifepr\"{u}fung gearbeitet.
Wie wenn ich mir diese Arbeit mit dem in der Vorpr\"{u}fung erreichten Zeugnis best\"{a}tigen lie{\ss}e und, sofern man mir die Hefte, "Weg zur Reifepr\"{u}fung" belie{\ss}e, in dem allgemeinen Stoffgebiet f\"{u}r mich arbeitete, das Hauptgewicht aber auf ein Einarbeiten und Vorbereiten meines Berufes legte.
Denn die erste Anforderung hei{\ss}t f\"{u}r mich in Deutschland - auch das soll nach dem Fall {\color{red} [ein Wort geschw\"{a}rzt, wohl vom Schreiber] } nicht meher allzulange dauern! - , so schnell wie m\"{o}glich eine vollwertige einsatzf\"{a}hige Arbeit als Verwaltungsbeamter im Intendanturdienst zu leisten.
Und damit und mit den anderen Briefen ist mir ein gewaltiger Auftrieb geworden.
So soll mir ihre Frage eine ernste, wenn auch bittere Erfahrung sein.
Sie m\"{u}sste eigentlich zu Ferntrauung mit beiden H\"{a}nden greifen.
K\"{o}nnte ihr Zweifeln kr\"{a}ftiger behoben sein?
Ich will \st{I} ihr einen sch\"{o}nen Brief schreiben, den Eltern vor allem u. Hertha, Paul u. auch vielleicht Grumann: die Geschichte war nett.
\clearpage
