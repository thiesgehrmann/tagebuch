\subsection{27. Juni 1944}
%\mktitle

Nun mu{\ss} ich noch einen Brief an die Eltern schreiben und die letzte Karte bekommt sie.
Dann bin ich wieder mitten drin und erwarte h\"{o}chst gespannt und auch noch erfreut ihre Antwort auf meine Februarpost.
Im Juli kann ich gleich anfangs in zwei langen ausf\"{u}hrlichen Briefen auf vieles eingehen, das liegen bleiben mu{\ss}te.
Sie soll ja Freude haben von meinen Breifen.
- mein Entschlu{\ss}, mich in meiner Arbeit nun nach einem Jahr Abiturvorbereitung \st{ganz} wieder auf mich allein umzustellen, reift langsam aber wohl sicher in mir.
Ich mu{\ss} nur erst noch die richtige Form finden und mir zurechtlegen, in der ich es dem professor unterbreiten kann.
Denn dabei mu{\ss} ich \"{a}u{\ss}erst geschickt vorgehen.
Vorerst will ich mir einen genauen Arbeitsplan aufstellen, nach/dem ich in einem beweglichen, anpassungsf\"{a}higen System die beiden Ziele meiner Lernarbeit f\"{u}r die Zeit bis zum Ende der Gefangenschaft im richtigen Verh\"{a}ltnis verquicken kann.-
F\"{u}r mich liegt es klar: Ich kann mich im Grunde genommen nicht l\"{a}nger einer Zensierung durch \st{v\"{o}llig} wohl fachlich ausreichende Lehrkr\"{a}fte unterziehen, die aber Spezialisten sind und denen eben die geschulte Weitsich{\color{red} [t] } des P\"{a}dagogen restlos fehlt.
Au{\ss}erdem halte ich mich jeder Zeit \st{imst} instande, nach der Heimkehr die Reifepr\"{u}fung an jeder h\"{o}heren Schule abzulegen, w\"{a}hrend die Aussichten auf eine hiesige Pr\"{u}fungsabnahme mit Riesenschritten davoneilen.
Es mu{\ss} wieder mal etwas unternommen werden.

\clearpage
