\def\day{28. Juni 1944.}
\mktitle

Heute habe ich  mit dem professor gesprochen.
Es is{\color{red} [t] } anders gekommen.
Fest steht vorerst allein folgendes: Der Lehrgang soll bis zum 1. Dezember durchgef\"{u}hrt werden.
Die Pr\"{u}fungsbestimmungen, die hier vorliegen, sind eine rein private Ausarbeitung von 14 Studienr\"{a}ten aus dem Lager Bowmanville.
Auch diese sind ohne Richtlienien aus Deutschland entstanden.
Immerhin sind sie gut.
An die Reisegenehmigung einer Pr\"{u}fungskommision in Kanada glaube ich nicht, obwohl Dr. Wulf heute noch nichts davon wu{\ss}te.
Also verbleibt eine Schlu{\ss}pr\"{u}fung im Rahmen der hiesigen Lehrkr\"{a}fte unter seinem Vorsitz mit gew\"{a}hlten Amateurbeisitzern.
Danach kann man uns eine Bescheinigung \"{u}ber die Absolvierung des zuende gef\"{u}hrten geschlossenen Lehrgangs aussstellen.
Bleibt die Frage, was man in in Deutschland daf\"{u}r geben wird.
Ich teile auch jetzt noch nicht seine Ansicht, da{\ss} die Ablegung einer Pr\"{u}fung in Deutschland als Soldat u. Kriegsgefangener, erhebliche Schwierigkeiten bieten k\"{o}nne.
Aber nachdem ich mit der Erw\"{a}gung des Planes einer auf mich gestellten Arbeit wieder den n\"{o}tigen Abstand zur Sache gewonnen habe, kann ich mich dem Einwand nciht verschlie{\ss}en, da{\ss} eine halb durchgef\"{u}hrte Sache nicht aufgegeben werden darf.
So werde ich weiterarbeiten und damit an mir arbeiten.
Keine schlechte Erziehung daf\"{u}r ist auch der augenblickliche Konflikt mit Kunert.
Noch ist die Zeit nicht gekommen, Hans, da{\ss} Du frei entschliessen kannst.
Aber halte Dich bereit.

\clearpage
