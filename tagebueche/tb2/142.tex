\subsection{1. Juli 1944.}
%\mktitle

Ich habe noch keine Aprilpost, also warte ich auch noch voll innerer Ungeduld und Spannung auf die Beantwortung meiner Februarbriefe.
- Im Lager \"{u}berschlagen sich die ereignisse.
300 Neue sollen kommen, heute sind es nur noch 1000.
E{\color{red} [nglisch] u. F[ranz\"{o}sisch] } werden er\"{o}ffnet, heute nur noch E.
Schulklassen aus E u. F sind exmittiert.
Wir sollen eine Kopfbaracke zugewiesen bekommen.-
Das Buch von Bodenreuth habe ich erworben.
Ein sch\"{o}ner Besitz f\"{u}r mich, so ein Buch.-
Taggewitter, Blitzeinschlag in C4.
Alles rennt, die pomp\"{o}se Feuerwehr kommt herangebraust.
Erstaunlich schnell kommet Wasser, aber der Handfeuerl\"{o}scher hatte schon alles erledigt.
Man sah den guten Willen.
- So recht hundem\"{u}de bin ich heut abend.
Gestern konnte ich noch einmal die Fram besuchen.
Es war ein Wetter wie in den Hundstagen.
Herrliche sch\"{o}n u. strahlend der Morgen.
Im Schatten einiger Pappeln verzehrte ich das Fr\"{u}hst\"{u}cksbrot, zu dem mein Weggenosse guten Bohnenkaffee mitgebracht hatte.
Mittag im zelt.
Nachmittag am Flu{\ss}.
Das Baden ist streng verboten.
Das mu{\ss} es auch, sonst ginge ja jeder Lackel hinein, wenn er auch nicht im mindesten mit dem Wasser vertraut w\"{a}re.
Aber herrlich ist es, sich frei in die flut zu werfen und mit dem starken Strom zu treiben.
Ich habe mich wie lange nicht mehr ganz im vollbesitz aller jugendlichen kraft gef\"{u}hlt.
(Was brauchte ich dazu ein{\color{red} [sic] } Badehose!)
Es war ein sch\"{o}ner Tag.

\clearpage
