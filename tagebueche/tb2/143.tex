\def\day{22. juli 1944.}
\mktitle

Einer wahnsinnigen Intrige soll unsere Liebe zum Opfer fallen.
Am 18. Juli erhalte ich ihren brief vom 17. April.
Ich lese ihn und begreife nichts.
Ich mu{\ss}te aus diesem Brief annehmen, da{\ss} es ihre enzige Antwort auf meinen Februarbrief ist.
Mu{\ss}te es, kein Wort nimmt Bezug auf einen fr\"{u}heren Brief, den sie geschrieben haben k\"{o}nnte, nachdem sie den Meinen erhalten hatte.
So mu{\ss}te ich handeln, und mu{\ss}te ich nicht auch so handeln, wie ich es tat?
K\"{o}nnte ich diesen ihren Brief vom 17. April als einzige Reaktion darauf ertragen, da{\ss} ich ihr die Hand zum Lebensbunde reichte? -
Wer hat diesem Trug verbrochen, da{\ss} ich ihren Brief vom 9. April erst heute erhalte?
Wer hat \st{ihr} sie den zweiten Brief zu schreiben gehei{\ss}en?
Am 9.4. war sie allein und konnte mir mit ihren eigenen Gedanken diesen Brief schreiben, von ihrer Einstellung zu Kirche - o, diese einf\"{a}ltigen, unschuldigen Gedanken!
Wie achte und ehre ich sie an meiner Frau!
Ich werde mit ihr zur Kirche gehen.
Wo sollte sie auch den Trugschlu{\ss} \"{u}ber die Fr\"{o}mmigkeit \st{an} heraus erkennen! -
da{\ss}{\color{red} [sic] sie meine Frage nach dem, was sie hat, nicht versteht, aber doch antwortet. [???] }
Alles scheint gut und klar vor ihrem Herzen und ihrem verstande.
Da mu{\ss} einer gekommen sein, der sie verleitet hat, jenen anderen Brief zu schreiben.
Zwei Punkte, die erst klar \"{u}berwunden waren, wenn etwas zu \"{u}berwinden war, greift sie heraus und rechtet mit mir darum.
Wer hat sich ihr das eingegeben?
Und ich mu{\ss}, durch die "menschenfreundliche Zensur", darauf hineinfallen und mu{\ss}, ich konnte nicht anders, an ihre entschlossene Ablehnung glauben.

\clearpage
