\def\day{22. 7. 1944}
\mktitle

O, wenn ich meinem n\"{a}chsten Brief Fl\"{u}gel geben k\"{o}nnte, wenn die Vorsehung ein wenig Gnade nur zwei reinen liebenden Herzen gew\"{a}ren m\"{o}chte, die an der gro{\ss}en Not der Zeit allein wo sonst verzagen m\"{o}chten.
K\"{a}me meine karte doch erst, wenn sie dann schon getraut, ginge mein n\"{a}chster Brief doch einmal nur schnell als meine letzte Karte.
Ewig wollte ich dem Schicksal danken, da{\ss} es das liebe Herz, das so viel um mich duldet und leidet, die Kr\"{a}nkung, die daraus kommen mu{\ss}, weil ich, wenn auch ohne eigene Schuld, ja, vielleicht grade durch anderer Absicht oder Schuld, verblendet sie in ein Ungl\"{u}ck sto{\ss}en k\"{o}nnte.-
Nichts sehe ich mehr, als ihre Liebe, die ich durch geschriebene Briefe bedroht glaubte.
Nun mu{\ss} ich erkennen, da{\ss} eine dritte feindliche Gewalt dieses grausame Spiel heraufbeschworen hat, das unsere Liebe nun vielleicht zerst\"{o}ren sollte.
K\"{o}nnte sie das?
Und wenn sie nur wartete, ich m\"{u}{\ss}te ihr verzeihen.
Und wie kann ich von ihr lassen!
Sie war die Meine und eine andere kann ich nie mehr so lieben.
Sie mu{\ss} es werden und wenn die Welt dr\"{u}ber verginge.
B\"{o}s sieht es aus in ihr.
Doch \"{u}ber allem leuchtet der Sieg.
{\color{red} [Zwei worte geschw\"{a}rzt, "Das Reich"?] } wird ihn erringen.
Gott hat es so gewollt.
Der Gott, der den men Menschen das Blut gegeben hat, aus dem sie an ihm glauben.

\clearpage
