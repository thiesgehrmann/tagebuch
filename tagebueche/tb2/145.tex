\subsection{{\color{red} [22. 7. 1944] }}
%\mktitle
{\color{red} [9.5 Zeilen weggeschnitten, vermutlich von der Ehefrau] }

- Ich sehe den Abend in Hamburg vor mir.
Es ist der Tag gewesen, an dem zum allerersten Mal im ersten Kriegsjahr ein Wort \"{u}ber das fiel was die Herzen schon lange beschlossen hatten.
Abends baren wir dann im Dunkeln schon an der Au{\ss}enalster entlanggewandelt.
Ganz deutlich sehe ich das M\"{a}dchen vor mir.
Wir standen auf dem Steg, rings lag das tiefschwarze Wasser nur nach Osten hin \st{fols} flo{\ss} das glitzernde Mondeslicht in silbernen Str\"{o}men.
Dort war es, als ihr mit meinem Versprechen zur Ehe auch gleich so deutich und eindringlich es sagte, da{\ss} einmal, wenn es so weit w\"{a}re, diese Frage von ihr endg\"{u}ltig entschieden sein m\"{u}{\ss}te, da{\ss} sie dann nur zu mier halten k\"{o}nne und mich dann nichts zu einem Kompromi{\ss} bewegen k\"{o}nne.
Ich konnte es ihr nicht klarer hinstellen.
Ob sie wohl auch an jenen Abend mitgedacht hat?
Warum mu{\ss}te ich sie so lange allein lassen?
Und wie lange noch?
Alles wollte ich tragen, wenn sie nur zu mir st\"{a}nde.

\clearpage
