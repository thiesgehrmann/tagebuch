\def\day{28. Juni 1944}
\mktitle

Langsam aber sicher geht das Leben doch in Wahnsinn \"{u}ber.
Erst verlog ich alle Freundschaft allein durch die Umst\"{a}nde.
Jeder hat ja mit sich selbst zu tun und was z\"{a}hlt schon ein Kriegsgefangener!
Die Jahre gehen dahin.
Es verbleibt nicht mehr viel.
Verdampt wenn man doch einmal mit der Faust dreinschlagen k\"{o}nnte.
Schreibt sie mir einfach, ich sei ihr fremd geworden.
Wenn ich noch einmal schreiben sollte, dann wohl nur an meine Frau.
Sollte sie es fertigbringen, die Ferntrauungsurkunde zur\"{u}ckgehen zu lassen, sei es, das sie vor dem Standesamt "nein" sagt oder die Frist bis zum 7. November verstreichen l\"{a}{\ss}t, dann sehe ich keinen Weg mehr f\"{u}r uns.
Sie war es, die einen Zweifel aufstellte zwischen uns.
Und ich war so dumm, zu glauben.
Auch ihren ersten Brief kan ich allem Dingen unterschreiben.
Warum aber widerruft sie ihn mit dem zweiten.-
Ihre Behauptung seit ihren Geburtstag fortlaufend, da{\ss} drei Jahre eine lange Zeit seihen!
Wann bin ich je darauf eingegangen?
Ich habe ihr wahrlich die Treue gehalten und seit ich in der Gefangenschaft sitze - Du Dummer, Hans, warm bist Du denn hineingegangen? - , war sie mein einziger Gedanke.
Jetzt sch\"{a}me ich mich, den Kameraden zu begegnen.
Was soll ich sagen, wenn sie fragen?
Soll ich sagen, meine Braut versteht mich nicht mehr!?!
- Ihr armen Gefangenen, ihr seid verflucht mit gebundenen H\"{a}nden dem Leben zuzusehen.
Und keiner fragt nach euch{\color{red} [sic] }

\clearpage
