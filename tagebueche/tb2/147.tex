\def\day{29. Juli 1944}
\mktitle

Heute ist Vaters Brief vom 16. April gekommen.
Also so sieht es aus!
Was sind wir doch ein armes, armes Volk.
Der Staat gibt seinen B\"{u}rgern die M\"{o}glichkeit, die Ehe auch dann einzugehen, wenn der eine Partner nicht Anwesendsein kann.
Aber zu einer Willenserkl\"{a}rung {\color{red} [2 Worte geschw\"{a}rzt] }, was einem besseren Vereinsaustritt gleichkommt, \st{erl}l\"{a}{\ss}t dieser selbe Staat noch Gesetze bestehen, die ein pers\"{o}nliches Erscheinen des Betreffenden erfordern.
In meiner augenblicklichen Wut m\"{o}chte ich den Amtsgerichtsrad alle Knochen entzweischlagen und ihn an die Ostfront jagen.
Aber der ist ja auch blo{\ss} ein armer Paragraphenhengst.
Wass soll dan werden?
Der Wisch wird zur\"{u}ckkommen.
Erreiche ich nichts, als dass durch diese verfluchte zensierte Briefschreiberei alle Beteiligten ver\"{a}rgert und verbittert sind.Und ich sitze hier und lerne f\"{u}r das Abitur.
Es ist zum Verzweifeln.
Wass soll ich denn da \"{u}berhaupt noch schreiben!
- Da sitzt man nn in einem Lager, {\color{red} [1 Zeile geschw\"{a}rzt] }.
Es stinkt zum Himmel!
- Ja, Hans, Du sollst dich \"{u}ber gar nichts freuen, alles soll Dir scheinbar versagt sein.
Das geringste bi{\ss}chen Freude ist verg\"{a}llt.
Ich danke bald f\"{u}r so ein Hundeleben.
Und da gehen diese ausgetauschten Idioten noch dabei und halten Vortr\"{a}ge \"{u}ber die Gefangenschaft.
Wahnsinn, Wahnsinn, Wahnsinn!

\clearpage
