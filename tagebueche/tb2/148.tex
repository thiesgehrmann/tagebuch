\def\day{4. 8. 1944.}
\mktitle

So langsam sind un von allen beteiligten Seiten die Briefe eingetroffen; es ist gut von allen.
So auch von ihrer Mutter.
Da konnte ich mir schon bedeutent eher ein Bild machen.
So konnte ich auch an sie schreiben.
Ich danke es dem Schicksal, das mir ein Freund bereitwillig dabei hilft.
Wie dieser Brief schon entstanden ist!
Nachdem ich schon aus mehreren R\"{a}umen hinausgeflogen war, traf ich Walter N., der mir ohne zu fragen sein Zimmer, das er aus besonderen Gr\"{u}nden bewohnt, sofort zur Verf\"{u}gung stellte.
Dort konnte ich drei Stunden ungest\"{o}rt in einem Gu{\ss} einen Brief aufsetzen, wobei das gro{\ss}e Orchester die herrlichsten Melodien \"{u}bte.
Ich freute mich allein schon dar\"{u}ber.
Auch brauchte ich meinen Brief in nichts mehr abzu\"{a}ndern, als Gertruds Brief vom 2.5. mit denen von ihr (1.5.) u.meinen Eltern (1.5.) kamen.
Nur eindeutiger konnte ich meine Frage an sie stellen, ob die W\"{u}nsche ihrer Mutter auch ihre Entschl\"{u}sse sind.
Mein Gott, was hat man mir aus dem M\"{a}del gemacht?
Und ich liebe sie, mu{\ss} sie lieben!
Und sie wei{\ss} ja nicht - wenn ich es ihr doch sagen k\"{o}nnte! - welches leben ich ihr zu geben imstande und willens bin.
Alles was sie sich w\"{u}nscht wird sich darin erf\"{u}llen.
Ihr\st{en} einziger Zweifel, den sie wohl hegen mochte, war ja nur die Angst, die Selbst\"{a}ndigkeit ihres eigenen Wesens und damit freien Entschlusses, auch \"{u}ber ihre Liebe, verlieren zu k\"{o}nnen.
Daran war doch nur die Notzeit schuld, die mich in 24 Zeilen zu harten Worten zwingt.-

\clearpage
