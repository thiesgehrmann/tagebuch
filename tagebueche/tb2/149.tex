\def\day{9. 8. 1944.}
\mktitle

Jetzt schreibe ich etwas ganz Ver\"{u}cktes nieder.
Ich sehe es klaren Auges als Unsinn und doch k\"{o}nnte ich mich ganz an diesen traumhaften Gedanken verlieren allein aus dem brennenden Wunsch, in absehbarer Zeit vor sie hintreten zu d\"{u}rfen und sie zu fragen: Hilde, wa{\ss} h\"{a}lt Dich vor mir, die meine{\color{red} [sic] } zu sein?
Das also ist es: Man spricht ger\"{u}chteweise von Austausch in allern\"{a}chster Zeit von 14 000 Internierten und 6000 Krigsgefangenen.
Das w\"{a}re die Zahl der ersten beiden Kriegsjahre bis zum 31.8.1941 einschlie{\ss}lich.
\hline
Rasch zu etwas anderen.
Sind ihr ihre Zweifel tats\"{a}chlich allein aus einem \st{g} verletzten Gef\"{u}hl ihrer eigenen Selbstverantwortlichkeit vor ihrer Liebe gekommen, was ihre Briefe mir aber widerlegen, so will u. mu{\ss} ich den Kampf um sie wiederzugewinnen, aufnehmen.
Dann wird sie mir auch gleiche Rechte in dem Kampf zugestehen u. nichts bestehendes{\color{red} [sic] } l\"{o}sen, so lange ich ihr nicht Rede u. Antwort stehen kann.
Dass mu{\ss} ich ihr noch in dem n\"{a}chsten Brief schreiben.
Da darf sie nichts meinen Eltern entgelten u. mir auch nicht.
Ich mu{\ss} den Eltern noch schreiben, wass sie von ihr auf meinen Brief vom 7.8.44.{\color{red} [sic] } erwarten d\"{u}rfen.
Es mu{\ss} alles sehr, sehr genau erwogen werden.
Und das ist so schwehr f\"{u}r ein Herz das brennend Liebt.

\clearpage
