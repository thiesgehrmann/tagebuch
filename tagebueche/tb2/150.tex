\subsection{Canada, 27. August 1944}
%\mktitle

Das heft zur Neige.
Das was ich noch h\"{a}tte hineinschreiben k\"{o}nnen auch.
3 Wochen kam keine Post, ich hatte Ruhe und die tat mir so gut.
Ich mu{\ss} u. will doch arbeiten und lernen.
Wof\"{u}r?
Da{\ss} das Banner hochgehalten werden soll.
Ich kann nicht sagen, da{\ss} mich die Ruhe froh gemacht hatte.
Es war nur Abgekl\"{a}rtheit, wie der Brief es aussagte: Man mu{\ss} sich nur erst mal damit abgefunden haben.
Ein "feiner" Satz!!!
Ein verzicht auf alles scheint mir nun gar nicht mehr so unm\"{o}glich.
Was liegt an uns, an mir gar!
Den Glauben hat man mir mit Zweifel genommen, so ist auch die freudige Hoffnung dahin.
Bleibt die Pflicht.
Die Pflicht ist gut.-
Nun bin ich soweit, wie sie mich vielleicht haben wollten.
Ach, meine lieben guten Eltern.
Nein, sie wollten es wohl nicht.
Aber haben sie nicht auch ein wenig Schuld mit in der leidigen Kirchenfrage?
Gewi{\ss} ich gestehe ihnen allen meine eigene gr\"{o}{\ss}te Schuld ein: Ich bin zu lange von Deutschland fortgegangen.
Das ist es.
Nun darf ich zusehen, wie das Leben mir doch in den H\"{a}nden zerrinnt.
Aber die Ruhe will ich mir erhalten, auch wenn sie noch so kalt ist.
Was gehen mich die beiden Briefe, die gestern kamen, noch an?
Aus Berlin vom 3.7. nach dem 22.5. u. aus Flensburg vom 24.6. nach dem 7.5.
Nein, sie sollen mich nicht so kriegen.
Sie bleiben liegen, wie lange ist mir gleich.
Ich habe sie nicht gelesen.
Mich schaudert noch das Grauen, das nach meinem Herzen langte, als ich sie in die H\"{a}nde nahm.
Ich kann u. will nich mehr.

\clearpage
