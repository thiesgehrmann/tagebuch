\subsection{20. September 1944}
%\mktitle

Was ist nicht alles seitdem in unserem Lager passiert.
Den besten Freund hat man versetzt, zusammen mit 800 altvertrauten Gesichtern.
Der Sprachenlehrer ist aus dem Leben geschieden. {\color{red} [Dr. Uffz. Karl Lehmann: \url{http://www.findagrave.com/cgi-bin/fg.cgi?page=gr&GRid=41496308}] }
Der Unterricht war unterbrochen.
Franz\"{o}sisch soll ich nun noch als Arbeitsgemeinschaft leiten.
Mache ich gern, doch heute am ersten Tage f\"{a}llt es erstmal aus, weil \"{u}ber die H\"{a}lfte sich schon zum Boxabend abgemeldet hat. -
Ihr letzter Brief war ein kleines Bekenntnis mit dem Willen wiedergutzumachen.
Ach der dumme, dumme Peter!
Und dennoch hat sie keine rechte Entschlu{\ss}kraft.
Aufschub u. noch einmal Aufschub.
Begreift sie denn gar nicht, was es f\"{u}r mich hie{\ss}, mich ein dreiviertel Jahr in meinen Entschl\"{u}ssen und Entscheidungen von vornherein festzulegen?
Sieht sie denn nicht das \"{U}berma{\ss} von Vertrauen, das sie sich in Ehrem erhalten sollte?
Es ist alles zu dumm; wenn ich nur ein Wort ihr sagen k\"{o}nnte.
Es w\"{a}re ganz gewi{\ss} herzlich und inniglich, aber genau so herzlich erfrischend und aufweckend klar f\"{u}r sie.-
So habe ich dem Ilschen einen lieben Brief versprochen.
Mein Gott, wenn ich denke, das M\"{a}del war verheiratet, hat nun ein Kind und ist schon Witwe.
Die Erinnerung an sie ist mir lieb vertraut.
Und Gretchen schweigt.
Ich las gestern Storms "Immensee".
Die Eltern haben es als Farbfilm im Kino gesehen.

\clearpage
