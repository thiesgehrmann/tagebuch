\def\day{21. Dezember 1944.}
\mktitle

Die letzte Post vom alten Jahr ist fort.
Die Schule ist abgeschlossen.
(Wof\"{u}r? Wozu? F\"{u}r Wen?)
Das Jahr 1944 geht mit einer ungeheuerlichen Bilanz zu Ende: Meine Braut hat es f\"{u}r n\"{o}tig befunden, die von ihr selbst angeregte und von mir in Treu und Glauben eingereichte Ferntrauung auf Anraten ihrer Mutter zur\"{u}ckgehen zu lassen.
Ist das noch eine Braut?
Ist solch eine Handlung denn noch ehrlicher als die Hurerei der Soltau?
Ich wei{\ss} nicht mehr was ich von allem Halten{\color{red} [sic] } soll. -
Und dann ihre neuesten Briefe mit der Einsicht der Selbstt\"{a}uschung.
Ein Hohn!
Das ist nun mein Weihnachten.
Womit habe ich das verdient, das ein Weib alles so einfach abtun konnte?
Was soll da alles Jammern jetzt noch?
Und ich soll trotzdem glauben?
Woran denn? -
Als ich hinausfuhr, sprang mein Ring.
Wie der Februarbrief kam, klang eine d\"{u}nne Stimme: So f\"{a}ngt es an.
Als die offene Ablehnung ihm Brief kam, sprach sie schon laut und deutlich: Das ist das Ende.
Ich wollte doch so gerne glauben und schloss die ganze Sehnsucht nach der Heimat in einen Namen: Meine Hilde.
So fing ich jeden Brief an.
Nun liegt der Ring tief, tief unten.
Er brannte mir am Finger.

\clearpage
