\subsection{31. Dezember 1944.}
%\mktitle

Ich will heut noch kurz niederschreiben, was mir am Heiligen Abend begegnete.
Nach der eindrucksvollen Feierstunden in dem festlich geschm\"{u}ckten Speisesaal - es war eine reine Deutsche Weihnachtsfeier, so wie sie es {\color{red} [7,5cm sind geschw\"{a}rzt] noch gewesen ist- machten wir es uns in unseren [1 wort geschw\"{a}rzt] } \"{a}rmlichen H\"{u}tten so gut wie es eben ging auch eine Weihnachtsstunde.
Wir sa{\ss}en noch gar nicht lange beisamen, da kam ein mir lieber Gast zu besuch.
Heiz Wolf war eigens gekommen, um mich zu der bestandenen Reifepr\"{u}fung zu begl\"{u}ckw\"{u}nschen.
Ich war dar\"{u}ber so erfreut und \"{u}berrascht, das kan ich ihm nie vergesssen.
Er setzte sich auch zu uns, ich konnte zu dem Geb\"{a}ck auch etwas Obst aus meinem Weihnachtspaket auf den Tisch bringen.
So lief die Zeit in froher Unterhaltung, bis seine Frage kam, von der er ja nicht ahnen konnte, was sie in mir anrichten mu{\ss}te.
Ich war ja schon einigerma{\ss}en Ruhig geworden in letzter Zeit.
Heinz W. wu{\ss}te von meiner Ferntrauung schon seit Monteith.
Also konnte er annehmen, da{\ss} die Trauung schon vollzogen sein m\"{u}{\ss}te.
Mir stand das Herz still als er fragte: "Na und was macht die Post, wie geht's der jungen Frau?"
Ich wu{\ss}te nichts zu sagen.
Ich gab ihm nur die beiden Briefe ihrer Mutter, die er im Kerzenlicht las und dann kopfsch\"{u}ttelnd weglegte.

\clearpage
