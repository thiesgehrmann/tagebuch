\subsection{{\color{red} [31. Dezember 1944. fortgesetzt] }}
%\mktitle

Ich habe es dann sofort aufgegeben, an den beiden Weihnachtsfeiertagen noch einen Kameraden aufzusuchen.
Das war meine Weihnachtsfreude.
- Gewi{\ss} das Paket von ihr war sehr, sehr lieb gepackt und r\"{u}hrend zurechtgemacht.
Aber warum schreibt sie denn in ihren Briefen wenn sie nur wu{\ss}te womit sie mich so recht erfreuen k\"{o}nnte.
Ja, ich h\"{a}tte mich wohl riesig gefreut, wenn sie meine Frau geworden w\"{a}re.
Sie will es ja auch wieder, sogar noch eine gute und treue.
Aber mu{\ss} dazu noch erst eine Trauungsurkunde zur\"{u}ckgehenlassen? -
Wenn ich mir da den zweiten Brief ihrer Mutter ansehe, so begreife ich \"{u}berhaupt nichts mehr.
So eine Art Mitgiftj\"{a}ger sieht man in mir.
Das alles daf\"{u}r, da{\ss} ich es "gewagt hatte", in dem Augenblick, als die Ehe f\"{u}r mich als geschlossen galt, sie danach zu fragen, was sie denn nun h\"{a}tte, womit wir uns ein Heim schaffen wollten.
Eben noch beschimpft, im n\"{a}chsten Satz "unser lieber Sohn".
Wie w\"{a}r's denn, wenn die Mutter ihre Tochter davor bewart zu haben glaubte, um der Mitgiftwillen geheiratet zu werden, und nur erreichte, da{\ss} ihre Tochter ihre sch\"{o}nsten Jugendjahre somit umsonst vertan h\"{a}tte? -
Das begreife doch, wer will.
Je l\"{a}nger ich dar\"{u}ber nachdenke, desto wahnsinniger werde ich.
Wenn ich doch blo{\ss} das M\"{a}del aus meinem Herzen rei{\ss}en k\"{o}nnte!
Aber das kann ich nicht.
Wann hat das alles ein Ende?

\clearpage
